\doublespacing

\chapter[Responses of multiple cell types]{ High temporal resolution 
           of the responses of multiple cell 
           types and the necessity of toxin glucosylation }\label{chapter:imp}


\section{From transcriptomics of the epithelial-layer to the roles
           of different cell types in the epithelial layer}

\subsubsection{Localizing toxins}
The gene expression changes in vitro and in vivo showed
the many changes during pathogenesis, yet it is difficult
to determine if specific responses are a direct
reaction to toxin or a secondary response.
To help elucidate this, I aimed to characterize the location
of toxin within the intestine.
Presumably, toxins enter
epithelial cells in order to damage the epithelial barrier.
However, other cell types such as macrophages, neurons, and dendritic cells
drive inflammatory responses. Unfortunately, there is no
antibody for TcdB that has proven to work in immunohistochemistry.
Therefore, I directly labeled TcdA and TcdB with a fluor by amine conjugation.
The concentration required to detect the labeled TcdB
was approximately 1\textmugreek{}m. With this limit, much
of the labeled toxins would go undetected. Therefore, I here take
an indirect approach, looking at the responses of different cell
types to wide ranges of toxin concentrations.

\subsubsection{Precise quantification of toxin responses}
To compare the response in a reproducible and consistent manner,
I decided to track broad structural changes by measuring how electrical
impedance across the surface of a cell culture as described in this chapter.
This precise quantification system had the added benefit as a
quality control to compare the potency of labeled versus unlabeled toxin
as well as different toxin purifications.

\subsubsection{The glucosyltransferase, the primary pathogenic activity?}
The glucosyltransferase domains of TcdA and TcdB have
long been considered the primary molecular mechanisms of pathogenesis.
However, recent studies have challenged this line of thought.
In this chapter, I take advantage of the precise assay mentioned above
and the profiling of multiple cell types to undestand the necessity of
glucosyltransferase activity.


\section{Synopsis}
\textit{Clostridium difficile} toxins A and B (TcdA and TcdB), homologous 
proteins essential for \textit{C. difficile} infection, affect the behavior 
and morphology of several cell types with different potency and 
timing. However, precise morphological changes over various time scales, 
which help explain the roles of cell types, are poorly characterized. 
The toxins' glucosyltransferase domains are critical to their 
deleterious effects, and cell responses to glucosyltransferase-independent 
activities are incompletely understood. By tracking morphological 
changes of multiple cell types to \textit{C. difficile} toxins with high 
temporal resolution, newly characterized cellular responses to 
TcdA, TcdB, and a mutant, glucosyltransferase-deficient TcdB 
(gdTcdB) are elucidated.
Human umbilical vein endothelial cells, J774 macrophage-like 
cells, and four epithelial cell lines were treated with TcdA, 
TcdB, and gdTcdB. Impedance changes across cell cultures were 
measured to track changes in cell morphology. Metrics from 
impedance data, developed to quantify rapid and long-lasting 
responses, were used to build standard curves with wide dynamic 
ranges that defined cell line-specific toxin sensitivities. Except 
for T84 epithelial cells, all cell lines were more sensitive to TcdB 
than TcdA. Macrophages rapidly stretched and arborized, and then 
increased in size in response to TcdA and TcdB but not gdTcdB. 
High concentrations of TcdB and gdTcdB ($>$10 ng/ml) resulted 
in loss of intact macrophages. In HCT8 epithelial cells, 
gdTcdB (1000 ng/ml) elicited a cytopathic effect only after 
several days, yet it was capable of delaying TcdA and TcdB's 
rapid effects. gdTcdB did not delay TcdA's stimulation of macrophages.
Epithelial and endothelial cells have similar responses to toxins 
yet differ in timing and degree. Relative potencies of TcdA and 
TcdB in mouse epithelial cells in vitro do not correlate with 
potencies in vivo. gdTcdB is not entirely benign in HCT8 cells. 
TcdB requires glucosyltransferase activity to stimulate macrophages, 
but cell death from high TcdB concentrations is 
glucosyltransferase-independent. Competition experiments with gdTcdB show 
TcdA or TcdB round HCT8 cells through common mechanisms, yet 
macrophages are stimulated through potentially different pathways. 
This first-time, precise quantification of multiple cell lines 
provides a comparative framework for contextualizing previous 
research and delineating the roles of different cell types and 
toxin-host interactions.

\section{Introduction}
\textit{Clostridium difficile} infections, with an annual occurrence in the 
US of over 300,000, cause potentially fatal diarrhea and colitis \cite{Lucado:2012wl}. 
These pathologies arise from the release of two potent, homologous, 
protein toxins—TcdA and TcdB—into the host gut. The toxins' 
interactions with many cell types lead to disease, yet the relative 
sensitivities and roles of different cell types remain poorly 
understood. Both toxins disrupt the epithelial barrier by causing 
epithelial cells to round and detach \cite{Pothoulakis:2000via}. Neutrophil infiltration 
and activation of other immune cells, driven by inflammatory 
signals, are also key to toxin-induced enteritis \cite{Kelly:1994cu}. Though 
several molecular mediators of disease have been identified, 
little is understood about the host cell dynamics and the role 
of each cell type involved \cite{Sun:2010kt,Shen:2012cm}. To explore the toxins' 
effects on different cells, facets of the host response have 
been studied using cell lines treated with TcdA and/or TcdB 
(e.g., release of cytokines \cite{Sun:2010kt,Castagliuolo:1998um,Kelly:1994wd}, changes in cell 
morphology \cite{Grossmann:2000cg,Brito:2002kq}, 
gene expression \cite{DAuria:2012bd,Gerhard:2005dg}, 
and cell death \cite{Solomon:2005dv,Gerhard:2008wz}). 
Most of these assays used in previous studies are limited to 
few time points, and since both toxins affect cells rapidly 
(in less than one hour), it is unknown if either toxin has 
additional effects on finer time scales and if any of these 
effects are consistent across cell lines at comparable concentrations.

We and others have tracked temporal changes in cell 
morphology and attachment in response to TcdA or TcdB 
by continuously measuring electrical impedance across 
the surface of a cell culture \cite{He:2009hg,Ryder:2010jo,DAuria:2013jo}. When cells 
grow or increase their footprint or adherence, impedance 
rises. In contrast, cell rounding, shrinking, and/or 
death correspond to decreased impedance. This assay 
has primarily been used as a sensitive diagnostic—as 
a more quantitative replacement of assays that are 
dependent on visualization of cell rounding. In 
this study, we recognize that this impedance data, 
in addition to indirectly detecting the amount of toxin 
in samples, can further be analyzed to reveal previously 
unrecognized, dynamic responses of host cells. Our analyses 
and associated metrics also allow precise comparisons 
between the effects of TcdA and TcdB and between different 
cell types. Using epithelial and endothelial cells, 
these analyses identify characteristics such as the minimal 
effective toxin concentrations and the shortest time to 
measurable toxin effects; standard curves with wide 
dynamic ranges can also be derived. Impedance changes 
of other cells, such as macrophages, are not as easily 
linked to known cell functions, but the data reveal toxin 
effects that would not otherwise be observed at lower 
temporal resolution. This knowledge contextualizes the 
potential roles and relative abilities of different cell 
types to respond directly to toxin during an infection. 

Impedance curves that profile cell responses also provide 
insight into the toxins' molecular functions. TcdA and 
TcdB have glucosyltransferase domains that inactivate small 
GTPases. With the use of engineered mutant toxins, 
glucosyltransferase activity has been found necessary 
for cell rounding \cite{Teichert:2006jo}. However, evidence that some 
glucosyltransferase-deficient mutants of TcdB (gdTcdB) 
are cytotoxic has raised questions about whether there 
are other, previously unknown toxin activities \cite{Chumbler:2012co}. In 
order to identify changes dependent and independent of 
glucosyltransferase activity, we use gdTcdB to evaluate 
the dynamics of the response of macrophage and epithelial 
cell lines to gdTcdB, elucidating changes dependent and 
independent of glucosyltransferase activity. We also 
leverage the unique response profiles to TcdA, TcdB, 
and gdTcdB in order to investigate synergy or antagonism 
between toxins.

The cell response profiles define the dynamics of basic 
changes in cell physiology (e.g., cell rounding) across 
multiple cell types in response to TcdA, TcdB, and gdTcdB. 
This understanding identifies those times most representative 
of the entire cell response, delineates the contribution 
of glucosyltransferase activity to overall toxin effects, 
and suggests the relative roles of various cell during 
toxin-mediated disease.

\section{Methods}

\subsection{Cell Culture}
HCT-8 cells were cultured in RPMI-1640 supplemented with 10\% 
heat-inactivated fetal bovine serum (HI-FBS) and 1 mM 
sodium pyruvate. J774A.1 cells were cultured in DMEM high 
glucose media supplemented with 10\% HI-FBS, 1 mM, and 
MEM nonessential amino acids (Gibco 11140). HUVEC cells 
(passage 3) were cultured in endothelial growth medium 
(EGM-Bullet Kit CC-3124, Lonza group). T84 cells were 
grown in an equal mixture of Ham's F12 and Dulbecco's 
modified Eagle's media supplemented with 2.5 mM L-glutamine 
and 5\% HI-FBS. All cells were incubated at 
37\textdegree{}C/5\% \ce{CO2}. In our analyses, we include 
our previous data from immortalized, mouse, cecal 
epithelial cells (hereon referred to as IMCE cells) 
which were derived by Becker et al. and incubated at 
33\textdegree{}C as described by Becker et al. \cite{DAuria:2013jo,Becker:2010io}. 
TcdA and TcdB, isolated and purified from strain VPI-10643, 
were a generous gift from David Lyerly (TECHLAB Inc., Blacksburg, VA). 
Recombinant gdTcdB and TcdB were a generous gift from the 
laboratory of Aimee Shen.

\subsection{Electrical impedance assay}
Impedance was measured using the xCELLigence RTCA 
system (ACEA Biosciences), which consists of an RTCA 
DP Analyzer and 16-well E-plates. PBS was added around 
all wells to prevent evaporation. In each well, 100 \textmugreek{}L 
media was incubated at room temperature for 30 minutes, 
and one baseline reading was taken. Cells in 100 \textmugreek{}L 
media were then added and allowed to settle at room 
temperature for 30 minutes. Plates were then moved inside 
the RTCA DP Analyzer inside a \ce{CO2} incubator at 37\textdegree{}C. 
Subsequent readings were taken at frequencies ranging between 
every 4 seconds to every 10 minutes, with higher frequency 
measurements reserved for times directly before toxin 
addition to at least 6 hours after addition (complete 
protocols and data files available in the Supplemental Material).

Since the impedance measurements are sensitive to 
slight movements or vibrations, the method by which 
toxin was added to cells was an important consideration. 
In our initial experiments, mechanical agitation and 
replacement of media sometimes caused small, sharp 
spikes in electrical impedance. To minimize disturbances, 
plates were not removed from the RTCA Analyzer once 
seated. Toxins prepared in media (10x) were gently 
pipetted using only one to two depressions. Media was 
not replaced after the addition of toxin.

\subsection{Analyses}
The protocols, data, computer code, and instructions 
for running the code that reproduce all results and 
figures are provided in the \autoref{apx:imp}.


\section{Results}

\subsection{Quantification of the cytopathic effects 
              elicited by TcdA and TcdB}
In order to assess the cytopathic effects of TcdA and TcdB, 
we measured changes in impedance across the surface of 
electrode-embedded wells (Methods). Impedance is dependent 
upon cell number, adherence, and morphology. It increases 
as cells proliferate or spread and decreases when toxin is 
added and cells round up (\autoref{imp:fig1}). The rate at which impedance 
decreases is dependent on the toxin, toxin concentration, and cell 
type (\autoref{imp:fig2}A and \autoref{imp:fig2}B). To summarize the data-rich 
``impedance curves'', we calculated simple metrics: the 
area between the curves of control and toxin-treated cells 
(ABC, gray area in inset of \autoref{imp:fig1}), the maximum slope of 
a curve (MaxS), and time for a curve to decrease by 50\% (TD\textsubscript{50}, 
\autoref{imp:fig1}). A negative ABC indicates that the impedance curves 
of toxin-treated cells are below the curves of untreated 
cells. Blue, dashed lines in Figure 2C show the variability 
of the ABC of control cells from their average impedance 
curve. Standard curves relating TD\textsubscript{50} to toxin concentration 
have been generated before \cite{Xu:2012db}, and we found that our 
metrics, ABC and MaxS, also produce log-linear calibration 
curves (\autoref{imp:fig2}C). Among replicates, differences in 
timing translated to differences in TD\textsubscript{50}, as expected, 
whereas MaxS values were more similar. In simpler terms, 
for replicates within and between experiments, the time 
required to observe a change in impedance was more 
variable than the rate of the change. For this reason, 
we found that MaxS, instead of TD\textsubscript{50}, better quantified 
rapidity of the cell response to high toxin concentrations. 
The other metric, ABC, captures long-term effects by 
integrating readings over several hours. The minimal 
concentration to induce a change in impedance from control 
is denoted as the minimal cytopathic concentration 
(MCC; \autoref{imp:fig2}C). When ABC and MaxS are considered together, 
toxin concentration can be determined with a dynamic 
range spanning six orders of magnitude or more (depending 
on toxin and cell type). Together, these metrics allow 
for millions of data points and hundreds of wells to be 
simultaneously visualized and summarized to dozens of numbers 
or fewer that can be easily interpreted (e.g., \autoref{imp:fig2}D 
and \autoref{apx:imp}).

\begin{figure}[h!]
  \centering
  \includegraphics[width=\textwidth]{imp/Figure1.png}
  \caption[Measurement of toxins' cytopathic effects by tracking 
            electrical impedance across the surface of a cell culture]{
       \textbf{Measurement of toxins' cytopathic effects by tracking electrical 
             impedance across the surface of a cell culture.}
       All impedance readings were normalized to the impedance at 
       the time toxin was added. Shaded regions above and below lines 
       represent the standard deviation of technical replicates (n=2). 
       Readings were taken as quickly as every four seconds (Methods). 
       The brightness of each photograph was adjusted digitally 
       (uniformly across an entire photograph) to make the overall 
       brightness across all photographs similar.
  }
  \label{imp:fig1}
\end{figure}

\begin{figure}[h!]
  \centering
  \includegraphics[width=0.8\textwidth]{imp/Figure2.png}
  \caption[SQuantification of cytopathic effects]{
       \textbf{Quantification of cytopathic effects. }
       \textbf{(A and B)} The cytopathic effects between cell types and 
       toxins can easily be distinguished. \textbf{(C)} The impedance 
       curves can be analyzed to produce two metrics, ABC and 
       MaxS, which can then be used to define the minimal 
       cytopathic concentration (MCC). \textbf{(D)} The MCC of TcdA and 
       TcdB for five cell lines define cell line specific sensitivities.
  }
  \label{imp:fig2}
\end{figure}

\subsection{Epithelial and endothelial cells: similar characteristic responses
              but different sensitivities to TcdA and TcdB}
In the first set of comparisons, we chose four well-characterized 
cell types or cell lines—one endothelial (HUVECs) and three 
epithelial (CHO, HCT8, and T84)—and one immortalized, cecal, 
mouse epithelial cell line (IMCE, see Methods). For these 
five cell lines, the MCC for TcdA and TcdB varied over ranges 
of 0.1-1 ng/ml and 0.1-100 pg/ml, respectively (\autoref{imp:fig2}D). 
We did not find a maximal effective concentration of either 
toxin (1 μg/ml was the highest concentration tested). TcdB 
was consistently 100-1000 times more potent than TcdA, except 
in T84 cells, which were equally sensitive to TcdA and 
TcdB (as measured by MCC). The curves were largely similar 
in that they all consisted of a short delay followed by a 
sharp decrease that then leveled off (\autoref{imp:fig2}A); differences 
were primarily in scale. Determining the time to the onset 
of the first toxin effects was complicated slightly by the 
physical process of adding toxins to wells—a process which 
caused disturbances that temporarily affected impedance 
(note the early ``bump'' in \autoref{imp:fig2}A). Nevertheless, differences 
between control and toxin-treated cells can be distinguished. 
Across all cell types, the time required for an impedance 
curve to diverge from control was more than ten minutes. 
Nothing clearly suggested an immediate response to toxin binding. 
Morphological changes might not occur until after toxins 
enter cells and glucosylate Rho proteins. We next examined 
early effects of toxins on macrophages and investigated the 
contribution of glucosyltransferase activity to the dynamics 
of cell responses.

\subsection{Macrophages: rapid, sensitive, complex concentration-dependent 
              responses to TcdA and TcdB}
J774 mouse macrophages were as sensitive and responsive to 
TcdA and TcdB as epithelial cells. The impedance of 
macrophages treated with TcdA (300 ng/ml) and TcdB (10 ng/ml) 
diverged from controls in 10 and 20 minutes, respectively 
(\autoref{apx:imp}). In contrast to epithelial cells, however, the 
impedance of macrophages increased after toxin addition (\autoref{imp:fig3}), 
and the responses of J774 cells to TcdA and TcdB differed in 
shape and scale. TcdA caused a rise in impedance at 0.1 ng/ml, 
and the magnitude and speed of this rise increased until 
TcdA concentration reached 100 ng/ml (\autoref{imp:fig3}A). At 
higher concentrations (300 and 1000 ng/ml), the slope of 
the rise continued to increase, yet the rise was inhibited, 
as if stopped prematurely before reaching its peak, and 
then impedance dropped below that of control cells (Figure \autoref{imp:fig3}A
and \autoref{apx:imp}). Considering now TcdB, 0.1, 1, and 10 ng/ml 
caused impedance to rise and stabilize at approximately double 
the initial value; only the slope of the rise (not the final 
height) was affected by toxin concentration (Figure 3A and 
\autoref{apx:imp}). These curves allowed us to resolve time 
points that would be of most interest for cell imaging. 
Increases in impedance, for TcdA and TcdB, correlated 
with rapid stretching and arborization of cells (appearance 
of many filopodia in \autoref{imp:fig3}B) that suggest macrophage 
activation. Over the next 48 hours, cells increased in size 
and became more circular (\autoref{imp:fig3}B). Subsequently for TcdA, 
decreased impedance correlated with a decrease in intact 
cells (\autoref{imp:fig3}B). These results support complex 
dynamic responses of J774 cells to different toxin 
concentrations or a response driven by two or more 
cellular functions (e.g. activation and apoptosis are 
known to occur in monocytes and macrophages in response 
to TcdA and TcdB \cite{Solomon:2005dv,Siffert:1993ue,Linevsky:1997wt,MeloFilho:1997ws}.

\begin{figure}[h!]
  \centering
  \includegraphics[width=\textwidth]{imp/Figure3.png}
  \caption[Macrophage responses to TcdA and TcdB]{
       \textbf{Macrophage responses to TcdA and TcdB}
       \textbf{(A)} Impedance curves from a selection of toxin 
       concentrations for TcdA and TcdB. Both graphs 
       represents one multi-well experiment where confluent 
       cells were treated with toxin. 
       \textbf{(B)} Pairing of impedance data with photographs 
       to show the morphological changes represented 
       in the impedance data. Since wells with electrodes 
       are opaque, technical replicates in transparent 
       wells were used for microscopy. Sub-confluent 
       cultures were used so that structural changes 
       in individual cells could more easily be observed.
  }
  \label{imp:fig3}
\end{figure}

At 100, 300, and 1000 ng/ml of TcdB, the impedance curves were 
entirely different than lower concentrations. Instead of 
rising, impedance fell (see loss of intact cells in bottom 
row of images of \autoref{imp:fig3}B). Hence, at a concentration 
between 10 and 100 ng/ml, the response of macrophages to 
TcdB switches from cell stretching to a degradation of 
normal cell structure. We hypothesized that the 
glucosyltransferase activity of TcdB, when at or above 
100 ng/ml, is not necessary to induce the loss of intact 
cells. To investigate this, we used gdTcdB. First, to 
better understand the effects of gdTcdB, we examined its 
ability to induce the well-known cytopathic effects of TcdA 
and TcdB in epithelial cells.

\subsection{Glucosyltransferase-deficient TcdB alters the effects of TcdA and TcdB
              on macrophages and epithelial cells}
Since the cytopathic effects of TcdA and TcdB have been 
attributed to their glucosyltransferase activities, we expected 
that gdTcdB would not cause cell rounding. Indeed, the 
impedances of HCT8 cells treated with gdTCB (100 and 1000 ng/ml) 
and untreated cells were indiscernible in the first ten 
hours after toxin addition (\autoref{imp:fig4}A). However, gdTcdB 
caused an unexpected slow rise in impedance above that of 
control cells (\autoref{imp:fig4}A). During this rise, imaging revealed 
continued growth, elongation, and close apposition of 
untreated cells, while cells treated with 100 or 1000 ng/ml 
of gdTcdB rounded slightly but remained attached (\autoref{imp:fig4}A). 
The increased impedance elicited by gdTcdB was followed by 
a slow decrease towards that of TcdB-treated cells. This 
decrease, which took more than six days, was due to 
detachment of cells and disruption of cell morphology 
(\autoref{imp:fig4}A). Hence, though gdTcdB does not round 
cells quickly as with TcdB, it still has unexplained, 
slow effects that alter the morphology of HCT8 epithelial cells.

\begin{figure}[h!]
  \centering
  \includegraphics[width=\textwidth]{imp/Figure4.png}
  \caption[Response of HCT8 epithelial cells to gdTcdB, 
             TcdA+gdTcdB, and TcdB+gdTcdB]{
       \textbf{Response of HCT8 epithelial cells to 
               gdTcdB, TcdA+gdTcdB, and TcdB+gdTcdB}.
       \textbf{(A)} Impedance curves of HCT8 cells treated with 
       gdTcdB and corresponding photographs. 
       \textbf{(B)} HCT8 cells 
       treated with TcdB or gdTcdB and TcdB in combination. 
       \textbf{(C)} HCT8 cells treated with TcdA or gdTcdB 
       and TcdA in combination. The three graphs are 
       from the same multi-well experiment but are 
       representative of three independent experiments.
  }
  \label{imp:fig4}
\end{figure}

To investigate if TcdA and TcdB have overlapping activity, 
we performed experiments with gdTcdB plus TcdA or TcdB. We 
anticipated that gdTcdB would attenuate or delay the effects 
of TcdB and perhaps TcdA. Indeed, a tenfold excess gdTcdB 
delayed the onset of the effects of TcdA and TcdB in HCT8 
cells (\autoref{imp:fig4}B and \autoref{imp:fig4}C). 
Competition for shared substrates 
(Rho family proteins) of gdTcdB with TcdA and TcdB likely 
account for the delay, although other factors such as 
shared receptors may be responsible.

We next determined glucosyltransferase-dependent toxin effects 
on J774 macrophages. Low gdTcdB concentrations did not cause 
macrophages change their morphology as did TcdB (\autoref{imp:fig5}A). 
However, gdTcdB at 100 and 1000 ng/ml resulted in a loss 
of intact macrophages, similar to TcdB at 100 ng/ml 
(\autoref{imp:fig5}A). Hence, glucosyltransferase activity is 
required for macrophage stretching and arborization 
but not required for the loss of intact cells for 
concentrations of TcdB at or above 100 ng/ml.

\begin{figure}[h!]
  \centering
  \includegraphics[width=\textwidth]{imp/Figure5.png}
  \caption[Response of J774 macrophages to gdTcdB, 
             TcdA+gdTcdB, and TcdB+gdTcdB]{
       \textbf{Response of J774 macrophages to 
               gdTcdB, TcdA+gdTcdB, and TcdB+gdTcdB}.
    \textbf{(A)} Impedance curves of J774 cells treated with gdTcdB 
    and corresponding photographs. Concentrations 
    at or below 10 ng/ml are denoted as ``low'', and 
    other concentrations are denoted as ``high''. This 
    data is derived from the same experiment shown in 
    Figure 3B. 
    \textbf{(B and C)} J774 cells treated with TcdB; gdTcdB and 
    TcdB in combination; or gdTcdB and TcdA in 
    combination. The data in the three graphs are 
    derived from three independent experiments.
  }
  \label{imp:fig5}
\end{figure}


Since TcdA and TcdB caused different effects at and 
above 100 ng/ml, we hypothesized that TcdA and TcdB 
have one or more distinct activities in macrophages. 
As expected, gdTcdB delayed the effects of TcdB on 
J774 cells (\autoref{imp:fig5}B). However, gdTcdB did not clearly 
attenuate or delay the response of J774 cells to TcdA, 
suggesting that the prominent responses to TcdA and 
TcdB in these cells are due to distinct toxin activities 
or substrates (\autoref{imp:fig5}C).

\section{Discussion}
In this study we systematically profiled the dynamic 
responses of epithelial, endothelial, and macrophage 
cell lines to TcdA and TcdB, revealing relative 
sensitivities and complex concentration-dependent 
cell responses. While comparing results from different 
experimental systems is difficult, our data have allowed 
quantitative comparisons between cell types and between 
toxins under similar conditions, distinguishing which cell 
types may respond most quickly or most intensely when 
exposed directly to toxins. The impedance ``response profiles'' 
provide continuous readouts representing external changes 
in morphology and adherence that occur from several 
possible functions within the cell. We began to explore 
the mechanisms of these changes by using glucosyltransferase 
deficient TcdB (gdTcdB), revealing which molecular 
functions of the toxin contribute to different aspects 
of response profiles. The response profiles also raise 
many questions about the mechanisms for the novel differences 
we observed. Although addressing each of these in detail 
is beyond the scope of this study, we highlight, in the 
following text, the findings that bring about these questions, 
discuss their relevance to previous studies, and so explain 
how they improve our current understanding of host cell 
responses to TcdA and TcdB.

The cytopathic effects of TcdA and TcdB that led to their 
discovery are still used as the gold standard diagnostic 
for infection \cite{Larson:1977th,Planche:2011fm}. Since most cytotoxicity assays 
are endpoint assays, the kinetics of these effects that 
are key to scientific research and clinical practice have 
not been characterized. With a continuous assay, we 
were better able to observe immediate effects of toxin. 
Although toxins may interact immediately with the toxin 
surface, the morphological differences (represented by 
impedance) occurred after a delay of ten minutes or more. 
Since TcdA (2.65 \textmugreek{}g/ml) has been found to enter 
HT29 cells in 5-10 minutes, the delay we observed is likely 
because toxins must enter HCT8 cells to alter their morphology \cite{Olling:2011es}.

Epithelial and endothelial cell lines had the same 
characteristic changes in morphology, yet the rapidity 
of the changes distinguished different cell types, 
toxin concentrations, and TcdA versus TcdB. These differences 
could be summarized by condensing the data into metrics 
that represented the greatest rate of the change (MaxS) 
and the cumulative amount of change over several hours 
(ABC). When these metrics are considered together, standard 
curves over many orders of magnitude can be used to 
measure toxin concentration and determine the minimal 
amount of toxin necessary to induce an effect (MCC, Figure 2D). 
The CHO cell line was second-most sensitive to TcdB, making 
CHO cells a good choice for toxin detection. Indeed, a 
modified CHO cell line was used in the development of an 
ultrasensitive assay of toxin activity \cite{He:2009hg}. T84 cells, 
the least sensitive to TcdB, were similarly sensitive to 
TcdA and TcdB, as has been found previously \cite{ChavesOlarte:1997cs}. For 
TcdB, the two rodent cell lines (CHO and IMCE) were more 
sensitive than the three human cell lines (HCT8, HUVEC, 
and T84), although more cell lines would be needed to 
confirm any species-specific sensitivity. For TcdA, 
cell line sensitivities were less variable than for TcdB, 
indicating that factors that make cells vulnerable to 
TcdA may be more consistent between cell lines.

Comparisons between TcdA and TcdB have often been a 
prominent research focus. TcdB is more cytotoxic in cell 
culture; TcdA is more enterotoxic in animal intoxication 
models \cite{DAuria:2013jo,Lyerly:1985dx}; and there are varying results about which 
toxin is essential for \textit{C. difficile} infection \cite{Kuehne:2010hv,Lyras:2009jx}. 
Identifying which toxin contributes most to disease helps 
prioritize therapeutics. Comparisons between toxins are 
also valuable scientific tools. Differences in the toxins' 
effects provide clues about their molecular activities. 
Also, by correlating differences in host cell responses 
to differences in disease severity, particular cell 
types or toxin activities can be prioritized. For instance, 
TcdA is more enterotoxic than TcdB in mice and hamster 
ceca, damaging the epithelial barrier \cite{DAuria:2013jo,Libby:1982wm}. This 
agrees with findings that TcdB binds weakly in the hamster 
intestine, and TcdA binds epithelial cells \cite{Rolfe:1991vx,Keel:2007jh}. 
One might then expect that cecal epithelial cells from 
mice of the same genetic background as those used in the 
aforementioned in vivo studies (IMCE cells) would be 
more sensitive to TcdA than TcdB. Instead, IMCE cells 
were over 100 times more sensitive to TcdB than TcdA, 
suggesting that factors in addition to the cytopathic 
effects on epithelial cells are important in explaining 
the pathologies of toxins in vivo. The extracellular 
environment or other cell types may be the key mediators 
determining disease severity.

Macrophages are likely exposed to toxin after epithelial damage 
and play an important part in disease, changing morphology 
and releasing molecules that exacerbate inflammation \cite{Linevsky:1997wt,Rocha:1998ua}. 
Previous studies have quantified the viability either TcdA- 
or TcdB-treated macrophages at one or two time points \cite{Siffert:1993ue,MeloFilho:1997ws,He:2009bs}. 
We characterized the effects of both toxins on macrophages, 
and the other cell types already presented, over many more 
concentrations and time points. J774 macrophages, HUVECs, 
and epithelial cells had similar sensitivity to toxins, 
indicating that all may be affected directly by toxins 
during disease. TcdA or TcdB rapidly stretched and 
arborized macrophages, which was reflected in increased 
macrophage impedance. However, the timing and 
concentration-dependent effects of TcdA and TcdB were 
different, as discussed below.

TcdA increased macrophage impedance, and a subsequent 
decrease from the peak of the increase towards the impedance 
of control cells correlated with a loss of intact cells. 
This agrees in part with Melo Filo et al. who reported 
that TcdA and TcdB killed 30\% and 60\%, respectively, 
of primary mouse macrophages (1 \textmugreek{}g/ml at 24h) \cite{MeloFilho:1997ws}. The 
balance of activation and death may therefore account for 
the rise and fall of impedance of TcdA-treated macrophages. 
At 100 ng/ml (the concentration at which the rise in 
impedance was greatest), two effects appear to be balanced. 
At higher concentrations, the stimulatory effect that 
raised the impedance occurred more rapidly but did not 
reach the same height, indicating that higher concentrations 
move the balance away from stimulation towards death and 
decreased adherence.

TcdB caused two distinct responses in J774 macrophages: 
stretching (with ``low'' concentrations at or below 10 ng/ml) 
or a loss of intact structure (with ``high'' concentrations 
at or above 100 ng/ml). Siffert et al. showed TcdB-treated, 
human macrophages arborize with little loss of viability 
(1\textmugreek{}g/ml at 3h and 24h) \cite{Siffert:1993ue}. This arborization 
corresponds with the morphological responses of J774 
macrophages to low TcdB concentrations. It is possible 
that TcdB also causes two distinct response in human 
macrophages, but Siffert et al. only reported results at 
one concentration. Although much remains to be determined 
about the mechanisms of these effects, we have identified 
new characteristics of the dynamic responses of 
macrophages and these effects help to explain the 
role of macrophages during disease. In the intestine, 
macrophages likely respond to several signals begun during 
intoxication, and given their high sensitivity, may 
also respond directly to toxin in the intestine. Early 
stimulation of macrophages may contribute to acute 
inflammation, while eventual death correlates with 
macrophage depletion and neutrophil accumulation in 
\textit{C. difficile} associated diarrhea \cite{Johal:2004jq}.

The cell responses described above prompted questions about 
toxin mechanisms. For instance, microinjection of TcdB's 
glucosyltransferase domain is sufficient to induce 
cytopathic effects, yet are there changes in cell structure 
independent of glucosyltransferase activity \cite{Rupnik:2005dq}? We found 
that 100 or 1,000 ng/ml of gdTcdB raised the impedance of 
epithelial cells above controls two days after toxin 
addition, and this difference was observed by tightly 
packed, visibly distinct control cells versus a smoother 
monolayer and slightly rounded gdTcdB-treated cells. The 
mechanisms for these differences are unclear. It is possible 
another toxin activity is unmasked when the strong 
cytopathic activity is removed or that the mutant 
glucosyltransferase affects substrates differently. After 
several days, gdTcdB causes cytotoxic or cytopathic effects. 
Chumbler et al. found that glucosyltransferase mutants were 
cytotoxic to HeLa cells after only 2.5h \cite{Chumbler:2012co}. The different 
cell types (HCT8 versus HeLa) and different glucosyltransferase 
mutants may account for the differences in timing. It 
is also possible that any residual glucosyltransferase 
activity of the mutant toxin is not revealed until 
several days after treatment. The effects of mutant 
toxins have never been assessed over such long time 
scales with such great sensitivity.

The relatively benign effects of gdTcdB in the first 
hours after addition to HCT8 cells allowed us to 
investigate the effects of gdTcdB in combination with 
TcdA and TcdB. Since TcdA and TcdB are homologous, one 
might expect that gdTcdB should interfere with TcdA. 
Indeed, gdTcdB delayed the cytopathic effects of TcdA 
and TcdB. A first interpretation of this result is 
that TcdA and TcdB compete for cell entry. However, 
two studies using truncated toxins found that (1) 
the C-terminal domain (which is believed to be 
necessary for toxin internalization) of TcdA does 
not inhibit the effects of TcdB and (2) the TcdB 
C-terminus inhibits neither TcdA or TcdB-induced 
cell rounding \cite{Frisch:2003jo,Dingle:2008gu}. Hence, at the point of cell 
entry, TcdA and TcdB likely do not interfere with one 
another. Since the glucosyltransferase domains of TcdA 
and TcdB have many of the same Rho-family proteins as 
substrates, another interpretation is that the toxins 
compete after internalization. In this scenario, our 
results would indicate that gdTcdB is processed by 
the host cell, and its glucosyltransferase domain is 
still capable of binding Rho proteins or is at least 
close enough to interfere with TcdA. Although 
gdTcdB-mediated changes have the potential to reveal 
interesting mechanisms independent of glucosyltransferase 
activity, the results overall confirm the central 
role of the glucosyltransferase domains in eliciting 
the rapid, full effects of TcdA and TcdB in epithelial 
cells. However, as described later, glucosyltransferase 
activity may not be required for all toxin effects in 
all cell types.

gdTcdB often delayed the onset of cytopathic effects 
by one hour or less. Without high temporal resolution, 
we would have likely missed the time window in which 
TcdB+gTcdB was different than TcdB alone. This has 
implications in other studies that wish to identify 
other host factors that enhance or attenuate toxin 
effects. Without precisely tracking changes in cell 
structure, several potential inhibitors of toxin 
effects could be missed.

Since macrophages detect a variety of antigens, 
one might expect that the responses to toxin might 
not be entirely dependent on glucosyltransferase activity.
The stretching of macrophages with low TcdB 
concentrations required glucosyltransferase activity. 
However, high TcdB concentrations destroyed intact 
macrophage structure by an unknown, 
glucosyltransferase-independent mechanism.

The high sensitivity of epithelial cells, endothelial cells, 
and macrophages to TcdA and TcdB suggests that all of 
these cells could be damaged by direct toxin interaction 
in the host. However, the amount and location of toxin 
during infection is very poorly understood. With 
sensitivities of cells reaching as low as 100 pg/ml, 
tracking toxins by immunohistochemistry is technically 
challenging. Antibody labeling has only detected toxin 
on fixed, toxin-treated tissues with concentrations 
greater than 1 \textmugreek{}g/ml \cite{Keel:2007jh}. Assessing sensitivities 
in vitro provides an indirect measure of the roles of 
different cell types in isolation. In addition to their 
direct effects, TcdA and TcdB initiate a cascade of 
deleterious events involving multiple cells. Neuronal 
signals have been implicated in beginning the disease 
process, stimulating mast cells or macrophages that may then 
recruit other cells 
\cite{Sorensson:2001da,Pothoulakis:1998vca,
Castagliuolo:1998fu,Castagliuolo:1994ta}. 
Neutrophil infiltration is a 
hallmark of intoxication, yet neutrophils in vitro require 
much higher toxin concentrations than all other cell types 
to be affected or recruited (>1 μg/ml) 
\cite{Kelly:1994cu,Brito:2002kq,Shah:1991ww,Pothoulakis:1988dk,Dailey:1987vo}. 
Hence, it is thought that neutrophils may be primarily 
recruited by signals secondary to toxin 
damage \cite{Kelly:1994cu,Sun:2010kt,Voth:2005di}. 
To confirm the low toxin-sensitivity of neutrophils, we 
did attempt to measure impedance changes of neutrophils 
in response to toxins, yet the variability in these 
primarily non-adherent cells (impedance largely measures adherence) 
was too high to identify differences (\autoref{apx:imp}). 
Elements of the toxin responses of other cell types 
(e.g., mast cells \cite{Meyer:2007kj,Calderon:1998tr,Gerhard:2011hm}, 
dendritic cells \cite{Lee:2008jf,Jafari:2013ji}, 
neurons \cite{Xia:2000gg,Neunlist:2003ba}, 
fibroblasts \cite{Wedel:1983vd,ChavesOlarte:1996jy}, etc.) have been 
studied, yet the dynamics of their responses—and in many 
cases concentration-dependent effects—are unknown. In the 
future, precisely capturing the time and concentration-dependent 
responses to TcdA and TcdB will better contextualize their 
potential roles in the host. Our analyses of endothelial 
cells, epithelial cells, and macrophages in the same 
experimental framework set a precedent for such comparisons. 
Furthermore, we show how data from sensitive, continuous 
assays, could be used to gain insight into cell function 
and molecular mechanisms and generate new hypotheses. The 
framework and simple analyses may also be used to investigate 
synergy, antagonism, or interactions between bacterial toxins 
and other host factors that affect cells over a wide range of 
time scales.


\section{Acknowledgements}
This chapter is a manuscript in preparation, and I would
like to thank all co-authors: Meghan Bloom, Yesenia Reyes,
Mary Gray, Edward van Opstal, Jason Papin, and Erik Hewlett.
We are exceedingly grateful to David Lyerly and 
TECHLAB Inc. for providing purified TcdA and TcdB 
as well as Aimee Shen for providing recombinant TcdB 
and gdTcdB.




