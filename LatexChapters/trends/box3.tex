\begin{pabox}[label=trends:box:caveats,float*=h!,width=\textwidth]{Knowledge gaps and caveats to metabolic network analysis}
\small
\begin{multicols}{2}
Multiple steps in the model-building process and subsequent 
analyses are prone to errors that may greatly affect \gls{flux} 
and growth predictions and, consequently, predicted drug 
targets. \gls{fba} provides a quality-assurance check that ensures 
\gls{massbalance}. Growth rates and gene essentiality predictions 
are validated against experimental data to ensure the 
model truly reflects biological processes. Gene essentiality 
predictions (commonly between 55\% and 90\%) provide 
confidence that downstream analyses are based on a 
high-quality model \cite{Feist:2007dq,Jamshidi:2007ei}. To ensure the usefulness of 
any computational pipeline, drug target predictions 
should be compared to known targets from the literature. 
Below we discuss more of the particular difficulties and 
limitations encountered when using metabolic models to 
identify drug targets.

\subsection*{Genome annotation}
In any metabolic \glssymbol{reconstruction}, there may be 
hundreds of putative metabolic enzymes with no 
experimentally identified function. These reactions 
and associated \glspl{gpr} may be assembled 
strictly based on existing functional annotations 
of the genome or based on evidence from related 
organisms. Even some very well characterized enzymes 
may have other unexpected activities. In such cases, 
misannotated enzymes may yield incorrect model 
predictions leading to errors in drug targeting. For 
instance, an enzyme may be incorrectly predicted to be 
essential if the activity of another enzyme, which is 
not included in the network, can account for the same function. 
Therefore, this represents one important knowledge gap in 
the assembly of metabolic networks, and the inclusion of 
more refined enzyme annotations will directly improve 
drug target predictions \cite{Hsiao:2010bu,Szappanos:2011gu}.

\subsection*{Nutrient availability}
An important challenge in reconstructing and modeling 
metabolic networks is determining the composition of 
\gls{insilico} media. Nutrients available in host environments 
are poorly characterized. In addition, for any nutrients 
that are identified, quantitative data on uptake rates 
are unavailable. Knowledge of the transporters of an 
organism elucidates which metabolites are transported 
into the cell. However, information on transporters 
is particularly limited and, in general, transport 
reactions lack any experimental evidence or gene 
associations supporting their presence. Instead, 
transport reactions are added for proper functioning 
of the computational model. Because model predictions are 
dependent on the media environment, nutrients and 
transporters must be carefully defined.

\subsection*{Objective function}
As stated previously, in \gls{fba}-based modeling of microbial organisms, 
a \gls{biomass} has often been used. The purpose of 
the reaction is to ensure a drain of metabolites that 
are deemed essential to support the growth of the 
organism. Starting with the estimated weight fraction 
of important macromolecular components of the cell 
(e.g.\ protein, lipid, RNA, DNA and carbohydrate), the 
relative abundance of metabolites comprised in each 
group (e.g.\ amino acids, phospholipids, nucleotides) 
can be computed \cite{Feist:2010hq}. Among several available, a few 
experimental methods to measure \glssymbol{biomass} components 
include chloroform–methanol extraction (lipids), 
colorimetric protein assays, and gas chromatography–mass 
spectrometry (protein content) \cite{Boyle:2009jr}. \Glspl{flux} measured 
directly by metabolic flux analysis, which tracks 
the movement of 13C from an initial 13C-labeled 
substrate, may also help to define the \glssymbol{objectivereaction} 
function \cite{Chen:2011dj,Blank:2005bt}.

~~~~~~An \glssymbol{objectivereaction} function is most likely to cause 
errors if metabolites are entirely missing or 
incorrectly included; the relative amounts of each 
metabolite in the \gls{objectivereaction} (i.e.\ the 
stoichiometric coefficients) do not greatly affect 
\gls{fba} results \cite{Varma:1993hd,Varma:1994tb}. Therefore, logically 
deducing a \gls{biomass} is often adequate 
to estimate the growth of an organism. However, 
additional experimental data can reveal interesting 
peculiarities that may be used to design a specific 
\gls{biomass} for a particular organism. 
Data on growth- and non-growth-associated ATP 
maintenance can be included in the \gls{biomass} \cite{Forster:2003jb}. Under in vivo conditions when a 
pathogen is interacting with its host, an aspect 
that is often unclear is which \glssymbol{biomass} component(s) 
to include in the reaction. The composition of the 
\gls{biomass} is likely to vary under different 
physiological conditions, and the choice of metabolites 
can directly influence model predictions regarding 
drug targets. For example, failure to include a 
particular cell-wall component will not necessarily 
direct \gls{flux} through reactions that may be crucial in 
vivo, and therefore the associated enzymes will not 
be targeted.
\end{multicols}
\end{pabox}
