
\chapter{ Future directions: toxin responses and metabolism }\label{chapter:trends}

\section{Toxins alter expression of many metabolic genes}
As discussed in a previous chapter, many genes encoding enzymes
for lipid and fatty acid pathways were differentially expressed
after toxin treatment. We therefore hypothesized that metabolism
of lipids and fats will alter the host responses to TcdA.
To test this, we placed mice on a high fat diet and performed
cecal injection experiments as described previously.
When the two factors were accounted for (western diet versus regular diet and
toxin-treated versus sham), there was no significant effects
caused by a high fat diet 16h after toxin injection (data not shown).
Nevertheless, many metabolic genes were altered. 
Since the greatest expression differences
occurred at 16 hours, it is possible that these changes
reflect upstream changes in central regulators known to
control the expression of many of these genes (e.g., nuclear receptors).

\section{Integrating and visualizing transcriptional changes within a metabolic network}
Since well-curated, genome-scale networks have been
constructed for human and mouse, it is possible to overlay
the transcriptional data in this dissertation onto these networks
to visualize modules or entire pathways that are altered.
In addition, the transcription may be used to turn enzymes on or
off in these models in order to determine shifts in metabolism
during pathogenesis or repair from TcdA and TcdB. 

Integration of transcription data with metabolic 
is therefore are a potential future direction that could
identify novel host responses to toxin. Although this dissertation
does not use metabolic models, I have had significant experience
with them while working with colleagues. In this chapter,
I present a review of methods for identifying drug targets
from metabolic reconstructions of microbes. However, the last
section is dedicated to how the host metabolism can controlled
or modified for new treatment strategies.

%Term definitions
\newglossaryentry{biomass}{
  name={biomass}, 
  text={biomass reaction},
  symbol={biomass},
  description={an objective 
  reaction consisting of important metabolites that the cell 
  needs in order to grow (e.g.\ amino acids, lipids and 
  carbohydrates). Correlated reaction sets: groups of reactions 
  whose \glspl{flux} always change in relation to one another}}
\newglossaryentry{corset}{
  name={correlated reaction set}, 
  text={correlated reaction set},
  symbol={CoSet},
  description={groups of reactions 
  whose \glspl{flux} always change in relation to one another}}
\newglossaryentry{exchangereaction}{
  name={exchange reaction}, 
  description={a reaction that transports metabolites 
  into or out of the system}}
\newglossaryentry{fluxspace}{
  name={feasible flux space}, 
  description={the possible combination of allowable 
  \glspl{flux} given a defined set of constraints which bound 
  reaction \glspl{flux}, for example reversible and irreversible 
  reactions}}
\newacronym
  [description={a linear programming problem formulated to maximize
  the \glspl{flux} through an objective reaction, which the analyst defines, under
  steady-state flux constraints that are derived from the
  stoichiometric matrix of the metabolic network},
  name={flux balance analysis (FBA)}]
  {fba}{FBA}{flux balance analysis}
\newglossaryentry{fluxdistribution}{%
  name={flux distribution}, 
  description={the set of all reaction 
  \glspl{flux} within a metabolic network}}
\newacronym
  [description={a \gls{linearprogramming}-based method which 
  determines the minimum and maximum reaction \glspl{flux} that 
  allow for optimal or near-optimal \gls{flux} through the objective 
  reaction},
  name={flux variability analysis (FVA)},
  symbol={flux variability}]
  {fva}{FVA}{flux variance analysis}
\newacronym
  [description={the combination of proteins or protein 
  components sufficient to carry out an enzymatic reaction and 
  the combination of genes sufficient to express each of 
  the protein components},
  name={gene-protein-reaction relationship (GPR)}]
  {gpr}{GPR}{gene-protein-reaction relationship}
\newglossaryentry{insilico}{%
  name={in silico}, 
  description={in contrast to in vivo or in vitro, 
  the term indicates a computational process in a 
  simulated environment}}
\newglossaryentry{linearprogramming}{%
  name={linear programming}, 
  description={also termed linear optimization, an area 
  of mathematics developed for maximizing a linear 
  combination of variables (e.g.\ $A v_1 + B v_2 + C v_3$), 
  such that the variables are constrained by many linear 
  equalities and inequalities (e.g.\ the constraint $v_1 - v_2 = 0$ 
  implies that \gls{flux} $v_1$ is constrained to be twice \gls{flux} $v_2$)}}
\newglossaryentry{massbalance}{%
  name={mass balance}, 
  description={a requirement that the mass entering 
  the system or any pathway within the system equals 
  the mass exiting the system or pathway; a crucial 
  characteristic of metabolic \glssymbol{reconstruction}s from which 
  functional mathematical models can be derived}}
\newglossaryentry{reconstruction}{%
  name={metabolic network reconstruction},
  symbol={reconstruction},
  description={a manually-curated computational network 
  of the metabolism of an organism with all the \glspl{gpr} 
  assembled from a functionally annotated genome, biochemical 
  data, and literature, that are compiled into a 
  \gls{smatrix}, which serves as the framework 
  for further computational analysis}}
\newglossaryentry{objectiveflux}{%
  name={objective flux},
  description={the \gls{flux} through the \gls{objectivereaction}}}
\newglossaryentry{objectivereaction}{%
  name={objective reaction},
  symbol={objective},
  description={a reaction which sets a demand for 
  particular metabolites in the network. The typical 
  goal in formulating this reaction is to simulate 
  a biological objective, whether it be growth, 
  energy, virulence, or a combination of other factors}}
\newglossaryentry{flux}{%
  name={reaction flux},
  text={flux},
  plural={fluxes},
  description={moles consumed in a reaction per unit time}}
\newglossaryentry{steadystate}{%
  name={steady state},
  description={with respect to \gls{flux}, a key assumption 
  in \gls{fba} that the reaction \glspl{flux} within the system, and 
  therefore the amounts of each metabolite, do not change 
  over time, an assumption often justified by the very short 
  time-scale of metabolic reactions compared to the time 
  necessary for changes in cell phenotype}}
\newglossaryentry{smatrix}{%
  name={stoichiometric matrix (S matrix)},
  text={stoichiometric matrix},
  description={a mathematical formalization of a 
  metabolic \glssymbol{reconstruction}; a matrix in which each 
  element contains the stoichiometric coefficient 
  for a metabolite (row) participating in the 
  corresponding reaction (column)}}

\section{Synopsis}
For many infectious diseases, novel treatment options are needed 
in order to address problems with cost, toxicity and resistance to 
current drugs. Systems biology tools can be used to gain valuable 
insight into pathogenic processes and aid in expediting drug discovery. 
In the past decade, constraint-based modeling of genome scale 
metabolic networks has become widely used. Focusing on pathogen 
metabolic networks, we review \gls{insilico} strategies used to identify 
effective drug targets and highlight recent successes as well as 
limitations associated with such computational analyses. We further 
discuss how accounting for the host environment and even targeting the 
host may offer new therapeutic options. These systems-level approaches 
are beginning to provide novel avenues for drug targeting against 
infectious agents.

% Make a glossary section
\printglossary
Clicking on a glossary term within this chapter returns you to its definition in this section.


\section{Systems biology and pathogen metabolism}
Systems biology methods have been applied extensively to the 
study of infectious diseases across multiple scales of 
biological organization to generate predictions ranging from pathogen 
gene lethality in particular microenvironments 
\cite{Jamshidi:2007ei,Fang:2010gc} to dynamics 
involved in the host immune response to infection \cite{SegoviaJuarez:2004bn}. Utilizing 
the predictive power of computational modeling and systems analysis, 
wideranging questions related to pathogen virulence, disease 
progression and host response can be explored to generate hypotheses 
for more thorough experimental investigation.

The increasing availability of high-quality genome-scale 
metabolic \glssymbol{reconstruction}s \cite{Oberhardt:2009jw} presents an opportunity for the 
rational and systematic identification of metabolic drug 
targets in a pathogen of interest. Built bottom-up from functional 
genome annotations (and a variety of other data sources) and 
analyzed with computational methods such as \gls{fba}, these biochemical networks can account 
for hundreds to thousands of metabolites participating in enzymatic 
reactions across a range of metabolic subsystems (e.g.\ carbohydrate, 
amino acid, lipid, nucleotide and energy metabolism) and cellular 
localizations (e.g.\ extracellular space, cytosol and compartments 
specific to particular organisms) (\autoref{trends:fig1}A) \cite{Thiele:2010fr}. Since the 
initial genome-scale \glssymbol{reconstruction}s of \textit{Escherichia coli} \cite{Edwards:2000cn,Reed:2003bj} 
and Haemophilus influenzae \cite{Edwards:1999ix,Schilling:2000ig}, the metabolic networks of over 
50 organisms (bacteria, archaea and eukaryotes) have been 
reconstructed (reviewed in \cite{Oberhardt:2009jw}). Elements of this network 
\glssymbol{reconstruction} process have been automated, allowing the preliminary 
analysis of hundreds of draft network \glssymbol{reconstruction}s \cite{Henry:2010dg}. Among 
these, metabolic networks have been reconstructed for several pathogenic 
organisms (\autoref{trends:table1}). Indeed, the study of pathogen metabolism-for the 
elucidation of highpriority drug targets and metabolic factors 
contributing to pathogenicity—is an exciting application for 
metabolic network modeling and systems biology.

\begin{figure}[h!]
  \centering
  \includegraphics[width=0.95\textwidth]{trends/Figure1}
  \caption[The iterative process of model building and refinement]{
       \textbf{The iterative process of model building and refinement.}
       \textbf{(A)} A 
       functionally annotated genome together with data from the 
       biochemical literature are used to assemble a network 
       \glssymbol{reconstruction}. \gls{fba} allows for 
       modeling and simulation of the reconstructed network. 
       Advanced network analyses (such as gene essentiality or 
       \glssymbol{fva}) allow for identifying potential 
       antimicrobial drug targets. These targets can then be 
       associated with drugs using bioinformatics approaches 
       and obtaining target drug information from a variety 
       of publicly available databases (e.g.\ STITCH or DrugBank). 
       Predictions involving targets and drugs can be experimentally 
       validated. Any discrepancies between computational 
       predictions and experimental validation can be informative 
       to improving upon and refining the original \glssymbol{reconstruction} 
       and modeling platform. \textbf{(B)} \glspl{gpr}, 
       central to the assembly of a metabolic 
       \glssymbol{reconstruction}, define the genes and gene products needed 
       for each enzymatic reaction. Isozymes can be represented 
       with 'OR' statements, whereas enzyme subunits required 
       to function together to catalyze a particular reaction 
       can be represented with 'AND' statements.
  }
  \label{trends:fig1}
\end{figure}

\begin{sidewaystable}
\begin{tabular}{ l  c  c  c  c  c  c  c  c | c  c | c l l }%14 columns
\multicolumn{1}{l}{} & \multicolumn{8}{c}{\textbf{Drug target-related analysis}} & 
                     \multicolumn{2}{c}{\textbf{\specialcell{Valid-\\ation}}} & \multicolumn{1}{c}{}
                      &  & \\ \cmidrule(r){2-9} \cmidrule(r){10-11}
     \multicolumn{1}{l}{\textbf{Pathogen}} & 
     \multicolumn{1}{c}{\rotatebox[origin=l]{90}{Gene/reaction essentiality}} & 
     \multicolumn{1}{c}{\rotatebox[origin=l]{90}{Minimal media prediction}} & 
     \multicolumn{1}{c}{\rotatebox[origin=l]{90}{Conditional essentiality}} & 
     \multicolumn{1}{c}{\rotatebox[origin=l]{90}{Synthetic lethality}} & 
     \multicolumn{1}{c}{\rotatebox[origin=l]{90}{Flux variability analysis}} & 
     \multicolumn{1}{c}{\rotatebox[origin=l]{90}{Enzyme robustness}} & 
     \multicolumn{1}{c}{\rotatebox[origin=l]{90}{Metabolite essentiality}} & 
     \multicolumn{1}{c}{\rotatebox[origin=l]{90}{Correlated reaction sets}} & 
     \multicolumn{1}{c}{\rotatebox[origin=l]{90}{Literature derived}} & 
     \multicolumn{1}{c}{\rotatebox[origin=l]{90}{Novel experimental validation~}} &
     \multicolumn{1}{c}{\rotatebox[origin=l]{90}{Novel compounds identified}} & 
     \multicolumn{1}{l}{\textbf{Disease}} & \textbf{Refs} \\ %\cline{2-12}
\textit{Acinetobacter baumannii} &
     \cellcolor[gray]{0.8} & & & & & & \cellcolor[gray]{0.8} &  
     & \cellcolor[gray]{0.8} & & & Opportunistic; cepacia syndrome & \cite{Kim:2010cu} \\ \cline{2-12}
\textit{Burkholderia cenocepacia} &
     \cellcolor[gray]{0.8} & & \cellcolor[gray]{0.8} & & & & &  
     & \cellcolor[gray]{0.8} & & & Opportunistic; nosocomial infection & \cite{Fang:2011io} \\ \cline{2-12}
\textit{Francisella tularensis} &
     \cellcolor[gray]{0.8} & & \cellcolor[gray]{0.8} & & 
     \cellcolor[gray]{0.8} & \cellcolor[gray]{0.8} 
     & &  & \cellcolor[gray]{0.8} & \cellcolor[gray]{0.8} 
     & & Tularemia & \cite{Raghunathan:2010gl} \\ \cline{2-12}
\textit{Haemophilus influenzae} & 
     \cellcolor[gray]{0.8} & \cellcolor[gray]{0.8} & & \cellcolor[gray]{0.8} & 
     & & & & \cellcolor[gray]{0.8} & & 
     & Otitis media and respiratory infections & \cite{Edwards:1999ix,Schilling:2000ig} \\ \cline{2-12}
\textit{Helicobacter pylori} & 
     \cellcolor[gray]{0.8} & \cellcolor[gray]{0.8} & 
     \cellcolor[gray]{0.8} & \cellcolor[gray]{0.8} & 
     & & & & \cellcolor[gray]{0.8} & & 
     & Gastritis; peptic ulceration; gastric cancer & \cite{Thiele:2005el,Schilling:2002ck} \\ \cline{2-12}
\textit{Klebsiella pneumoniae} & 
      & & & & & & & & & \cellcolor[gray]{0.8} 
      & & Klebsiella pneumonia; urinary tract infection & \cite{Liao:2011fp} \\ \cline{2-12}
\textit{Mycobacterium tuberculosis} & 
     \cellcolor[gray]{0.8} & & \cellcolor[gray]{0.8} 
     & \cellcolor[gray]{0.8} & 
     \cellcolor[gray]{0.8} & \cellcolor[gray]{0.8} 
     & & \cellcolor[gray]{0.8} & \cellcolor[gray]{0.8} & \cellcolor[gray]{0.8} 
     & & Meningitis; meningococcal septicemia & \cite{Jamshidi:2007ei,Fang:2010gc,Beste:2007bi} \\ \cline{2-12}
\textit{Neisseria meningitidis} & 
      & \cellcolor[gray]{0.8} & & & & 
      & & & & \cellcolor[gray]{0.8} & 
      & Opportunistic; cepacia syndrome & \cite{Baart:2007cz} \\ \cline{2-12}
\textit{Porphyromonas gingivalis} & 
     \cellcolor[gray]{0.8} & \cellcolor[gray]{0.8} & & & 
     & & & & \cellcolor[gray]{0.8}  & & 
     & Periodontal disease & \cite{Mazumdar:2009gj} \\ \cline{2-12}
\textit{Pseudomonas aeruginosa} &
     \cellcolor[gray]{0.8} & & & & & & &  
     & \cellcolor[gray]{0.8} & & & Opportunistic; nosocomial infection & \cite{Oberhardt:2008fr} \\ \cline{2-12}
\textit{Salmonella} Typhimurium &
     \cellcolor[gray]{0.8} & & & \cellcolor[gray]{0.8} & 
     \cellcolor[gray]{0.8} & & &  
     & \cellcolor[gray]{0.8} & \cellcolor[gray]{0.8} 
     & & Gastroenteritis; diarrhea & \cite{AbuOun:2009co,Thiele:2011fy,Raghunathan:2009kf} \\ \cline{2-12}
\textit{Staphylococcus aureus} & 
     \cellcolor[gray]{0.8} & \cellcolor[gray]{0.8} & & \cellcolor[gray]{0.8} & 
     & & & & \cellcolor[gray]{0.8} & \cellcolor[gray]{0.8} & 
     & Opportunistic; nosocomial infection & \cite{Lee:2009fc,Heinemann:2005fc,Becker:2005hf} \\ \cline{2-12}
\textit{Vibrio vulnificus} & 
      & & & & & & \cellcolor[gray]{0.8} & & \cellcolor[gray]{0.8} & \cellcolor[gray]{0.8} 
      & \cellcolor[gray]{0.8} & Bubonic, pneumonic, and septicemic plague & \cite{Kim:2011jm} \\ \cline{2-12}
\textit{Yersinia pestis} & 
     \cellcolor[gray]{0.8} & \cellcolor[gray]{0.8} & & \cellcolor[gray]{0.8} & 
     & & & & \cellcolor[gray]{0.8} & & 
     & Opportunistic; nosocomial infection & \cite{Navid:2009en} \\ \cline{2-12}
\textit{Cryptosporidium hominis} & 
     \cellcolor[gray]{0.8} & & & & 
     & & & & \cellcolor[gray]{0.8} & & 
     & Cryptosporidiosis & \cite{Vanee:2010du} \\ \cline{2-12}
\textit{Leishmania major} & 
     \cellcolor[gray]{0.8} & \cellcolor[gray]{0.8} & & \cellcolor[gray]{0.8} & 
     & \cellcolor[gray]{0.8} & & & \cellcolor[gray]{0.8} & & 
     & Leishmaniasis & \cite{Chavali:2008gh} \\ \cline{2-12}
\textit{Plasmodium falciparum} & 
     \cellcolor[gray]{0.8} & & & \cellcolor[gray]{0.8} & 
     & & & & \cellcolor[gray]{0.8} & \cellcolor[gray]{0.8} & \cellcolor[gray]{0.8}
     & Malaria & \cite{Huthmacher:2010hi,Plata:2010dw} \\ \cline{2-12}
\textit{Trypanosoma cruzi} & 
     \cellcolor[gray]{0.8} & & & & \cellcolor[gray]{0.8}
     & & & & \cellcolor[gray]{0.8} & &
     & Chagas disease & \cite{Roberts:2009dl} \\ %\cline{2-12}
\end{tabular}
\caption[Drug targeting-related analysis of pathogen metabolic networks]{
          Drug targeting-related analysis of pathogen metabolic networks}
\label{trends:table1}
\end{sidewaystable}

In this review, we explore several techniques and approaches 
used to predict antimicrobial drug targets from metabolic network 
modeling using \gls{fba}. Where possible we present examples that have 
led to novel data, drug targets, or drugs. Metabolic network 
modeling is still in its infancy, but has allowed for predictions 
that align with previous data and has provided many hypotheses 
that continue to be developed. We first discuss the fundamental 
aspects of network analysis and \gls{fba} in particular. Subsequently, 
we delve into how computational metabolic \glssymbol{reconstruction}s can be 
used to prioritize drug target predictions. Furthermore, we 
review recent developments on model-guided pipelines for drug 
target discovery against pathogens. Finally, we extend the discussion 
to include host cell metabolism and propose directions for future 
modeling efforts in infectious disease.

\section[Reconstructing metabolic networks]{Reconstructing the metabolic network and defining an objective}

A \gls{reconstruction} is assembled piece-by-piece 
by compiling data on known enzymes, genes encoding these enzymes, 
and the stoichiometry of the reactions catalyzed by these enzymes 
(see \cite{Durot:2009kb} for a list of databases containing such data). \glspl{gpr}, 
in the form of Boolean logic statements, define 
which genes are necessary for each enzyme and which enzymes are 
necessary for each reaction \cite{Thiele:2010fr} (\autoref{trends:fig1}B). The information for 
all the reactions in a network \glssymbol{reconstruction} with m metabolites 
and n reactions can be stored in an m by n table or matrix, the 
\gls{smatrix}. Each element or cell in this matrix 
corresponds to the stoichiometric coefficient of one particular 
metabolite in one particular reaction \cite{Thiele:2010fr,Lee:2006je}. The \gls{smatrix} enables 
strict accounting for the underlying biochemistry and allows a 
quantitative description of complex interactions between metabolites 
that are responsible for driving a cellular phenotype. This matrix 
formalism facilitates interrogation of the structural and functional 
properties of the network.

The application of \gls{fba} to a network \glssymbol{reconstruction} results in the 
identification of combinations of reaction \glspl{flux} that correspond 
to a maximum \gls{flux} through a targeted reaction (an \glssymbol{objectivereaction}) while 
requiring that constraints are satisfied, for example that the 
mass entering the network is equal to the mass exiting the network (\gls{massbalance}). 
In more mathematical terms, \gls{fba} involves the use of a 
\gls{linearprogramming} formulation wherein an objective is optimized subject 
to a set of governing constraints (\hyperref[trends:box:fba]{Box \ref{trends:box:fba}}) 
\cite{Oberhardt:2009jw,Lee:2006je,Gianchandani:2010gs}. In addition 
to requiring \gls{massbalance} for every reaction, thermodynamic, 
topological, environmental, and regulatory data may provide 
additional constraints that dictate the \gls{fluxspace} \cite{Price:2004hx}. 
An \glssymbol{objectivereaction} often used with \gls{fba} is \glssymbol{biomass} production, which is 
represented \gls{insilico} as a drain capturing crucial metabolites 
necessary for growth of the organism \cite{Feist:2010hq,Varma:1993hd}. With the ultimate 
goal of identifying antimicrobial drug targets (and associated 
drugs) to slow or stop the growth of a pathogen, a \glssymbol{biomass} objective
is often very appropriate for computational modeling efforts. 
Network \glssymbol{reconstruction}s of most pathogenic organisms have incorporated 
\glspl{biomass} as their \glssymbol{objectivereaction}s (\autoref{trends:table1}).

%%%%%% BOX 1 %%%%%%%%%
\begin{pabox}[label=trends:box:fba,float*=t,width=\textwidth]{A brief primer on \glsentrytext{fba}}
\small
\begin{multicols}{2}
Nutrient availability, restrictions on surrounding 
environmental pH, and temperature are examples of basic 
constraints that, when imposed upon cells, can affect 
the resulting phenotypes \cite{Lee:2006je}. Such constraints can 
be mathematically described, and they serve to narrow 
the operating range of the cell and yield a set of feasible 
reaction \glspl{flux} for a metabolic network. In other words, 
the constraints (which can be physicochemical, 
topological, environmental and regulatory) restrict 
the number of possible phenotypes, which then allows 
the function of the biochemical network to be 
characterized \cite{Lee:2006je,Price:2004hx}.

~~~~~~\gls{fba} is one constraint-based 
method that has been extensively applied in the study 
of prokaryotic and eukaryotic metabolic networks \cite{Oberhardt:2008fr,Duarte:2007iu}. 
First, reactions of a metabolic network are assembled 
into a \gls{smatrix}, whose elements 
correspond to the stoichiometric coefficients describing 
the conversions from reactants to products (see figure below) \cite{Thiele:2010fr,Lee:2006je}. 
This simple matrix formalism permits quantitative 
description of the complex interactions between metabolites. 
Following network \glssymbol{reconstruction}, the concentrations of 
metabolites and \glspl{flux} through reactions can be represented 
as follows:

\begin{equation} \label{trends:eq:massbalance}
\frac{dC}{dt} = S v
\end{equation}

Here, $C$ is a vector of concentrations of metabolites, $t$ is 
time, $S$ is the \gls{smatrix} consisting of $m$ rows 
of metabolites and $n$ columns of reactions, and $v$ is a 
vector of \glspl{flux} through the corresponding reactions. 
Invoking the \gls{steadystate} assumption so that the rate 
of production of every metabolite equals its rate of 
consumption yields the following:

\begin{equation} \label{trends:eq:steadystate}
Sv = 0
\end{equation}

Limits can be applied to individual \glspl{flux} as follows:

\begin{equation*}
v_{min} \leq v \leq v_{max}
\end{equation*}

Particular reactions can have set upper limits ($v_{max}$) that 
may align closely with experimental enzyme capacity measurements, 
whereas other irreversible reactions will have $v_{min}$ set to 0.

~~~~~~The principle physicochemical constraints in \autoref{trends:eq:steadystate} 
represents a set of $m$ linear equations. 
Because there are typically more unknown variables ($m$ reaction \glspl{flux}) than equations ($n$ steady state
equations for each metabolite), the system of equations is ``indeterminate'' \cite{Lee:2006je}.
In other words, there may be many sets of \glspl{flux} (many \glspl{fluxdistribution}) that 
can satisfy the steady state constraints as well as the $v_{min}$ and $v_{max}$ of each reaction.
The goal then is to use \gls{linearprogramming} to identify which of these feasible \glspl{fluxdistribution}
allow for the greatest \gls{flux} through the \gls{objectivereaction}.

~~~~~~Traditionally, maximization of \glssymbol{biomass} has been 
chosen as the \glssymbol{objectivereaction} of choice in \gls{fba} \cite{Lee:2006je}. A set 
of metabolites (e.g.\ amino acids, lipids, nucleotides and 
carbohydrates) that are necessary for the cell or organism 
to grow are typically included in the \gls{biomass}. 
Therefore, the optimization problem can be summarized as simply:
\begin{alignat*}{2}
\text{maximize  }   & v_{biomass}  \\
\text{subject to  } & Sv=0 \\
                   & v_{min} \leq v_i \leq v_{max} \text{ for } i=1...n
\end{alignat*}
Even with these constraints, \gls{linearprogramming} does not guarantee that the set of \glspl{flux} that
allow for the maximal \gls{objectiveflux} are unique. What is 
guaranteed is that the \gls{objectiveflux} ($v_{biomass}$) found is the
highest possible \gls{flux} through the \gls{objectivereaction} that the network can allow, under any state.
There are often are many possible \glspl{fluxdistribution} that achieve
the maximal \gls{objectiveflux}. See \hyperref[trends:box:fva]{Box \ref{trends:box:fva}} for
a method that helps investigate the many possible solutions.


%\begin{figure}[h!]
  \centering
  \includegraphics[width=\columnwidth]{trends/FigureBox}
  %\caption[A generic metabolic model]{
  %     \textbf{A generic metabolic model}
  %}
  \label{trends:figbox}
%\end{figure}

\end{multicols}

\end{pabox}

The inability of the metabolic network model to synthesize even one 
metabolite of the \gls{biomass} will result in a predicted value 
of zero for the \gls{objectiveflux} (\glssymbol{biomass}), and analogously no growth. 
Therefore, growth predictions are sensitive to metabolites that are 
placed in the \gls{biomass} reaction. \gls{fba} can be used to investigate the 
ability of the model to produce each metabolite within \glssymbol{biomass}. In 
the \textit{Porphyromonas gingivalis} metabolic network, the ability of the 
model to produce each of the 52 metabolites within the \gls{biomass} 
was evaluated after systematic reaction deletions \cite{Mazumdar:2009gj}. 
Crucial groups of reactions were identified that were responsible for 
lipopolysaccharide (LPS) production, coenzyme A production, glycolysis, 
or purine and pyrimidine biosynthesis \cite{Mazumdar:2009gj}. Corresponding enzymes that 
are essential for growth, as well as for the production of important 
bacterial components such as LPS, could serve as potential drug 
targets. In addition, in a study of \textit{Leishmania major} metabolism, 
the contribution of minimal media components to the synthesis of 
individual \gls{biomass} constituents was analyzed \cite{Chavali:2008gh}. The study found 
that the absence of cysteine and oxygen in the minimal media had a 
drastic impact on the overall metabolic network, limiting the 
generation of 30 of the 40 \glssymbol{biomass} constituents \cite{Chavali:2008gh}. Network analyses 
as delineated in these examples of \textit{P. gingivalis} and \textit{L. major} may 
permit the formulation of a hypothesis for the role of specific 
metabolites and their influence on growth. Therefore, defining an 
appropriate \gls{biomass} (\hyperref[trends:box:objective]{Box \ref{trends:box:objective}} and
\hyperref[trends:box:caveats]{Box \ref{trends:box:caveats}}) is crucial for useful 
predictions and identification of vulnerable parts of the metabolism 
of a pathogen.

%%%%%%%%% BOX 2 %%%%%%%%%%%%%%
\begin{pabox}[label=trends:box:objective]{Cellular objectives of pathogenic organisms}
A \gls{biomass} is not applicable under 
all conditions, and very often growth alone may not be 
a realistic objective. Other \glssymbol{objectivereaction} such as maximizing 
or minimizing ATP or maximizing the production of 
particular cellular by-products (e.g.\ lactate or pyruvate) 
can also be used.

~~~~~~The metabolism of an organism may be adapted for increased 
virulence or pathogenicity. Pathogens and host cells may 
also temporarily opt for alternative objectives while under 
selective pressures (e.g.\ changes in nutrients or 
environmental influences of secreted toxins). In addition, 
different morphological stages of a pathogen (e.g.\ the sporozoite 
stage of Plasmodium falciparum in mosquitoes vs the merozoite 
stage in humans) may be characterized by varying metabolic 
requirements. Consequently, the \glssymbol{objectivereaction} function must be 
appropriately defined to find relevant enzyme targets crucial 
in particular stages of infection and/or environmental conditions.

~~~~~~To explore the effects of targeted perturbations 
(pharmacological or genetic), different \glssymbol{objectivereaction}s 
can be explored. It may be that the evolutionary 
pressures that dictate wild-type cells are different 
for knockout mutants. Additionally, mutants may 
not have the ability for immediate regulation of 
\glspl{flux} that allow for optimal growth. Based on these 
ideas, an approach termed minimization of metabolic 
adjustment (MOMA) was developed. The requirement for 
optimal growth is relaxed for gene deletions in MOMA. 
Instead, MOMA assumes that the overall \gls{fluxdistribution} 
of a gene-knockout mutant will probably not change 
significantly from that of the corresponding wild type [71]. 
In terms of \gls{flux} values, the gene-knockout mutant will 
remain as close as possible (in Euclidean distance) to 
the wild-type optimal \gls{fluxdistribution}. MOMA aided in correcting 
gene essentiality predictions associated with knockouts 
of fructose-1,6-bisphosphatate aldolase, triosephosphate 
isomerase and phosphofructokinase in the E. coli 
central metabolic model [71]. These genes were predicted 
to be nonessential when \glssymbol{biomass} was used as the \glssymbol{objectivereaction} 
for E. coli growth on glucose, which was inconsistent with 
supporting literature evidence. MOMA yielded a suboptimal 
\gls{fluxdistribution} for a knockout mutant that would not 
necessarily equal the optimum as dictated by traditional 
\gls{fba} [71].

~~~~~~Approaches such as MOMA that consider alternate hypotheses 
for the objective of metabolic networks provide a basis 
for understanding a potential biological goal for pathogenic 
organisms, especially considering the complexity of the 
environment surrounding the pathogen of interest.
\end{pabox}


\begin{pabox}[label=trends:box:caveats]{Knowledge gaps and caveats to metabolic network analysis}
Multiple steps in the model-building process and subsequent 
analyses are prone to errors that may greatly affect \gls{flux} 
and growth predictions and, consequently, predicted drug 
targets. \gls{fba} provides a quality-assurance check that ensures 
\gls{massbalance}. Growth rates and gene essentiality predictions 
are validated against experimental data to ensure the 
model truly reflects biological processes. Gene essentiality 
predictions (commonly between 55\% and 90\%) provide 
confidence that downstream analyses are based on a 
high-quality model \cite{Feist:2007dq,Jamshidi:2007ei}. To ensure the usefulness of 
any computational pipeline, drug target predictions 
should be compared to known targets from the literature. 
Below we discuss more of the particular difficulties and 
limitations encountered when using metabolic models to 
identify drug targets.

\subsection*{Genome annotation}
In any metabolic \glssymbol{reconstruction}, there may be 
hundreds of putative metabolic enzymes with no 
experimentally identified function. These reactions 
and associated \glspl{gpr} may be assembled 
strictly based on existing functional annotations 
of the genome or based on evidence from related 
organisms. Even some very well characterized enzymes 
may have other unexpected activities. In such cases, 
misannotated enzymes may yield incorrect model 
predictions leading to errors in drug targeting. For 
instance, an enzyme may be incorrectly predicted to be 
essential if the activity of another enzyme, which is 
not included in the network, can account for the same function. 
Therefore, this represents one important knowledge gap in 
the assembly of metabolic networks, and the inclusion of 
more refined enzyme annotations will directly improve 
drug target predictions \cite{Hsiao:2010bu,Szappanos:2011gu}.

\subsection*{Nutrient availability}
An important challenge in reconstructing and modeling 
metabolic networks is determining the composition of 
\gls{insilico} media. Nutrients available in host environments 
are poorly characterized. In addition, for any nutrients 
that are identified, quantitative data on uptake rates 
are unavailable. Knowledge of the transporters of an 
organism elucidates which metabolites are transported 
into the cell. However, information on transporters 
is particularly limited and, in general, transport 
reactions lack any experimental evidence or gene 
associations supporting their presence. Instead, 
transport reactions are added for proper functioning 
of the computational model. Because model predictions are 
dependent on the media environment, nutrients and 
transporters must be carefully defined.

\subsection*{Objective function}
As stated previously, in \gls{fba}-based modeling of microbial organisms, 
a \gls{biomass} has often been used. The purpose of 
the reaction is to ensure a drain of metabolites that 
are deemed essential to support the growth of the 
organism. Starting with the estimated weight fraction 
of important macromolecular components of the cell 
(e.g.\ protein, lipid, RNA, DNA and carbohydrate), the 
relative abundance of metabolites comprised in each 
group (e.g.\ amino acids, phospholipids, nucleotides) 
can be computed \cite{Feist:2010hq}. Among several available, a few 
experimental methods to measure \glssymbol{biomass} components 
include chloroform–methanol extraction (lipids), 
colorimetric protein assays, and gas chromatography–mass 
spectrometry (protein content) \cite{Boyle:2009jr}. \Glspl{flux} measured 
directly by metabolic flux analysis, which tracks 
the movement of 13C from an initial 13C-labeled 
substrate, may also help to define the \glssymbol{objectivereaction} 
function \cite{Chen:2011dj,Blank:2005bt}.

~~~~~~An \glssymbol{objectivereaction} function is most likely to cause 
errors if metabolites are entirely missing or 
incorrectly included; the relative amounts of each 
metabolite in the \gls{objectivereaction} (i.e.\ the 
stoichiometric coefficients) do not greatly affect 
\gls{fba} results \cite{Varma:1993hd,Varma:1994tb}. Therefore, logically 
deducing a \gls{biomass} is often adequate 
to estimate the growth of an organism. However, 
additional experimental data can reveal interesting 
peculiarities that may be used to design a specific 
\gls{biomass} for a particular organism. 
Data on growth- and non-growth-associated ATP 
maintenance can be included in the \gls{biomass} \cite{Forster:2003jb}. Under in vivo conditions when a 
pathogen is interacting with its host, an aspect 
that is often unclear is which \glssymbol{biomass} component(s) 
to include in the reaction. The composition of the 
\gls{biomass} is likely to vary under different 
physiological conditions, and the choice of metabolites 
can directly influence model predictions regarding 
drug targets. For example, failure to include a 
particular cell-wall component will not necessarily 
direct \gls{flux} through reactions that may be crucial in 
vivo, and therefore the associated enzymes will not 
be targeted.
\end{pabox}


\section[Drug targets in metabolic networks]{The quest for drug targets in metabolic networks of pathogens}

\subsubsection{Gene essentiality analysis}
The most common method to identify potential drug targets has been 
through the prediction of essential genes (\autoref{trends:table1}). The enzymes 
encoded by essential genes are typically hypothesized as drug targets. 
Gene knockouts might lead to a redistribution of \gls{flux} through the 
network if the perturbed gene or gene product affects the removal 
of a particular \gls{flux}-carrying reaction \cite{Oberhardt:2009jw}. A \gls{gpr} aids in mapping 
the effects of a genetic (or pharmacological) perturbation on the 
associated reactions, and thus the network. Gene-level perturbations 
that result in reduced or zero \gls{flux} through a \gls{biomass} 
correspond to growth-reducing or lethal gene knockouts, 
respectively (\autoref{trends:fig2}). For example, in a \glssymbol{reconstruction} of 
\textit{Mycobacterium tuberculosis}, five previously known drug targets 
were encoded by genes predicted to be essential from computational 
analysis \cite{Beste:2007bi}. For metabolic \glssymbol{reconstruction}s of pathogens, 
enzymes that are predicted to be essential will offer new 
experimental hypotheses and avenues for drug discovery. In the 
following subsections we discuss several other approaches for 
drug targeting using metabolic network analysis. These other 
approaches may provide a separate list of potential 
targets; however, they can also be used in tandem with 
gene essentiality analysis for step-by-step prioritization 
of drug targets.

\begin{figure}[h!]
  \centering
  \includegraphics[width=1.05\textwidth]{trends/Figure2}
  \caption[Drug targeting in metabolic networks]{
       \textbf{Drug targeting in metabolic networks.}
       Various strategies are illustrated for identifying drug 
       targets by performing \gls{fba} on metabolic \glssymbol{reconstruction}s. 
       The sample network shows an input media that represents 
       the environment and \glspl{exchangereaction}, intracellular 
       reactions, and an \gls{objectivereaction} that drains metabolites 
       out of the system. An essential reaction and metabolite 
       that, when removed, block any \gls{flux} through the \glssymbol{objectivereaction} 
       are highlighted in red. In the conditionally essential 
       panel, the absence of the metabolite highlighted in blue 
       causes the highlighted reaction to become essential in 
       the selected media. One of the synthetic lethal pairs 
       of the network is denoted by `SL'. The dashed line in 
       the \glssymbol{fva} illustration may represent `near-optimal'
       \gls{objectiveflux}. A robust reaction maintains near-optimal 
       \gls{objectiveflux} over a larger range of reaction \glspl{flux}.
  }
  \label{trends:fig2}
\end{figure}

\subsubsection{Enzyme robustness and flux variability}
In addition to an analysis of gene essentiality, assessing enzyme 
robustness could identify vulnerable or sensitive portions of a 
metabolic network suitable for drug targeting. To determine the 
robustness of a metabolic network to the inhibition of an 
enzyme-catalyzed reaction, the \gls{flux} of a reaction may be constrained 
to a fraction of its wild type \gls{flux} (simulating partial to 
complete inhibition), and the effects on an \gls{objectiveflux} 
(e.g.\ \glssymbol{biomass} production) can be evaluated \cite{Edwards:2000cz} (\autoref{trends:fig2}). 
Such an approach was used in analyzing the network \glssymbol{reconstruction} 
of \textit{Francisella tularensis}, which revealed that the growth rate 
was sensitive to changes in \ce{H+} and \ce{NH4} \gls{flux} in a simulated in vitro 
medium but not in a simulated in vivo medium that mimicked the 
environment during infection \cite{Raghunathan:2010gl}. Analyzing enzyme robustness with 
different constraints provides a detailed view of possibly nonlethal 
reactions whose change in \gls{flux} has a strong effect on the \glssymbol{objectivereaction} 
of a network under different environmental conditions, therefore 
suggesting important reactions during an infection or perturbations 
for drug targeting.

Alternatively, the \glssymbol{objectivereaction} function could be constrained to a fixed 
percentage of its wild type \gls{flux} and the allowable range of \gls{flux} 
for each reaction can be determined using an approach termed 
\gls{fva} (\hyperref[trends:box:fva]{Box \ref{trends:box:fva}}) \cite{Mahadevan:2003di}. A recognized shortcoming 
of \gls{fba} is that the typical implementation calculates only one 
of many possible solutions that optimize \gls{flux} through the 
\gls{objectivereaction}. Consequently, there may be many possible routes through 
the network that achieve the same optimal \gls{flux} for a given 
\glssymbol{objectivereaction} \cite{Price:2004hx,Mahadevan:2003di}. 
In an effort to circumvent this shortcoming, 
FVA was developed to determine the range of \glspl{flux} over which 
a particular reaction operates, while still allowing for 
optimal, or near-optimal, \gls{objectiveflux} \cite{Mahadevan:2003di}. FVA also identifies 
blocked reactions (reactions incapable of carrying any \gls{flux} in 
a given model under specified constraints) or reactions with 
different \gls{flux} ranges in various media. Enzymes that catalyze 
reactions with little to no variability in \gls{flux} for a given 
\glssymbol{objectivereaction} could be selected as potential drug targets given that 
the network may be sensitive to even modest inhibition of its 
activity. In analyzing the \glssymbol{reconstruction} of \textit{M. tuberculosis}, the 
average of the highest and lowest allowable \gls{flux} for each reaction 
was used as a point of comparison for the model under constraints 
for two growth rates \cite{Beste:2007bi}. In the study, the reaction catalyzed by 
isocitrate lyase---an enzyme important for persistence in a 
host---was predicted to have increased \gls{flux} during slow growth, 
and the activity of the enzyme was experimentally confirmed to 
be greater in slow-growing cells of the closely related 
\textit{Mycobacterium bovis} \cite{Beste:2007bi}. Hence, \glssymbol{fva} and enzyme 
robustness aid in further prioritizing drug targets identified 
from other methods such as gene essentiality.

\begin{pabox}[label=trends:box:fva,float,floatplacement=h!]{A brief primer on FVA}
\small
After the application of various constraints on the 
biochemical network, the number of allowable network 
states (or possible phenotypes) is typically large. 
Depending on the size and interconnectedness of the 
cellular network, there may be several alternative 
optimal phenotypes \cite{Price:2004hx}. \gls{fba} calculates one of many 
feasible solutions that result in the same optimal 
value of the cellular \glssymbol{objectivereaction}.

~~~~~~\gls{fva} calculates the range of 
\gls{flux} in each reaction that allows for the same optimal 
\gls{flux} through the \gls{objectivereaction}. The \gls{objectiveflux} 
of the reaction is specified as an additional 
constraint and multiple optimizations are performed 
to compute the maximum and minimum flux for every reaction in the network:
\begin{alignat*}{2}
\text{Max/Min  }   & v_{i}  \\
\text{subject to  } & Sv=0 \\
                    & v_{biomass} = Z_{obj} \\
                    & v_{min} \leq v_i \leq v_{max} \text{ for } i=1...n
\end{alignat*}
Here, $n$ refers to the number of reactions in the 
biochemical network, and Zobj is the optimal value 
of the cellular \glssymbol{objectivereaction} ($v_{biomass}$) as obtained by \gls{fba} \cite{Mahadevan:2003di}.
\end{pabox}

\subsection{Metabolite essentiality}
Many current drugs have high similarity to natural metabolites and 
compete for and/or inhibit normal enzymatic activity \cite{Dobson:2009kg}. Therefore, 
another very interesting avenue towards identifying drug targets 
in metabolic networks is via the prediction of essential metabolites. 
In contrast to the more traditional individual gene knockouts 
mentioned previously, metabolites in the \gls{smatrix} can also be 
systematically removed. Consequently, all reactions in which a 
given metabolite participates are removed, and the resultant 
effects on the \glssymbol{objectivereaction} are assessed (\autoref{trends:fig2}). A total of 
211 essential metabolites in the \textit{Acinetobacter baumanii} 
\glssymbol{reconstruction} were narrowed to 9 following removal of (i) currency 
metabolites (ATP, NADH, \ce{H2O}, etc.); (ii) metabolites present in 
the human metabolic network; and (iii) metabolites participating 
in reactions catalyzed by enzymes with human homologs \cite{Kim:2010cu}. 
Enzymes that catalyze reactions involved in the consumption or 
production of these essential metabolites may be considered 
as drug targets. Moreover, structural analogs of essential 
metabolites may be considered as test compounds for experimental 
evaluation, thus sidestepping extensive screening or 
computational predictions \cite{Kim:2011jm}.

\subsection{Combination gene and reaction perturbations}
Many anti-infectives on the market act on multiple targets. 
This multiplicity of targets was exemplified in a network 
analysis of 890 FDA-approved drugs targeting 394 human 
proteins (derived from the DrugBank database), where 
approximately 38\% of the drugs were associated with more 
than one protein target, and a few drugs were associated 
with as many as 14 targets \cite{Yildirim:2007hc}. In that regard, a drug 
discovery strategy incorporating compounds known to act 
simultaneously on multiple targets can be adopted for 
microbial pathogens. \gls{fba} allows predictions involving 
the perturbation of multiple genes or reactions in a rapid 
timeframe (minutes or hours)—a single \gls{insilico} combination 
taking only a fraction of a second \cite{Becker:2007cu}. In a \glssymbol{reconstruction} 
of \textit{M. tuberculosis} that accounted for features specific to 
an in vivo environment, all non-trivial double-deletion 
mutants (synthetic lethal pairs) were tested for \gls{insilico} 
growth using \gls{fba} \cite{Fang:2010gc}. Of two experimentally-characterized 
double gene deletions, the model accurately predicted reduced 
\gls{insilico} growth in both in vitro and in vivo environmental 
conditions \cite{Fang:2010gc}. A combination of drugs that affect synthetic 
lethal targets may act synergistically to inhibit growth of 
a pathogen, thereby paving the way for model-guided predictions 
of drug synergy. Experimentally screening for all possible 
drug combinations against a particular pathogen is costly 
and is often not feasible. Therefore, predicted combinations 
of drugs associated with synthetic lethal targets can direct 
more specific experiments and perhaps reveal entirely novel 
treatment strategies.

\subsection{Groups of targets and network topology}
Other approaches characterizing the structure of a genome-scale 
network \glssymbol{reconstruction} identify sets of reactions that act 
together and therefore may be targeted as an entire pathway. 
Sets of correlated reactions (or Co-sets) consist of groups 
of reactions whose \glspl{flux} are linked, and which represent 
functional modules within a biochemical network \cite{Papin:2004eh}. Co-sets, 
which can aid in suggesting alternative drug targets by 
identifying reactions that are functionally related to each 
other, can be divided into several categories. A perfect 
Co-set consists of a group of reactions such that, for any 
given pair in the group, a non-zero \gls{flux} in one reaction 
implies a non-zero \gls{flux} in the other, with a fixed ratio . 
Other categories include partial Co-sets (pairs of linked 
reactions, but with a variable \gls{flux} ratio) and directional 
Co-sets (a non-zero \gls{flux} in one reaction implies a nonzero 
\gls{flux} in the other, but the converse is not true) \cite{Xi:2011gc}. By 
calculating hard-coupled reaction (HCR) sets---a subgroup 
of perfect Co-sets where sets of reactions are defined by 
participating metabolites sharing a consumption to production 
connectivity of 1:1 (i.e. two reactions are linked by a 
metabolite that is connected to no other reaction)---one study 
found 25 of 147 HCR sets contained previously identified drug targets 
in \textit{M. tuberculosis} \cite{Jamshidi:2007ei}. Because an altered \gls{flux} in one reaction 
results in an altered \gls{flux} of all reactions within a HCR set, only 
one enzyme needs to be targeted. This approach aids in prioritizing 
the list of potential drug targets by identifying linked enzymatic 
reactions. Hence, analyses of the topology of a metabolic network 
can reveal key local and structural features that may be important 
drug targets when data related to environment \glspl{flux} or appropriate 
\glssymbol{objectivereaction} are not necessarily available.

\subsection{Environment and conditional essentiality}
Finally, the metabolic phenotype is dependent on the media 
environment and the exchange of metabolites into and out of 
the system (see \hyperref[trends:box:caveats]{Box \ref{trends:box:caveats}} 
for discussion of knowledge gaps 
associated with nutrient availability). By constraining uptake 
or secretion \glspl{flux}, a minimal set of metabolites that allows 
\gls{flux} through the \glssymbol{objectivereaction} 
can be computed \cite{Chavali:2008gh}. Moreover, 
enzymes or metabolites that are necessary for growth in 
various environments (e.g.\ minimal media, defined media, 
or rich media with an abundance of nutrients and carbon sources) 
can be predicted. In the \glssymbol{reconstruction} of \textit{F. tularensis}, 
genes that were essential in a simulated macrophage environment 
and five other environmental conditions were considered to be 
unconditionally essential genes (in other words, they 
represented a core set of genes that were essential regardless 
of the medium). Of the 17 virulence factors cataloged from 
previous literature, eight were unconditionally essential 
genes \cite{Raghunathan:2010gl}. By contrast, enzymes that may be necessary for 
growth under one condition, but not another, are conditionally 
essential (\autoref{trends:fig2}). With a careful consideration of an \glssymbol{objectivereaction}
and appropriate nutrient uptake, gene essentiality analysis may 
reveal new drug targets specific to particular growth conditions 
and environments in which a pathogen must survive. Such an 
analysis can also inform strategies for manipulating the 
environment of the pathogen that could be effective as a 
treatment option.

\section{From target to drug and the development of 
         model guided pipelines for drug discovery}
Computational analysis of metabolic processes in pathogens can 
yield a ranking of predicted drug targets. Bioinformatics and 
network analyses were performed to yield a high-confidence list 
of targets against \textit{M. tuberculosis} \cite{Raman:2008km}. By implementing a 
multilayered approach, targets that did not pass sequential 
cut-off values were removed (e.g.\ elimination of enzymes with 
human homologs or targets with no computationally predicted 
binding pocket) \cite{Raman:2008km}. In another proof-of-concept study it 
was noted that essential type II fatty acid biosynthesis (FAS II) 
reactions in the \textit{E. coli} MG1655 metabolic network were also 
essential in several \textit{Staphylococcus aureus} strains \cite{Shen:2010cy}. Following 
network analysis, a virtual screening strategy was employed 
whereby small molecules from a library of approximately 106 
compounds were docked to enzymes catalyzing essential reactions, 
and 41 inhibitors of FAS II enzymes were selected for experimental 
validation \cite{Shen:2010cy}. In cell viability assays, six of the inhibitors 
had growth-retarding effects against \textit{E. coli} and \textit{S. aureus} 
strains in standard LB agar plates \cite{Shen:2010cy}. Finally, following 
the identification of 163 essential metabolites, a third study 
used a layered approach to prioritize five essential metabolites 
in the metabolic network of the opportunistic pathogen 
\textit{Vibrio vulnificus} \cite{Kim:2011jm}. Currency metabolites, metabolites consumed by 
a single reaction, metabolites present in the human metabolic 
network, and metabolites associated with enzymes with human 
homologs were removed. The study screened 352 compounds found 
to be structurally similar to one of the five essential metabolites 
and identified one compound that most potently inhibited growth, 
more so than a currently used drug \cite{Kim:2011jm}. These studies provide 
various examples of model-guided pipelines to drug discovery by 
primarily using network analyses to identify and prioritize drug 
targets. Additional constraints such as enzyme druggability and 
elevated gene expression can also be used to prioritize drug 
targets, which can then guide the screening and selection of compounds.

A common approach in proposing drug targets using metabolic 
networks has been to rule out targets that overlap with host 
cell metabolism---the idea being that offtargets can be minimized 
and drug interference plus subsequent complications with the host 
can be avoided. However, there are arguments to be made in favor 
of retaining targets that overlap with human metabolism. First, 
accounting for the drug selectivity between host and pathogen 
targets at the respective binding sites may preclude off-target 
influences \cite{Hopkins:2008bb}. Second, if the goal is to discover drugs against 
infectious diseases quickly, then the best option may be to 
focus on finding new clinical indications for existing FDA-approved 
drugs (i.e. pursuing drug repurposing strategies) instead of 
developing new investigational compounds that are subject to 
regulatory hurdles \cite{Chong:2007dl}. Also, the majority of FDA-approved 
drugs target human proteins. Hence, eliminating pathogen 
targets that overlap with human proteins reduces the number 
of potential drugs that could be evaluated experimentally.

\section{A host cell perspective}
Interaction with a host cell is often crucial to the metabolism 
and survival of a pathogen. For instance, the kinetoplastid 
parasite \textit{L. major} is unable to synthesize several essential 
amino acids and therefore obtains them from the host macrophage \cite{McConville:2007df}. 
As another example, Legionella pneumophila, the bacterium 
responsible for Legionnaire's disease, ceases to replicate 
inside a host macrophage when it cannot access or process 
threonine \cite{Sauer:2005kf}. Consequently, identifying the particular 
niche of nutrients and resources in the host cell required 
by a pathogen is vital to discovering treatment options 
that specifically target host-pathogen interactions \cite{Brown:2008gm}.

A systems-level analysis of pathogen metabolism interfaced 
with cell type-specific host metabolic networks can also 
be conducted. Because \textit{P. falciparum} invades mature erythrocytes 
to establish infection in its human host, a metabolic model 
of the human erythrocyte was built in conjunction with the 
\textit{P. falciparum} \glssymbol{reconstruction} to make predictions which aligned 
closer to known conditions in the infected erythrocyte \cite{Huthmacher:2010hi}. 
This model, modifying an existing approach from Shlomi et al. \cite{Shlomi:2008ik}, 
integrated previous gene expression data where several enzymes 
were constrained to be 'on' and 'off' during specific life-cycle 
stages. The combined erythrocyte-blood stage \textit{P. falciparum} 
network correctly predicted metabolite exchanges between the 
microbe and host \cite{Huthmacher:2010hi}. A recently active area of research has 
been the development of algorithms to create cell type-specific 
metabolic networks by integrating gene and protein expression 
data with existing human metabolic \glssymbol{reconstruction}s 
\cite{Shlomi:2008ik,Becker:2008iq,Jensen:2011ie}. Inclusion 
of host-specific factors into pathogen metabolic-network 
\glssymbol{reconstruction}s or developing systems-level models of host and 
pathogen networks will continue to enable investigations into 
the complexities of the host-pathogen interplay.

Finally, inhibiting host pathways and perturbing the \gls{flux} of 
metabolites in the host cell may alter the ambient environment 
and require pathogens to adapt their metabolic needs. Therefore, 
targeting the machinery of a host cell at the host-pathogen 
interface can provide new therapeutic 
approaches \cite{Schwegmann:2008iq}. For 
example, host proteins hijacked for viral replication are 
potentially important drug targets. Recently, a high-throughput 
screening assay identified a lipophilic compound--NA255--that 
inhibits the host serine palmitoyltransferase, an enzyme 
needed for association of hepatitis C virus (HCV) with host 
lipid rafts \cite{Sakamoto:2005bd}. Moreover, to characterize transformations 
in host functions, data specific to the host cell preand 
post-infection must be obtained. The analysis of transcription 
profiles is one approach that has been successfully implemented 
to identify genes in the host cell that are differentially 
regulated due to pathogenic infection \cite{Handley:2006fc,Hossain:2006gl}. Similarly, 
profiling the proteome and lipidome of a hepatocyte over 
the time-course of infection and integrating these data with 
protein-protein interaction networks revealed multiple 
lipids and enzymes differentially regulated in HCV-infected 
cells \cite{Diamond:2010it}. A third approach for identifying factors in the 
host necessary for establishing infection involves the use 
of genetic screens in which largescale insertional mutagenesis 
is performed to develop null mutants in a human cell line \cite{Carette:2009ba}. 
Ultimately, the use of new experimental technologies along 
with metabolic modeling will be vital to discovering host 
components crucial to the survival of a pathogen.

\section{Next steps}

Advanced meta-network analyses, such as comparative modeling 
of metabolic \glssymbol{reconstruction}s across multiple strains and 
species or community-based modeling of metabolism across 
differing pathogenic organisms, have broad implications 
for understanding and investigating infectious diseases. 
Below, we highlight several future directions in this 
realm and provide a few examples of efforts already underway.

The recent completion of metabolic \glssymbol{reconstruction}s of the 
pathogen \textit{Pseudomonas aeruginosa} \cite{Oberhardt:2008fr} and the related nonpathogen 
\textit{Pseudomonas putida} \cite{Puchalka:2008ct} creates new opportunities for 
investigating species-specific differences in metabolism 
and the metabolic basis for virulence of \textit{P. aeruginosa}. 
Towards that end, a reconciliation of the two \glssymbol{reconstruction}s 
was completed such that any differences in the metabolic 
networks of \textit{P. aeruginosa} and \textit{P. putida} would be indicative 
of true biological variations as opposed to artifacts of 
the \glssymbol{reconstruction} and modeling process \cite{Oberhardt:2011jq}. In the reconciliation 
study, the model for each organism was analyzed to characterize 
the tradeoffs of producing \glssymbol{biomass} versus the production of 
individual metabolites. Compared to \textit{P. putida}, \textit{P. aeruginosa} 
was able to produce a small proportion of the shared virulence 
factor precursors with only a slight decrease in \glssymbol{biomass} 
production. In general, the metabolic flexibility analysis 
suggested that the virulence of \textit{P. aeruginosa} is complex 
and highly multifactorial, and has more flexibility than 
\textit{P. putida} in many metabolic pathways. This computational 
analysis paves the way for future modeling efforts of other 
infectious disease-causing agents and their basis for establishing 
virulence.

As another example, syntrophic mutualism between a sulfate-reducing 
bacterium, Desulfovibrio vulgaris, and a methanogen, \textit{Methanococcus 
maripaludis}, was investigated by performing \gls{fba} on a compartment-based 
model involving the metabolic \glssymbol{reconstruction}s of both organisms 
and a culture medium \cite{Stolyar:2007jh}. In another study, gene expression data 
were integrated with the genome-scale metabolic \glssymbol{reconstruction} of 
\textit{P. aeruginosa} in the context of a chronic cystic fibrosis lung 
infection over a 44-month time-course \cite{Oberhardt:2010bp}. This analysis provided 
a systems-level view of bacterial adaptations in a cystic fibrotic 
lung environment over time. Subsequent studies can shed light on 
the interactions between multiple organisms over the time course 
of an infection.

Finally, automated \glssymbol{reconstruction} platforms such as ModelSEED 
permit the rapid \glssymbol{reconstruction} of hundreds of draft bacterial 
metabolic networks \cite{Overbeek:2005dv,Aziz:2008ku}. Integration of many such reconstructed 
networks may help elucidate interactions within the host 
microbiome and partially explain the development of opportunistic 
infections that occur primarily because of an altered bacterial 
flora and environment. For example, an integrative metabolic 
analysis of organisms in the human gut microbiome could aid 
understanding of the intricate balance between non-pathogenic 
and potentially pathogenic organisms during healthy and 
infectious states in the gastrointestinal tract. An expected 
outcome of such analyses could result in the selection of 
drugs or drug cocktails that specifically target pathogens 
without eliminating non-pathogenic members.

\section{Concluding remarks}
As \glssymbol{reconstruction}s of metabolic networks become more standard 
and automated \cite{Thiele:2010fr}, the need for computational tools to characterize 
these networks becomes more apparent. In addition, the generation 
and management of large datasets pertaining to both host cell 
and pathogen intracellular processes of metabolism, signal 
transduction or regulation has necessitated a systems approach 
and, therefore, the computational methods used to analyze 
these data are becoming increasingly important. 
Experimental methods will continue to improve, thereby 
generating data that have so far been either impossible 
or prohibitively laborious to obtain, and which have 
constrained the value of some model predictions. For 
example, TraDIS (a new experimental method used to 
identify all essential genes simultaneously) directly 
measures gene essentiality that the model could only 
predict \cite{Langridge:2009ip}. However, the iterative relationship between 
modeling and experiment will always permit the generation 
of novel hypotheses and the contextualization of large 
datasets, often in a quicker and more cost-efficient 
manner (e.g.\ rapid essentiality prediction of all double 
gene knockouts). Network-based approaches such as 
genome-scale metabolic \glssymbol{reconstruction}s have been effective 
in drug target prediction and will continue to expand 
in scope and applicability. In addition, integration 
of networks and data into more standard pipelines that 
traverse the spectrum from computational prediction to 
experimental evaluation and back again will speed the 
process dramatically. By including many types of data 
sources that have yet to be coupled, entirely new classes 
of drug targets or treatment strategies may be found. The 
already enormous amount of data is only increasing, and the 
use of systems biology approaches will be vital to driving 
future research of drug and drug target discovery against 
infectious diseases.

\section{Acknowledgements}
The majority of this chapter was taken from
\fullcite{Chavali:2012cp}. I was co-primary author
on this paper and would like to thank Dr. Chavali, the other
primary author, and all other co-authors.
We would like to thanks Glynis Kolling and Erwin 
Gianchandani for reading the manuscript and providing 
insightful feedback.


