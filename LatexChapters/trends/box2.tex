%%%%%%%%% BOX 2 %%%%%%%%%%%%%%
\begin{pabox}[label=trends:box:objective]{Cellular objectives of pathogenic organisms}
A \gls{biomass} is not applicable under 
all conditions, and very often growth alone may not be 
a realistic objective. Other \glssymbol{objectivereaction} such as maximizing 
or minimizing ATP or maximizing the production of 
particular cellular by-products (e.g.\ lactate or pyruvate) 
can also be used.

~~~~~~The metabolism of an organism may be adapted for increased 
virulence or pathogenicity. Pathogens and host cells may 
also temporarily opt for alternative objectives while under 
selective pressures (e.g.\ changes in nutrients or 
environmental influences of secreted toxins). In addition, 
different morphological stages of a pathogen (e.g.\ the sporozoite 
stage of Plasmodium falciparum in mosquitoes vs the merozoite 
stage in humans) may be characterized by varying metabolic 
requirements. Consequently, the \glssymbol{objectivereaction} function must be 
appropriately defined to find relevant enzyme targets crucial 
in particular stages of infection and/or environmental conditions.

~~~~~~To explore the effects of targeted perturbations 
(pharmacological or genetic), different \glssymbol{objectivereaction}s 
can be explored. It may be that the evolutionary 
pressures that dictate wild-type cells are different 
for knockout mutants. Additionally, mutants may 
not have the ability for immediate regulation of 
\glspl{flux} that allow for optimal growth. Based on these 
ideas, an approach termed minimization of metabolic 
adjustment (MOMA) was developed. The requirement for 
optimal growth is relaxed for gene deletions in MOMA. 
Instead, MOMA assumes that the overall \gls{fluxdistribution} 
of a gene-knockout mutant will probably not change 
significantly from that of the corresponding wild type [71]. 
In terms of \gls{flux} values, the gene-knockout mutant will 
remain as close as possible (in Euclidean distance) to 
the wild-type optimal \gls{fluxdistribution}. MOMA aided in correcting 
gene essentiality predictions associated with knockouts 
of fructose-1,6-bisphosphatate aldolase, triosephosphate 
isomerase and phosphofructokinase in the E. coli 
central metabolic model [71]. These genes were predicted 
to be nonessential when \glssymbol{biomass} was used as the \glssymbol{objectivereaction} 
for E. coli growth on glucose, which was inconsistent with 
supporting literature evidence. MOMA yielded a suboptimal 
\gls{fluxdistribution} for a knockout mutant that would not 
necessarily equal the optimum as dictated by traditional 
\gls{fba} [71].

~~~~~~Approaches such as MOMA that consider alternate hypotheses 
for the objective of metabolic networks provide a basis 
for understanding a potential biological goal for pathogenic 
organisms, especially considering the complexity of the 
environment surrounding the pathogen of interest.
\end{pabox}
