\begin{pabox}[label=trends:box:fva]{A brief primer on FVA}
After the application of various constraints on the 
biochemical network, the number of allowable network 
states (or possible phenotypes) is typically large. 
Depending on the size and interconnectedness of the 
cellular network, there may be several alternative 
optimal phenotypes [14]. \gls{fba} calculates one of many 
feasible solutions that result in the same optimal 
value of the cellular \glssymbol{objectivereaction}.

~~~~~~\gls{fva} calculates the range of 
\gls{flux} in each reaction that allows for the same optimal 
\gls{flux} through the \gls{objectivereaction}. The \gls{objectiveflux} 
of the reaction is specified as an additional 
constraint and multiple optimizations are performed 
to compute the maximum and minimum flux for every reaction in the network:
\begin{alignat*}{2}
\text{Max/Min  }   & v_{i}  \\
\text{subject to  } & Sv=0 \\
                    & v_{biomass} = Z_{obj} \\
                    & v_{min} \leq v_i \leq v_{max}
\end{alignat*}
Here, n refers to the number of reactions in the 
biochemical network, and Zobj is the optimal value 
of the cellular \glssymbol{objectivereaction} ($v_{biomass}$) as obtained by \gls{fba} [22].
\end{pabox}