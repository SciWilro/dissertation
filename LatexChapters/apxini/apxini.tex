\chapter[Reproducing in vivo analyses]{Instructions for reproducing in vivo transcriptomics analysis}

\section{Introduction}\label{S:Intro}
To ensure that all computational analyses, tables, and figures
presented are reproducible, we have provided the necessary data
files, scripts, and accompanying explanations. All analyses 
were performed with the free, open source R programming language
and R software packages within the Bioconductor project \cite{Gentleman:2004tt}. 
In addition
to the instructions included in this document, every script file
is well commented, providing further details. The supplemental figures
and associated text are in a separate document. 

\section{Preparing software, files, and folders}\label{S:Files} 

\subsection{Directory structure}
All of the scripts and data, except for the microarray data, are stored 
in a compressed folder that can be accessed at
\url{http://bme.virginia.edu/csbl/downloads-cdiff}. The scripts,
without the large annotation and data files in the \texttt{./data} folder, 
are also accessible from an online version control system at 
\href{https://bitbucket.org/kdauria/cdiff}{bitbucket.org}.

Once extracted, this root folder serves as the working 
directory for all scripts. The folder is heretofore
referred to with the path \texttt{./}~. Note that for path
names on Windows operating systems, backslashes are used instead of
forward slashes. Several physiological measurements are
included in the tab delimited text file
\texttt{./PhysiologyMeasurements.csv}. Files associated with 
different steps of the analysis are included
in subfolders corresponding to subsequent sections of this document.

\subsection{R and R packages}
Version 2.15.2 of R was run on a \texttt{x86\_{}64-pc-linux-gnu} platform. 
The following R packages were 
used \cite{Gautier:2004kv,Bolstad:2003ia,Durinck:2009ki,Wickham:2009tn,Smyth:2005um,Wickham:2007tu} :

\begin{table}[h]
  \begin{center}
    \begin{tabular}{ l l l l  }
\texttt{affy\_{}1.36.1} & \texttt{digest\_{}0.6.2} & 
\texttt{MASS\_{}7.3-23} & \texttt{reshape\_{}0.8.4}\\
\texttt{affyio\_{}1.26.0} & \texttt{gcrma\_{}2.30.0} & 
\texttt{Matrix\_{}1.0-11} & \texttt{reshape2\_{}1.2.2}\\
\texttt{affyPLM\_{}1.34.0} & \texttt{gdata\_{}2.12.0} & 
\texttt{mouse4302cdf\_{}2.11.0} & \texttt{RSQLite\_{}0.11.2}\\
\texttt{AnnotationDbi\_{}1.20.3} & \texttt{ggplot2\_{}0.9.3}
& \texttt{mouse4302.db\_{}2.8.1} & \texttt{scales\_{}0.2.3}\\
\texttt{Biobase\_{}2.18.0} & \texttt{GO.db} & 
\texttt{mouse4302probe\_{}2.11.0} & \texttt{splines\_{}2.15.2}\\
\texttt{BiocGenerics\_{}0.4.0} & \texttt{gplots\_{}2.11.0} & 
\texttt{munsell\_{}0.4} & \texttt{stats4\_{}2.15.2}\\
\texttt{BiocInstaller\_{}1.8.3} & \texttt{grid\_{}2.15.2} & 
\texttt{org.Mm.eg.db\_{}2.8.0} & \texttt{stringr\_{}0.6.2}\\
\texttt{biomaRt\_{}2.14.0} & \texttt{gtable\_{}0.1.2} & 
\texttt{parallel\_{}2.15.2} & \texttt{tcltk\_{}2.15.2}\\
\texttt{Biostrings\_{}2.26.3} & \texttt{gtools\_{}2.7.0} & 
\texttt{plyr\_{}1.8} & \texttt{tools\_{}2.15.2}\\
\texttt{bitops\_{}1.0-5} & \texttt{IRanges\_{}1.16.4} & 
\texttt{preprocessCore\_{}1.20.0} & \texttt{XML\_{}3.95-0.1}\\
\texttt{caTools\_{}1.14} & \texttt{KernSmooth\_{}2.23-8} & 
\texttt{proto\_{}0.3-10} & \texttt{zlibbioc\_{}1.4.0}\\
\texttt{colorspace\_{}1.2-1} & \texttt{labeling\_{}0.1} & 
\texttt{qvalue\_{}1.32.0} & \\
\texttt{DBI\_{}0.2-5} & \texttt{lattice\_{}0.20-13} & 
\texttt{RColorBrewer\_{}1.0-5} & \\
\texttt{dichromat\_{}2.0-0} & \texttt{limma\_{}3.14.4} & 
\texttt{RCurl\_{}1.95-3} & 

    \end{tabular}
  \end{center}
\end{table}

\subsection{Downloading the microarray data}
The microarray data files are deposited in the NCBI GEO database 
as a data series with the accession number \texttt{GSE44091}. The 
matrix of preprocessed data is also uploaded to GEO. However,
for the purposes of this analysis, the compressed folder of 
\texttt{.CEL} files must be downloaded. On the 
GEO web page for \texttt{GSE44091}, there should be a link
to download the compressed folder which is probably in
a \texttt{tar} format (i\@.e\@.~with a \texttt{tar}
file extension). Once extracted, all of the \texttt{.CEL}
files should be placed in the \texttt{./data/CEL.files/}
folder. Each of the 32 files is approximately ten megabytes.

\section{Loading and organizing microarray data}\label{S:Loading} 
The script \texttt{./A-LoadingData/loadData.R} parses the
\texttt{.CEL} files into \texttt{AffyBatch} objects and
and saves them in the \texttt{./data/RData/abatches.RData} file.
There are four \texttt{AffyBatch} objects for four groupings
of the microarrays: 2hr, 6hr, 16hr, and all arrays.
The endpoint and toxin injection associated with each
array are also labeled and can be accessed using the
\texttt{pData} function on each \texttt{AffyBatch}.

\section{Preprocessing microarray data}\label{S:Preprocessing} 
\texttt{./B-Preprocessing/preprocessData.R} preprocesses
the \texttt{AffyBatch} objects. Once preprocessed, each
\texttt{AffyBatch} object leads to an \texttt{ExpressionSet}
object which contains the signal intensities
for every probe set on each microarray. 
The \texttt{ExpressionSet} objects are stored in
\texttt{./data/RData/esets.RData}.

\section{Microarray and gene annotation}\label{S:Annotations} 

\subsection{Background}

On the Mouse 430 2.0 Affymetrix array used here, there are 
approximately $1002 \times 1002=1$,$004$,$004$ 
probes, each of which has a different 25 base pair oligomer. Affymetrix
used the GenBank and Unigene databases to choose the 
probe sequences that align with all the genes in the 
mouse genome. Approximately half of the probes are ``mismatch'' probes, probes for
which the sequence differs by one nucleotide from the corresponding ``perfect-match'' probe. 
These mismatch probes are intended to measure nonspecific binding. However,
how to best use the readings from mismatch probes is 
still unclear \cite{Wang:2007dy}.
Several processing techniques successfully use only the ``perfect match''
probes; we do the same here.

Affymetrix arranged probes into probe sets. Each probe set includes
approximately 10-20 probes and is intended to represent the sequence
of a specific gene or transcript. The Affymetrix probe set
IDs can in turn be linked to databases of other identifiers
(e\@.g\@. Entrez gene, Ensembl, MGI, etc\@.). In order to 
understand and interpret the signal intensities of the probes
in reference to commonly used gene names (e\@.g\@. Jun, Rhob, 
etc\@.), the annotations from probe to probe set to gene name
are exceedingly important. Multiple databases, which we employ here,
have arisen to address this need.

\subsection{Annotation sources}
Sequence annotations are constantly updated as new experimental
findings come to light. To ensure our annotations and subsequent
analyses using these annotations are repeatable, 
we have either provided the
version number of annotation packages used or downloaded annotation files
from public databases. All downloaded annotation files are in the
\texttt{./data/RData/annotation.files} folder;
the file names are made to be self explanatory. The
parsed annotations files resulted in R variables used in all further analyses. These
variables are stored in the \texttt{./data/RData/annotation.RData} folder.
One could therefore skip this section of the ths document and simply load
these variables into the R workspace.

\subsubsection{Affymetrix probe set IDs}
The Affymetrix
probe to probe set mappings were taken from the 
\texttt{mouse4302.db} and \texttt{mouse4302cdf} packages, 
version 2\@.8\@.1 and 2\@.11\@.0, respectively. 
The DAVID \cite{Huang:2009uz} and 
Biomart \cite{Kasprzyk:2011ky}
databases were used for linking
Affymetrix probe set IDs to gene symbols (e\@.g\@. Rhob), 
gene names (e\@.g\@. Ras homolog gene family, member B),
MGI IDs (e\@.g\@. MGI:107949), and Entrez IDs (e\@.g\@. 11852).
Version 2\@.4\@.1 of the \texttt{org.Mm.eg.db} package was also used.
The \texttt{biomaRt} package, version 2\@.14\@.0, was used
to access version 0\@.8 of the BioMart database.

\subsubsection{Gene set annotations}

Gene ontology gene associations with MGI gene IDs 
were downloaded from \href{http://geneontology.org}{geneontology.org} and
are saved in the annotation folder \cite{Ashburner:2000ja}. Gene 
assocations from the Reactome
database (\href{http://www.reactome.org}{reactome.org}) with human Entrez gene IDs were 
also downloaded \cite{Matthews:2009kj}.

\subsubsection{Orthology annotations}

The Reactome database is built using Human Entrez IDs. To be able to use
Reactome and other databases with human data, annotations were downloaded
from NCBI homologene and MGI's flat files in order
to map mouse genes to their human 
orthologs \cite{Eppig:2012ix}. In order to compare mouse and human
transcriptional date sets, Ensembl transcript IDs and gene IDs were mapped
using the Biomart database.

\subsubsection{Additional annotations}
\label{sS:newProbeSets}

The relationship between Affymetrix probe sets and gene IDs is typically
\emph{many-to-many}. In other words, there are probe sets that map
to \emph{many} genes, and there are genes that map to \emph{many}
probe sets. For example, there are two microarray probe sets
(\texttt{1418652\_at} and \texttt{1456907\_at}) that map to \emph{Cxcl9}.
Which probe set should then be used to show the data for \emph{Cxcl9}, or how should
these probe sets be combined into a single probe set? We address this 
selection/summarization problem with functions described in 
section \ref{sS:collapse} of this document. 

As an alternative, we also
aligned probe sequences to gene sequences from Ensembl \cite{Flicek:2012ks} in order to 
define custom probe sets with \emph{one-to-one} mappings to Ensembl gene IDs. 
These Ensembl-defined probe sets then made MGI-defined 
probe sets possible using a mapping from Ensembl to MGI gene IDs. To map MGI 
IDs to Ensembl Gene IDs, the Biomart database was used and
a mapping file from MGI was also downloaded. Mappings from Ensembl
transcript IDs to Ensembl Gene IDs were downloaded from Ensembl. Probe
sequences were taken from version 2\@.11\@.0 of the \texttt{mouse4302probe}
Bioconductor package and written to a FASTA file. NCBI BLAST-2\@.2\@.27+
was used to map probes which aligned 100\% with Ensembl transcript IDs which
then mapped to Ensembl Gene IDs. 

The strict requirements we set reduced the number of
probes ($310$,$467$ from {\raise.24ex\hbox{$\scriptstyle\sim$}}$450$,$000$) 
and probe sets ($21$,$288$ from $45$,$101$). Redifining probe sets 
thus lost {\raise.24ex\hbox{$\scriptstyle\sim$}}30\% of the microarray data. 
Since this probe set redefinition did not greatly affect our overall 
analyses, we continued with the Affymetrix
defined probe sets. A parallel analysis to the one presented in this document
(but with our redefined probe sets) is contained in the
\texttt{./DiffProbeSets} folder yet is not discussed further.

\subsection{Generating annotation lists in R}

Several scripts generate the mappings from one type of 
ID to another. These mappings are contained in 
nested \texttt{list} objects (hereon referred to as annotation
lists).

\singlespacing
\begin{knitrout}
\definecolor{shadecolor}{rgb}{0.969, 0.969, 0.969}\color{fgcolor}\begin{kframe}
\begin{alltt}
\hlcomment{# Examples of annotation lists}
\hlfunctioncall{load}(\hlstring{"./data/annotation.RData/MsAnn.RData"})
\hlfunctioncall{load}(\hlstring{"./data/annotation.RData/MsAnnGO.RData"})
\hlfunctioncall{head}(MsAnn$MGI$Affy, 3)
\end{alltt}
\begin{verbatim}
## $`101757`
## [1] "1448346_at"   "1455138_x_at"
## 
## $`101758`
## [1] "1421566_at"
## 
## $`101759`
## [1] "1415844_at" "1415845_at"
\end{verbatim}
\begin{alltt}
\hlfunctioncall{head}(MsAnnGO$MGI$BP, 3)
\end{alltt}
\begin{verbatim}
## $`101757`
##  [1] "GO:0007010" "GO:0006928" "GO:0030010" "GO:0007015" "GO:0030030"
##  [6] "GO:0045792" "GO:0006468" "GO:0006606" "GO:0022604" "GO:0043200"
## [11] "GO:0000910" "GO:0001755" "GO:0030836" "GO:0001842"
## 
## $`101759`
## [1] "GO:0006810" "GO:0007269" "GO:0048489" "GO:0017158"
## 
## $`101760`
## [1] "GO:0006396" "GO:0006355" "GO:0006397" "GO:0008380" "GO:0006351"
## [6] "GO:0048025"
\end{verbatim}
\end{kframe}
\end{knitrout}
\doublespacing

The scripts which parse annotation files and access annotation databases
are within the .\texttt{./C-Annotations/} folder. The script names, 
the mappings the scripts generate, and in what variable the mappings (i.e. the
annotation lists) are stored 
are shown in Table \ref{Ta:annotations}. The variables are saved in
similarly named \texttt{.RData} files in the
\texttt{./data/annotation.RData} folder.

\begin{table}[h]
  \begin{center}
    \begin{tabular}{ l | l | l }
      Script   &    R variable    &    ID mappings   \\ \hline
      \texttt{ProbeSetToOthers.R} & \texttt{MsAnn} & Affy $\leftrightarrow$ MGI, Entrez, Gene Symbol, Gene Name \\                                              
      \texttt{GOAnnotations.R} & \texttt{MsAnnGO} & MGI $\leftrightarrow$ Gene Ontology Term \\
      \texttt{OtherGenesetAnnotations.R} & \texttt{GSAnn} & Human Entrez $\leftrightarrow$ REACTOME \\
      \texttt{OrthologAnnotations.R} & \texttt{MsAnn} & Human Entrez $\leftrightarrow$ Entrez, Affy \\
      \texttt{} & & Human Ensembl $\leftrightarrow$ Ensembl \\
      \texttt{EnsemblToOthers.R} & \texttt{MsAnn} & Ensembl gene $\leftrightarrow$ MGI \\
      \texttt{NewProbeSets.R} & \texttt{CustomPS} & Affy probe $\leftrightarrow$ Ensembl, MGI \\
      \texttt{GeneSymbols.R} & \texttt{MsAnn} & MGI $\leftrightarrow$ Gene Symbol
    \end{tabular}
    \caption{Annotation scripts and variables. ``Affy''=Affymetrix
             Mouse 430 2.0 probeset ID. ``Entrez''=Entrez gene ID,
             ``Ensembl''=Ensembl gene ID}\label{Ta:annotations}
  \end{center}
\end{table}

The data from our previous study of the HCT8 cell transcriptional response 
to TcdA and TcdB was reanalyzed in the same way this \emph{in vivo} data is analyzed. 
The annotation files are given in Table \ref{Ta:HSannotations}. The scripts are located in the 
\texttt{./InVitro/C-Annotations/} folder and the R variables are saved to the
\texttt{./InVitro/data/annotation.RData/} folder.

\begin{table}[h]
  \begin{center}
    \begin{tabular}{ l | l | l }
      Script   &    R variable    &    ID mappings   \\ \hline                                         
      \texttt{makeAnnotations.R} & \texttt{HsAnn} & Affy $\leftrightarrow$ Gene Symbol, Entrez, Gene Name \\
      \texttt{} & \texttt{HsAnnGO} & Symbol $\leftrightarrow$ Gene Ontology Term \\
      \texttt{NewProbeSetsHS.R} & \texttt{HsAnn} & Ensembl $\leftrightarrow$ Probe, Entrez, Gene Symbol \\
    \end{tabular}
    \caption{Annotation scripts and variables for Hgu133 Plus 2.0 chip }\label{Ta:HSannotations}
  \end{center}
\end{table}

\subsection{Transitive mapping}
Not all gene, probe, and probe set identifiers have direct mappings
available. However, it is sometimes possible
to map one identifier to another using intermediate mappings.
In symbolic terms, for IDs $a$,$b$, and $c$, if there are mappings $a \rightarrow b$
and $b \rightarrow c$, then an $a \rightarrow c$ mapping can be made. The
\texttt{TransitiveMapping} function in the \texttt{./C-Annotations/TransitiveMapping.R}
script generates such ``transitive mappings'' when given two mappings
related by one common ID. Annotation lists or two-column
matrices are taken as input. The function uses simple matrix multiplication for
efficient mapping, typically taking $<$5s for many genome-wide mappings.

\singlespacing
\begin{knitrout}
\definecolor{shadecolor}{rgb}{0.969, 0.969, 0.969}\color{fgcolor}\begin{kframe}
\begin{alltt}
\hlcomment{# An example of transitive mapping}
\hlfunctioncall{source}(\hlstring{"./C-Annotations/TransitiveMapping.R"})
\hlfunctioncall{head}(\hlfunctioncall{TransitiveMapping}(MsAnn$MGI$Affy, MsAnnGO$MGI$BP), 4)
\end{alltt}
\begin{verbatim}
## $`1415670_at`
## [1] "GO:0006810" "GO:0006886" "GO:0015031" "GO:0016192" "GO:0072384"
## [6] "GO:0051683"
## 
## $`1415671_at`
## [1] "GO:0006810" "GO:0006811" "GO:0006200" "GO:0015992" "GO:0034220"
## [6] "GO:0015991"
## 
## $`1415672_at`
## [1] "GO:0006893"
## 
## $`1415673_at`
## [1] "GO:0008152" "GO:0008652" "GO:0006564" "GO:0006563"
\end{verbatim}
\end{kframe}
\end{knitrout}
\doublespacing


\subsection{Many probes to many genes mapping}
\label{sS:collapse}

As introduced in section \ref{sS:newProbeSets}, the relationship between 
Affymetrix probe sets and gene IDs is typically many-to-many. 
For example, one usually wants 
to report the expression of gene $A$, not the expression of 
probe set 1, probe set 2, and probe set 3 that map to gene $A$. There are two
solutions to this \emph{many probe set to gene} problem: (1) selecting only one of the several 
probe sets for each gene (selection), and (2) combining all of the probe sets 
into a new probe set (gene summarization).

Continuing with this example, a further problem may be 
that probe set 2 also maps to gene $B$.
Hence, it is possible that one probe set may represent or 
contribute to the measured expression of many different genes. This
\emph{probe set to many gene} problem is more difficult to resolve.
Perhaps the most straightforward solution is complete reannotation of
the microarray as discussed in section \ref{sS:newProbeSets}. We
reannotated the microarray to contain only one-to-one relationships, yet found that 
reducing the many-to-many relationship to only a many-to-one (not one-to-one) relationship 
led to similar results. The many-to-one relationship also kept the data
for many microarray probes that must be discarded
in the generation of one-to-one mappings.

The script \texttt{./C-Annotation/CollapseIDs.R} contains functions which
will ``collapse'' genes with multiple probe sets into a single
measurement, either by selection of a probe set or by gene summarization. In other
words, data with probe set IDs is converted to data with commonly used and
understandable gene IDs, even when given a many-to-many probe set to gene 
mapping. The criteria for selecting which probe set represents a gene or 
the criteria by which several probe sets are summarized into a gene are 
inputs to the \texttt{collapseExprMatrix} and \texttt{collapseExprVector}
functions defined in the \texttt{CollapseIDs.R} script. By default, the probe set
with the largest interquartile range is selected. For gene summarization,
the mean of probe sets is taken to represent the expression for a gene.

\singlespacing
\begin{knitrout}
\definecolor{shadecolor}{rgb}{0.969, 0.969, 0.969}\color{fgcolor}\begin{kframe}
\begin{alltt}
\hlcomment{# An example of 'collapsing' an expression matrix}
\hlfunctioncall{library}(affy)
\hlfunctioncall{source}(\hlstring{"./C-Annotations/CollapseIDs.R"})
\hlfunctioncall{load}(\hlstring{"./data/RData/esets.RData"})
data = \hlfunctioncall{exprs}(esets[[\hlstring{"6hr"}]])
\hlfunctioncall{head}(data, 3)
\end{alltt}
\begin{verbatim}
##              213   214   215   216   217   218    221   223   224   232
## 1415670_at 8.514 8.752 8.630 8.317 8.468 8.504  8.647 8.593 8.564 8.414
## 1415671_at 9.748 9.626 9.627 9.763 9.804 9.859  9.668 9.823 9.929 9.950
## 1415672_at 9.892 9.910 9.993 9.635 9.682 9.588 10.079 9.551 9.609 9.566
\end{verbatim}
\begin{alltt}
\hlfunctioncall{head}(\hlfunctioncall{collapseExprMatrix}(data, MsAnn$MGI$Affy, type = \hlstring{"select"}), 3)
\end{alltt}
\begin{verbatim}
##          213   214   215   216   217   218   221   223   224   232
## 101757 8.570 8.732 8.784 8.652 8.465 8.411 8.698 8.582 8.522 8.091
## 101758 2.025 1.844 1.720 2.214 1.491 2.178 1.844 1.654 1.992 1.832
## 101759 1.838 1.993 2.089 1.840 1.921 2.012 1.913 1.981 2.128 1.780
\end{verbatim}
\end{kframe}
\end{knitrout}
\doublespacing


\section{Analyzing differential expression}\label{S:DE}

The CyberT statistical test \cite{Baldi:2001ul} was used to generate p-values for the
differential expression of probe sets after toxin injection relative to sham
injection. The CyberT R source code was downloaded from 
\url{http://cybert.ics.uci.edu}; this source code
is in the \texttt{./D-DiffExpression/bayesreg.R} script. We wrote the
\texttt{cybertWrapper} function to automatically format \texttt{ExpressionSet}
objects for the \texttt{bayesT} function in the \texttt{bayesreg.R} script. 
The wrapper function automatically sets up toxin versus sham comparisons according to the
phenotype data within each \texttt{ExpressionSet}. Hence, the wrapper function
is highly customized for this study. The wrapper function outputs
a list containing matrices of parameters and p-values; the columns of each
matrix represent the different two-class comparisons. The 
\texttt{./D-DiffExpression/cybertRun.R} script applies the \texttt{cybertWrapper}
for all of the toxin-sham comparisons and saves the results to 
\texttt{./data/RData/cyberT.results.RData}.

\singlespacing
\begin{knitrout}
\definecolor{shadecolor}{rgb}{0.969, 0.969, 0.969}\color{fgcolor}\begin{kframe}
\begin{alltt}
\hlcomment{# Part of the output from CyberT test for differential expression}
\hlfunctioncall{load}(\hlstring{"./data/RData/cyberT.results.RData"})
\hlfunctioncall{head}(cyberT.results$pval, 3)
\end{alltt}
\begin{verbatim}
##            A2-Sham2 B2-Sham2 AB2-Sham2  A6-Sham6 B6-Sham6 AB6-Sham6
## 1415670_at   0.1456   0.6577    0.2659 0.3397942   0.4228 2.696e-01
## 1415671_at   0.4098   0.8156    0.4063 0.0174292   0.3247 9.285e-03
## 1415672_at   0.0409   0.8617    0.2724 0.0007157   0.6105 3.125e-05
##            A16-Sham16 B16-Sham16
## 1415670_at  0.0003463    0.08325
## 1415671_at  0.2354018    0.79854
## 1415672_at  0.3446540    0.16402
\end{verbatim}
\begin{alltt}
\hlfunctioncall{head}(cyberT.results$tstats, 3)
\end{alltt}
\begin{verbatim}
##            A2-Sham2 B2-Sham2 AB2-Sham2 A6-Sham6 B6-Sham6 AB6-Sham6
## 1415670_at   1.5087   0.4492    1.1417   0.9758  -0.8169     1.136
## 1415671_at  -0.8402  -0.2361   -0.8466  -2.5710  -1.0073    -2.879
## 1415672_at  -2.1721  -0.1762   -1.1257   3.9298   0.5167     5.345
##            A16-Sham16 B16-Sham16
## 1415670_at     4.1950     1.8108
## 1415671_at    -1.2185     0.2582
## 1415672_at    -0.9649    -1.4376
\end{verbatim}
\end{kframe}
\end{knitrout}
\doublespacing


\section{Gene set enrichment}\label{S:Enrichment}

\subsection{Competitive}
Competitive gene set enrichment was performed with the method CAMERA 
developed by Wu \emph{et al.} \cite{Wu:2012kp}.
The \texttt{camera} function, in the limma Bioconductor package,
calls other functions within limma to 
determine t-statistics for every gene. These t-statistics are then used 
in gene set enrichment. We modified \texttt{camera} to be able to accept
other t-statistics. Our modified function \texttt{cameraMod} is defined
in the script \texttt{./DiffExpression/cameraMod.R}.

Since gene sets are made from gene-level IDs such as MGI IDs or 
Entrez IDs, the data with Affymetrix probe sets must be collapsed
to data with gene-level IDs (see section \ref{sS:collapse} for more about collapsing).
Since CAMERA calculates gene-gene correlations, it must begin with
the data for each microarray (the \texttt{ExpressionSet} objects). The
\texttt{ExpressionSet} objects were collapsed, and CyberT was applied to obtain
test statistics for each gene ID.

\singlespacing
\begin{knitrout}
\definecolor{shadecolor}{rgb}{0.969, 0.969, 0.969}\color{fgcolor}\begin{kframe}
\begin{alltt}
\hlcomment{# saving the mapping when a matrix is collapsed}
data = \hlfunctioncall{exprs}(esets[[\hlstring{"6hr"}]])
result = \hlfunctioncall{collapseExprMatrix}(data, MsAnn$MGI$Affy, type = \hlstring{"select"}, func = \hlfunctioncall{function}(x) \hlfunctioncall{IQR}(x), 
    return.probes = TRUE)
collapsed.data = result$expMat
\hlfunctioncall{head}(result$collapseMap, 5)
\end{alltt}
\begin{verbatim}
##       101757       101758       101759       101760       101761 
## "1448346_at" "1421566_at" "1415844_at" "1438675_at" "1422851_at"
\end{verbatim}
\end{kframe}
\end{knitrout}
\doublespacing

Again, there are a few complications with annotations of genes and gene sets, which
we discuss using an example. Consider a gene set that includes genes $A$, $B$, \ldots , and $K$.
We then find that no probe sets map to genes $J$ or $K$. We then reduce the gene set to genes
for which there is data, genes $A$, $B$, \ldots , and $I$. Now consider that,
in collapsing probe sets to genes,  probe set $1$ was selected as the probe set
to represent the closely related genes $E$, $F$, $G$, $H$, and $I$. If probe set $1$
was strongly differentially expressed, then the majority of genes in this
gene set would be found to be differentially expressed, which may or may not
be the case. To avoid a probe set errantly making a gene set
significantly enriched, a probe set was not allowed to be repreated in a  gene set. 
In this example, the gene
set would then be genes $A$, $B$, $C$, $D$, and one gene represented by probe set $1$.
When gene set enrichment is performed on a collapsed matrix, it is thus necessary
to save which probes map to genes (note the \texttt{return.probes} option in the
previous example of the \texttt{collapseExprMatrix} function). Finally, gene sets with
few genes are more likely to be enriched by one or two outliers, and
gene sets with several hundred genes are often too vague for interpretation. Hence,
gene sets with very few or several hundred genes were excluded. The function
\texttt{indexGeneSets} accounts for all the problems mentioned above; it also
converts the format of the gene set annotation lists so that they may be
used with \texttt{camera} and \texttt{cameraMod}.

A wrapper function, \texttt{cameraModWrapper} in \texttt{cameraModWrapper.R}, 
uses the CyberT t statistic as input to \texttt{cameraMod}. The wrapper function
is used in the script \texttt{runCAMERA.R} and the results are saved to
\texttt{./data/RData/cameraMod.RData}.

\subsection{Self-contained}

The function for the self contained test, \texttt{cactus}, is defined in the
\texttt{./E-Enrichment/cactus.R} script. The \texttt{runCactus.R} script
runs \texttt{cactus} for every toxin-sham comparison and saves the p-values
from each gene set enrichment in the \texttt{./data/RData/cactus.RData} file.
The annotation problems mentioned for competitive gene set enrichment
were addressed in the same way for self-contained gene set enrichment.

\singlespacing
\begin{knitrout}
\definecolor{shadecolor}{rgb}{0.969, 0.969, 0.969}\color{fgcolor}\begin{kframe}
\begin{alltt}
\hlcomment{# Example of p-values from gene set enrichment}
\hlfunctioncall{source}(\hlstring{"./E-Enrichment/cameraModWrapper.R"})
eset = esets[[\hlstring{"16hr"}]]
result = \hlfunctioncall{cameraModWrapper}(eset, MsAnnGO$MF$MGI, \hlfunctioncall{list}(\hlfunctioncall{c}(\hlstring{"A16"}, \hlstring{"Sham16"})), MsAnn$MGI$Affy)
\hlfunctioncall{head}(\hlfunctioncall{sort}(result, decreasing = FALSE))
\end{alltt}
\begin{verbatim}
## GO:0043498 GO:0005525 GO:0003924 GO:0004364 GO:0005097 GO:0047485 
##  0.0002763  0.0021449  0.0022396  0.0031255  0.0038467  0.0043996
\end{verbatim}
\begin{alltt}
\doublespacing

\hlfunctioncall{source}(\hlstring{"./E-Enrichment/cactus.R"})
eset = esets[[\hlstring{"2hr"}]]
result = \hlfunctioncall{cactus}(eset, MsAnnGO$BP$MGI, \hlfunctioncall{list}(\hlfunctioncall{c}(\hlstring{"A2"}, \hlstring{"Sham2"}), \hlfunctioncall{c}(\hlstring{"B2"}, \hlstring{"Sham2"})), 
    MsAnn$MGI$Affy)
\hlfunctioncall{head}(result, 4)
\end{alltt}
\begin{verbatim}
##            A2-Sham2 B2-Sham2
## GO:0000038  0.53392   0.3159
## GO:0000045  0.07490   0.3596
## GO:0000060  0.04644   0.3564
## GO:0000070  0.65147   0.3277
\end{verbatim}
\end{kframe}
\end{knitrout}
\doublespacing

\section{Analysis of previous \emph{in vitro} data}

The analysis described in the previous sections was repeated for the transcriptional
response of HCT8 cells exposed to TcdA or TcdB \cite{DAuria:2012bd}. The scripts and files for this
analysis are in the \texttt{./InVitro/} folder. The directory structure is the same
as for analysis of our \emph{in vivo} data.

Our reanalysis of the \emph{in vitro} data uncovered a clear
batch effect between different runs of microarrays. The batch effect had little
effect on the ranking of genes or the gene set enrichment results we have
reported previously. Nevertheless, we corrected for this batch effect using
a simple linear model. Factors for the three different runs were included in the
model. A principal components plot of all the microarrays before and after 
batch correction are shown in Figure \ref{fig:batch}.

\begin{figure}[ht]
\centering
\includegraphics{apxini/batch.pdf}
\caption{Batch corrected HCT8 transcriptional data}
\label{fig:batch}
\end{figure}

\section{Comparisons to other transcriptomic and proteomic data}

Proteomic data from Zeiser \emph{et al.} was downloaded from the supplementary
material of their manuscript \cite{Zeiser:2013cu}. The transcriptional data with HCT8 cells was
downloaded from NCBI GEO. The scripts parsing these data are in the
\texttt{./G-Comparisons} folder. In merging the data between mouse and human, 
only one-to-one mappings were used.

\section{Cytotoxicity assay}

After toxin addition, measurements for the cytotoxicity assay 
were taken 6 times/minute. In order to have enough replicates, 
two plates were used. Toxin was added to the columns of each
plate using a multichannel pipette. The time at which toxin
was added to each column was recorded manually. This information was used
to interpolate the measurements from all wells to the exact same time points. 
The plotting functions
and calculations are in the \texttt{./F-Figures/CytotoxAssay.R} script. The
data is in the \texttt{./data/CytotoxData.txt} file.

\section{Scripts for generating figures and tables}\label{S:Figures}

Most figures and tables were produced from R scripts.
Unless otherwise noted, the scripts in Table \ref{Ta:figureScripts} are in the
\texttt{./F-Figures} folder. Many of these scripts depend on the scripts
and results described in the previous sections.

\begin{table}[h]
  \begin{center}
    \begin{tabular}{ l l  }
      Figure or Table   &  Script   \\ \hline
      Table 1 & \texttt{Table1.R} \\
      Tables 2 \& 3 & \texttt{Tables2and3.R} \\
      Figure 2 & \texttt{Figure2.R} \\
      Figure 3 \& S6 & \texttt{Figure3.R} \\
      Figure 4 & \texttt{Figure4.R} \\
      Figure S1, S3, S7, and S8 & \texttt{Supplement.R} \\
      Tables S1 \& S2 & Manually entered \\
      Figure S2 & \texttt{CytotoxAssay.R} \\
      Figure S4 & \texttt{flow.R} \\
      Figure S5 & \texttt{./G-Comparisons/Hirota-et-al/HirotaCytokines.R} \\
      Figures S9 & \texttt{./G-Comparisons/HCT8-cells/Sims.R} \\
      \hspace{1cm} \& S10 & \& \texttt{./G-Comparisons/Proteomics/loadData.R}
    \end{tabular}
    \caption{Scripts for producing tables and figures}\label{Ta:figureScripts}
  \end{center}
\end{table}

\section{Notes on formatting}\label{S:Formatting}

This document is written using \LaTeX{} and the \texttt{knitr} software 
package from Yihui Xie \cite{Xie:2013vq}; the document files are in the \texttt{./LaTeX} folder.
The figures were exported to \texttt{pdf} or \texttt{svg} files and 
further formatted with Adobe Illustrator.

\section{Acknowledgements}
This appendix is part of the supplement for \fullcite{DAuria:2013jo}.




