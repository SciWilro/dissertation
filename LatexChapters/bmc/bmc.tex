
\chapter{ Toxins A and B disrupt the cell cycle }\label{chapter:bmc}

\section{Synopsis}
Toxins A and B (TcdA and TcdB) are \textit{Clostridium difficile}'s 
principal virulence factors, yet the pathways by which they lead 
to inflammation and severe diarrhea remain unclear. Also, the 
relative role of either toxin during infection and the differences 
in their effects across cell lines is still poorly understood. 
To better understand their effects in a susceptible cell line, 
we analyzed the transciptome-wide gene expression response of human 
ileocecal epithelial cells (HCT-8) after 2, 6, and 24 hr of toxin exposure. 
We show that toxins elicit very similar changes in the gene 
expression of HCT-8 cells, with the TcdB response occurring sooner. 
The high similarity suggests differences between toxins are due 
to events beyond transcription of a single cell-type and that 
their relative potencies during infection may depend on differential 
effects across cell types within the intestine. We next performed an 
enrichment analysis to determine biological functions associated with 
changes in transcription. Differentially expressed genes were 
associated with response to external stimuli and apoptotic 
mechanisms and, at 24 hr, were predominately associated with 
cell-cycle control and DNA replication. To validate our systems 
approach, we subsequently verified a novel G$_{\text{1}}$/S and 
known G$_{\text{2}}$/M cell-cycle block and increased apoptosis 
as predicted from our enrichment analysis. 
This study shows a successful example of a workflow 
deriving novel biological insight from transcriptome-wide 
gene expression. Importantly, we do not find any significant 
difference between TcdA and TcdB besides potency or kinetics. 
The role of each toxin in the inhibition of cell growth and 
proliferation, an important function of cells in the 
intestinal epithelium, is characterized.


\section{Background}
\textit{C. difficile}, a Gram-positive, spore-forming 
anaerobe, colonizes the human gut and causes infections 
leading to pseudomembranous colitis.  This opportunistic 
pathogen flourishes in antibiotic-treated and immunocompromised 
patients and is frequently spread in hospitals, although 
community-acquired \textit{Clostridium difficile} 
infection (CDI) cases have also increased \cite{Freeman:2010bv}. 
The emergence of hypervirulent strains that possess more 
robust toxin production and increased sporulation has been 
correlated with outbreaks across Europe and North 
America \cite{Warny:2005kd}. In most areas, the number 
of cases has increased in the past decade. The number 
of patients hospitalized in the US with CDI doubled to 
approximately 250,000/year (from year 2000 to 2003) and 
fatalities increased at a similar rate \cite{Zilberberg:2008gd}. 
The US healthcare costs for CDI are estimated to be over \$1 
billion/year \cite{Dubberke:2009ic}. As TcdA and TcdB appear 
to be responsible for many of the clinical manifestations of 
CDI, understanding the intracellular and systemic effects 
of each toxin is critical to developing and improving strategies 
for treatment and prevention.

In light of the multiple events and pathways involved in 
the development of CDI, we chose to examine the toxins' 
effects from a systems perspective, focusing on epithelial 
cells in vitro. Both TcdA and TcdB bind to 
cells \cite{Frisch:2003ul}, enter an endosome by clathrin-mediated 
endocytosis \cite{Papatheodorou:2010io}, translocate and then 
cleave their catalytic domain into the cytosol which 
glucosylates and so inactivates Rho family 
proteins \cite{Egerer:2007fy}. The disruption of these crucial 
signaling regulators begins to explain cytotoxic effects 
such as deregulation of the cytoskeleton and the breakdown 
of the epithelial barrier \cite{Nusrat:2001cs}. However, 
other processes are likely affected by the trafficking 
and processing of these toxins. In addition, secondary 
effects of Rho glucosylation in relation to pathologies 
of CDI have not been fully elucidated.

We therefore investigated the transcriptional profile of 
HCT-8 \cite{Tompkins:1974ud} cells treated with TcdA or 
TcdB and identified pathways and cellular functions 
associated with differentially expressed genes. With 
respect to toxins, in vitro analyses of gene expression in 
host cells have been performed with type A botulinum 
neurotoxin, lethal toxin \cite{Comer:2005iza} and edema 
toxin \cite{Comer:2006wta} from \textit{Bacillus anthracis}, 
pertussis toxin \cite{Lu:2008cr}, Shiga toxin type 
1 \cite{LeyvaIllades:2010fv}, and several others. Such 
studies provide lists of differentially expressed genes or 
classes of genes that serve as a resource for the 
generation of new hypotheses. In this regard, we used 
bioinformatics analyses to identify cellular functions 
altered by TcdA and TcdB that are relevant to pathogenicity. 
The correct identification of the majority of functions 
found to be affected in previous research regarding TcdA 
and TcdB confirmed our analysis and experimental design, 
and experiments reported herein validated changes in cell 
function that were suggested by altered gene expression.

Among the genes that TcdA and TcdB affect, many are 
involved in the regulation of the cell cycle and induction 
of apoptosis. Bacterial factors such as cytotoxic necrotizing 
factor and cytolethal distending toxins that disrupt normal 
cell cycle progression have been described as ``cyclomodulins'' 
\cite{Nougayrede:2005dm}. In addition to effects of TcdA and 
TcdB on cells in the G$_{\text{2}}$/M phase which have been 
described previously 
\cite{Fiorentini:1998uh,Kim:2005km,Gerhard:2008wz,Nottrott:2007ep}, 
we found that TcdA and TcdB affect expression of cyclins and 
cyclin-dependent kinase (CDK) inhibitors controlling the 
G$_{\text{1}}$--\,S transition. Our experiments establish 
that alterations of cell cycle implicated in our analysis 
of gene expression do, in fact, occur in toxin-treated cells. 
In addition to effects on cell cycle, we also present the 
other cellular functions associated with differentially 
expressed genes, some of which enable novel hypotheses 
on the cellular activity and function of these toxins.

\section{Methods}\label{bmc:s:methods}

\subsection{Cell Culture}
HCT-8 cells were cultured in RPMI-1640 supplemented 
with 10\% heat-inactivated fetal bovine serum (Gibco) 
and 1mM sodium pyruvate (Gibco).  The cultures were 
maintained at 37\textdegree{}C/5\% \ce{CO2} up to 
passage 35. Toxin A and Toxin B, isolated from strain 
VPI-10643, were a generous gift from David 
Lyerly (TECHLAB Inc., Blacksburg, VA).

\subsection{Microarrays}
HCT-8 cells (5 x 106/flask) were grown overnight at 
37\textdegree{}C/5\% \ce{CO2}.  Media were replaced 
with 2.5 ml fresh media and  toxin was added (100 ng/ml). 
At the end of the indicated incubation period, cells 
were washed with 5 ml PBS (Sigma) and total RNA was isolated 
using the QIAshredder and RNeasy mini kits (Qiagen), 
according to the manufacturer's instructions. An RNase 
inhibitor was added (RNasin, Promega) and samples were 
stored at -80\textdegree{}C. RNA integrity was assessed 
using an Agilent 2100 BioAnalyzer prior to cDNA 
synthesis and RNA labeling using either the 3$'$ IVT 
expression or one-cycle target labeling methods (Affymetrix). 
Biotin-labeled RNA was hybridized to Human Genome U133 Plus 
2.0 chips, washed, stained and scanned using a GeneChip 
System 3000 7G (Affymetrix). Data from three independent 
microarray experiments were deposited into the NCBI Gene 
Expression Omnibus repository (GSE29008).

\phantomsection
\label{bmc:methods:informatics}

Microarray signal intensities were normalized using the gcrma 
package \cite{Wu:2004wh}. Treatment and control groups were 
contrasted with linear models; a Benjamini-Hochberg 
correction was applied across all the probes and the nestedF 
method within the limma software package was used for multiple 
testing across all contrasts \cite{Smyth:2004gh,Smyth:2005ht}. 
The Gene Ontology (GO) annotation database was used to map gene 
symbols to GO categories \cite{Barrell:2009br}. A gene symbol 
was considered differentially expressed if at least one of the 
probe sets annotated to it was significant. A probe set was 
considered significant if the p<0.1 and the magnitude of the 
fold change was above 1.5. Enriched GO categories were identified 
with the topGO package using Fisher's exact test to calculate 
p-values and the elim algorithm \cite{Alexa:2006hg}.

\subsection{Flow Cytometry}
HCT-8 cells were grown overnight to 50\% confluence, media 
were removed and replaced with fresh media, and toxin was added 
at the concentrations denoted in the text and figures. At 24h 
and 48h, non-adherent cells were removed and saved on ice. Adherent 
cells were treated with 1mL of 0.25\% trypsin and 1 mL of 
Accutase with EDTA for 30 min at room temperature and joined 
with the non-adherent cells in 5 mL PBS. After centrifugation, 
resuspension for counting cells, and another round of centrifugation, 
the dissociated cells were resuspended to 2$\times10^{6}$ cells/mL and 
0.5 mL was added to 5mL of 70\% ice-cold ethanol for fixation. 
Afterward, the fixed cells were resuspended in 5 ml PBS with 
2\% Bovine Serum Albumin and then resuspended and incubated 
for 30 min in a solution to stain DNA (PBS with 10\% Triton 
X-100, 2\% DNasefree RNase, 0.02\% propidium iodide(PI)). 
Single-cell fluorescence was measured with a Becton Dickinson 
FACSCalibur flow cytometer. The proportion of cells in each 
stage of the cell cycle was calculated using ModFit cell 
cycle analyzer. The 24h-samples were imaged with an Amnis 
Imagestream imaging flow cytometer, which photographs the 
bright field and fluorescent channels from every cell 
individually \cite{George:2004jj}. Using Amnis software, a 
bivariate gate—based on the contrast of the brightfield image 
and the area of nuclear stain—differentiated apoptotic and 
non-apoptotic cells \cite{Henery:2008jz}. All other image 
features were taken from the Amnis software.

\subsection{Quantitative real time PCR }
For each gene examined, primers were designed from the target 
sequences retrieved from the RefSeq Sequence Database, using the 
Primer Express 3.0 software (Applied Biosystems). Primers were 
then custom made through Invitrogen Oligo Program. RNA quality 
control was carries out using an Agilent 2100 BioAnalyzer. 
Approximately 2 \textmugreek{}g of each RNA sample was converted 
to cDNA. Quantitative PCRs were carried out in triplicates 
using equal amounts of each cDNA sample approximately equivalent 
to 50 ng of starting total RNA. Power SYBR Green Master 
Mix (Applied Biosystems, PN 4367660) was used with the 
respective forward and reverse primers at the optimized 
concentrations. Amplifying PCR and monitoring of the 
fluorescent emission in real time were performed in the ABI 
Prism 7900HT Sequence Detection System (Applied Biosystems) 
as described (ABI SYBR Green Protocol).  To verify that only 
a single PCR product was amplified per transcript, dissociation 
curve data was analyzed through the 7900HT Sequence Detection 
Software (SDS).  To account for differences in starting 
material, quantitative PCR was also carried out for each 
cDNA sample using housekeeping genes synthesized in-house, 
human glyceraldehyde-3-phosphate dehydrogenase (GAPDH) and 
hypoxanthine phosphoribosyltransferase I (HPRT1).  The data 
collected from these quantitative PCRs defined a threshold 
cycle number (Ct) of detection for the target or housekeeping 
genes in each cDNA sample. The relative quantity (RQ) of 
target, normalized to geometric means of the housekeeping 
genes and relative to a calibrator (the Rox reference dye 
in this case), is given by $RQ = 2 –\Delta \Delta Ct$ 
where $\Delta \Delta Ct$ represents the difference in 
Ct between the transcript and the housekeeping gene for the 
same RNA sample. The ratio of the RQs for the treated 
sample and the experiment sample was used to derive the fold change.
 
\subsection{Western Blots}
After toxin treatment, non-adherent cells were removed 
and centrifuged. The adherent cells were then treated 
with 200 \textmugreek{}L of homogenization buffer: 10 mM HEPES 
pH 7.4, 150 mM \ce{NaCl}, 10 mM sodium pyrophosphate, 10 
mM \ce{NaF}, 10 mM EDTA, 4mM EGTA, 0.1 mM PMSF, 2 μg/mL 
CLAP (Chymostatin, Leupeptin, Antipain, and Pepstatin), 
and 1\% Triton X-100. Each well was scraped and the 
sample with homogenization buffer was added to the resuspended 
non-adherent cells. This was centrifuged at 15,000 rpm 
for 5 min at 4\textdegree{}C and the supernatant was 
boiled for 5 min, placed on ice for 5 min, boiled 
again for 5 min, and stored at -20\textdegree{}C. Samples 
containing equal amounts of protein were loaded 
into each lane of a 12.5\% polyacrylamide gel. Gels were 
electrophoresed, transferred to PVDF, and the membranes were 
blocked with 5\% skim milk in PBS, 0.1\% Tween-20 for 2h. 
Primary antibodies (Cyclin D1 \#2922, Cyclin E2 \#4132, p57 
Kip2 \#2557, p27 Kip1 \#2552 from Cell Signaling and Cyclin A 
sc-751 from Santa Cruz Biotechnology) were added to the 
blocking solution and membranes were incubated overnight 
at 4\textdegree{}C on a platform shaker. The membranes were 
then washed six times in PBS-Tween 20. The appropriate 
secondary antibodies, either anti-mouse or anti-rabbit 
HRP-linked antibodies (\#7076, \#7074, Cell Signaling), 
were added and the membranes were incubated for 2h on a 
platform shaker. The membranes were subsequently washed 
6 times and proteins were detected by chemiluminescence 
with ECL reagents (Amersham). To strip antibodies, the 
membranes were incubated in 50 mM Tris, 2\% SDS, 0.7\% 
\textbeta{}-mercaptoethanol at 50\textdegree{}C for 30 min. 
The membranes were then washed 6 times, blocked, and reprobed 
with a GAPDH antibody (ab8245 from Abcam) as described above.


\section{Results}

\subsection{Transcriptional Responses}

Overall, the changes in gene expression are consistent as 
time progresses, but the number of differentially expressed 
genes increases (\autoref{bmc:fig1}A). Specifically, at 2h 
and 6h, there are 4 and 134 probe sets differentially 
expressed (relative to control) for TcdA and 57 and 294 for 
TcdB, respectively (\autoref{bmc:fig1}C).  Many more are 
differentially expressed by 24h—1,155 and 1,205 in TcdA- and 
TcdB-treated cells, respectively. In order to validate these 
data, qRT-PCR was performed on 10 representative genes 
(r=0.89 by Pearson correlation; \autoref{bmc:figs1}A). Since 
the glucosylation of Rho family proteins occurs within one hour 
of toxin treatment \cite{ChavesOlarte:1997cs}, many of the 
differentially expressed genes at 24h may reflect secondary 
effects from the initial toxin action or possibly other unknown 
activities and processing of the toxin.

\begin{figure}[b!]
  \centering
  \includegraphics[width=0.75\columnwidth]{bmc/Figure1}
  \caption[Overall transcriptional response of HCT-8 
  cells to TcdA and TcdB]{
       \textbf{Overall transcriptional response of 
       HCT-8 cells to TcdA and TcdB.}
       \textbf{(A)} A heatmap shows the number of differentially 
       expressed probe sets at 2h, 6h, and 24h. The color scale 
       represents the fold change (binary log scale) of genes 
       relative to untreated cells at the same time point. 
       ``A, 2hr'' indicates the gene expression of cells 
       after 2h of TcdA treatment. TcdA and TcdB concentration 
       is 100 ng/ml.
       \textbf{(B)} The correlation of transcriptional 
       profiles between TcdA and TcdB at the indicated 
       time points are displayed in a correlation matrix. 
       The values represent the Pearson correlation 
       coefficient calculated from the fold change of 
       all the probe sets within the microarray.
       \textbf{(C)} The number of differentially expressed 
       genes used to identify enriched GO categories was 
       determined from a linear model 
       (see \ref{bmc:methods:informatics}).
}
  \label{bmc:fig1}
\end{figure}

\begin{figure}[h!]
  \centering
  \includegraphics[width=\columnwidth]{bmc/FigureS1}
  \caption[qRT-PCR validation of and genes with high differential expression]{
  \textbf{qRT-PCR validation of and genes with high differential expression}
  \textbf{(A)} Fold changes of gene expression relative 
  to controls as measured by microarray 
  were validated by qRT-PCR. To summarize 
  multiple probe sets within a microarray that 
  were annotated to the same gene, the value for 
  the probe set with the 
  greatest fold change is shown. The notation 
  ``A, 2hr'' indicates the expression 
  of genes treated with TcdA for 2 hr. 
  \textbf{(B)} This list consists of genes that 
  were among the top 25 differentially expressed 
  genes (according to fold change) in at 
  least 3 of the 6 conditions.}
  \label{bmc:figs1}
\end{figure}


Though the transcriptional responses to the two toxins 
are similar overall, a notable difference between the 
two toxins is that TcdB-induced changes occur more 
rapidly (\autoref{bmc:fig1}A). At the later time points, 
however, the overall transcriptional response induced by 
TcdA becomes more similar to the TcdB-induced transcriptional 
changes (see correlations in \autoref{bmc:fig1}B). Among 
the most affected genes, immediate early-response genes 
such as JUN, KLF2, and RHOB are upregulated 2h after toxin 
treatment and remain increased compared to untreated cells 
through 24 hr (\autoref{bmc:figs1}B). While identification 
of the most-affected genes provides important insight, 
focusing on a small subset risks overlooking other toxin 
effects key to the disease process. We therefore analyzed 
the expression data in the context of broad functional categories.

\subsection{Functions associated with differentially expressed genes}

\begin{figure}[h!]
  \centering
  \includegraphics[width=0.75\columnwidth]{bmc/Figure2}
  \caption[Gene ontology categories associated with differentially expressed genes]{
  \textbf{Gene ontology categories associated with differentially expressed genes.}
  \textbf{(A)} The most significantly enriched GO categories 
  (Fisher Exact Test, topGO elim algorithm \cite{Alexa:2006hg}, 
  see \ref{bmc:methods:informatics}) at 2h and 6h are displayed 
  in a heat map. The color intensity in each cell corresponds to 
  the p-value (Fisher Exact Test) for the GO category that is 
  enriched. The dendrograms were generated from a hierarchical 
  clustering of GO Groups according to Resnik similarity \cite{Resnik:1999jl}.
  \textbf{(B)} The most significantly enriched GO 
  categories at 24h were determined similarly
}
  \label{bmc:fig2}
\end{figure}

\begin{figure}[h!]
  \centering
  \includegraphics[width=\columnwidth]{bmc/FigureS2}
  \caption[Gene set enrichment of biological processes and cellular components]{
  \textbf{Gene set enrichment of biological processes 
  and cellular components.}
  \textbf{(A)} Cellular Component GO categories 
  with $\text{p}<10^{-3}$ across all time points 
  are shown. Criteria for calculating p
  values and GO categories were the same 
  as in Figure 2.
  \textbf{(B)} The 25 most significant GO 
  Biological Processes at 24 hr were selected 
  by the criteria described in \autoref{bmc:fig2}.}
  \label{bmc:figs2}
\end{figure}

We employed the GO database, which contains extensive 
annotation of biological functions associated with specific 
genes, to identify cellular phenotypes associated with 
changes in gene expression. The terms in this database are 
separated into three ontologies: Molecular Functions, Cellular 
Components, and Biological Processes (detailed descriptions 
at http://www.geneontology.org). A GO category—here defined 
as all the genes associated with a single GO term—with a 
proportion of differentially expressed genes greater than 
would be expected by chance is considered overrepresented 
or enriched (see \ref{bmc:methods:informatics}). While 
some enriched categories might have been anticipated, 
others provide novel insights. Within the Biological 
Processes ontology, the most significantly enriched 
categories at 2 and 6 hr are listed in \autoref{bmc:fig2}A. 
Within the Cellular Component ontology, the mitochondrial 
outer membrane and the apical junction complex category 
are enriched most significantly at 6h (\autoref{bmc:figs2}A). 
Interestingly, many of the functions related to the 
enriched categories have been linked with toxin treatment 
in previous work. One or both of the toxins induce 
activation of caspases 
\cite{Gerhard:2008wz, QaaposDan:2002uj, Carneiro:2006cw, Brito:2002ky}, 
damage mitochondria and cause the release of cytochrome c 
\cite{Matarrese:2007ix, He:2000uc}, increase oxygen radicals 
and expression of cytokines \cite{He:2002cl,Qiu:1999us,Flegel:1991ws}, 
alter MAPK signaling \cite{Meyer:2007kj,Lee:2007gj,Na:2005bx}, 
and disrupt the organization of tight junctions \cite{Nusrat:2001cs}. 
Hence, our analysis of gene expression as summarized in 
\autoref{bmc:fig2} is consistent with the previously reported 
cellular responses to these toxins.


The most significantly enriched categories for each 
toxin at the later time points are related to cell 
cycle and DNA replication (\autoref{bmc:fig2}B). Categories 
such as telomere maintenance and nucleosome assembly 
provide more specific connections between the toxins 
and changes in DNA replication. A more extensive list 
reveals that several categories related to microtubule 
organization during cell division are also enriched 
(\autoref{bmc:figs2}B). By 24 hr, there are changes 
related to virtually all elements of the cell cycle, 
but those controlling G$_{\text{1}}$ and S phases 
are more significantly affected. Though many of the 
genes within the enriched categories were not among 
the most differentially expressed genes, the abundance 
of differentially expressed genes involved in the same 
functions provides evidence for toxin effects on control 
of cell cycle at the G$_{\text{1}}$ phase. Cyclins and 
other proteins necessary for progression from the 
G$_{\text{1}}$ phase into and through the S phase 
are downregulated (\autoref{bmc:fig3}A). Cyclin 
proteins expressed at different points are central in 
coordinating entry into or exit from different phases. 
They specifically bind and activate particular CDKs which 
then phosphorylate downstream targets effecting 
progression \cite{Murray:2004ge}. Inhibitors of cyclin-CDK 
complexes from the INK4 family (p15, p16, p18, and p19) 
and Cip/Kip family (p21, p27, and p57) may suppress 
cyclin-CDK signaling \cite{Denicourt:2004bw}. Expression 
of many of these and other genes, such as CDC6 and CDC25A 
that are required for progression from G$_{\text{1}}$ 
to the S phase, is altered by TcdA and TcdB. The decreased 
expression of G$_{\text{1}}$ cyclins along with the 
increased expression of inhibitors of G$_{\text{1}}$-associated 
cyclin-CDK complexes suggest altered regulation of the cell 
cycle specifically in G$_{\text{1}}$. We also measured 
expression of genes and proteins (\autoref{bmc:figs3}) 
after 6 and 24 hr of treatment with 0.1, 1, and 10 ng/ml 
of TcdA or TcdB in confluent and subconfluent cultures, 
which confirmed a consistent alteration of cell cycle 
genes and proteins across a variety of conditions.


\begin{figure}[h!]
  \centering
  \includegraphics[width=0.75\columnwidth]{bmc/Figure3}
  \caption[The altered gene expression of 
  G$_{\text{1}}$ phase cell cycle regulators 
  at 6h and changes in the distribution of cells 
  within the cell cycle]{
  \textbf{The altered gene expression of 
  G$_{\text{1}}$ phase cell cycle regulators 
  at 6h and changes in the distribution of cells 
  within the cell cycle.}
  \textbf{(A)} A schematic of cell cycle regulation 
  with proteins placed next to the phase of cell cycle 
  with which they are associated (p19 and p21 are the 
  products of the CDKN2D and CDKN1A genes, respectively). 
  Gray, blue, and red indicate genes with unchanged, 
  increased, or decreased expression, respectively, 
  post toxin treatment.
  \textbf{(B)} Cells in a subconfluent culture were 
  treated with the indicated concentrations of toxin 
  for 24h. The DNA content of cells in each condition 
  was quantified by PI fluorescence. The histograms 
  of the area of PI fluorescence are normalized to the 
  total number of cells (denoted as normalized cell count) 
  in the sample such that the area under each histogram 
  is equal to 1. In this way, the proportions of cells in 
  each phase of the cell cycle may be compared for different 
  size samples. The scale of the vertical axis is the same 
  in each histogram.
  \textbf{(C)} The percentage of cells in each phase of 
  the cell cycle was calculated using ModFit LT software. 
  Sub-G$_{\text{0}}$/G$_{\text{1}}$ cells were not included 
  in the calculations.
}
  \label{bmc:fig3}
\end{figure}

\begin{figure}[h!]
  \centering
  \includegraphics[width=\columnwidth]{bmc/FigureS3}
  \caption[Timing of RAC1 glucosylation and expression of 
             cyclin-related genes and proteins in
             subconfluent and confluent cultures.]{
  \textbf{Timing of RAC1 glucosylation and expression of 
             cyclin-related genes and proteins in
             subconfluent and confluent cultures.}
  (A) The expression of 10 genes in both subconfluent
   and confluent toxin-treated cells was measured 
  by qRT-PCR. Fold changes shown are relative 
  to untreated samples.
  (B) Densitometry was performed on immunoblots 
  of lysates from toxin-treated, confluent cells. The 
  intensity of each band was normalized to the intensity 
  of GAPDH, the loading control. The values 
  above each row indicate the amount of protein relative 
  to the amount in untreated cells. The relative 
  intensity for p57 was not calculated because the protein 
  was not detectable in untreated cells. The 
  blots show the presence of p57 after toxin treatment.
  (C) A Rac1 antibody (BD\#610650) that recognizes 
  non-glucosylated Rac1 (protein that has not been 
  glucosylated by either toxin) shows the activity of 
  TcdA and TcdB in HCT-8 cells. }
  \label{bmc:figs3}
\end{figure}

\subsection{Effects of TcdA and TcdB on the Regulation of Cell Cycle}

The functional changes suggested by altered gene and protein 
expression were then investigated by quantifying the proportion 
of cells in each phase of the cell cycle before and after 
toxin treatment. To focus on actively growing cells and 
avoid the effects of contact inhibition, subconfluent 
cultures were used. After 24 hr of 0.1 or 1 ng/ml TcdB 
treatment, the distribution of cells across the cell 
cycle changes significantly, with only a small increase 
in the proportion of cells with less than a 
G$_{\text{0}}$/G$_{\text{1}}$ amount of DNA—cells 
that are presumably dead or dying (\autoref{bmc:fig3}B). 
In agreement with our gene expression analysis, the 
percentage of G$_{\text{0}}$/G$_{\text{1}}$ cells increased 
from 67\% in untreated cells to 91\% and 94\% in cultures 
treated with 10 ng/ml TcdA and 1 ng/ml TcdB, 
respectively (\autoref{bmc:fig3}C). The magnitude of 
increase in the G$_{\text{0}}$/G$_{\text{1}}$ proportion 
is also concentration-dependent. The effect on cell cycle 
by the combination of TcdA and TcdB is comparable to 
those produced by TcdB alone (\autoref{bmc:fig3}C), 
indicating that,with respect to their influence on cell-cycle 
arrest, the toxins are neither synergistic nor antagonistic. 
As with gene and protein expression, TcdB is more potent or 
faster-acting than TcdA. Taken together, these data 
establish that the toxin-induced alterations in genes 
associated with cell cycle correlate with a block at 
the G$_{\text{1}}$–S interface. In other studies, a 
G$_{\text{2}}$/M arrest has been reported in human cell 
lines treated with different concentrations of TcdA 
or TcdB \cite{Kim:2005km, Gerhard:2008wz, Nottrott:2007ep}. 
This G$_{\text{2}}$/M arrest has been linked with a 
deregulation of the cell structure resulting in an 
inability of cells to complete cytokinesis 
\cite{Huelsenbeck:2009di}. We therefore investigated 
the cell cycle effects and morphology of cells treated 
for 24 hr with higher concentrations of TcdA (100 ng/ml) 
and TcdB (10 and 100 ng/ml).

Our analysis of single-cell images from toxin-treated cultures 
reveals two unanticipated observations: (1) a biphasic 
distribution of apoptotic cells with respect to stage of 
cell cycle and (2) two populations of cells at the 
G$_{\text{2}}$/M interface.  Cells with a high-contrast 
bright-field image and a low area of PI fluorescence are 
classified as apoptotic (\autoref{bmc:fig4}A). Typically, 
apoptotic cells are associated with a PI fluorescence level 
less than that of the G$_{\text{0}}$/G$_{\text{1}}$ 
population.  Here, a significant portion of the 
toxin-treated cells between the G$_{\text{0}}$/G$_{\text{1}}$ 
and G$_{\text{2}}$/M cell populations (typically 
associated with/attributed to the S-phase) are 
apoptotic (\autoref{bmc:fig4}B). Thus, the accumulation 
of toxin-treated cells with S-phase levels of 
PI-fluorescence is not the result of progression 
from G$_{\text{1}}$ but rather the apoptosis of 
G$_{\text{2}}$/M cells. Even 24 hr after the addition 
of 100 ng/ml of TcdB, apoptosis does not dominate or 
override effects on cell cycle. At the highest 
concentration tested (100 ng/ml), 68.6\% of TcdB-treated 
cells are still classified as non-apoptotic 
(\autoref{bmc:fig4}B). Of the total number of non-apoptotic 
cells, the proportion in the G$_{\text{2}}$/M phase 
increases as the concentrations of either TcdA or 
TcdB increases, indicating an inhibition of progression 
from G$_{\text{2}}$/M phase, in addition to the 
G$_{\text{1}}$--\,S block discussed above.


\begin{figure}[h!]
  \centering
  \includegraphics[width=0.75\columnwidth]{bmc/Figure4}
  \caption[Distribution of apoptotic versus non-apoptotic 
  cells within the cell cycle and characteristics of 
  G$_{\text{2}}$/M phase, toxin-treated cells]{
  \textbf{Distribution of apoptotic versus non-apoptotic 
  cells within the cell cycle and characteristics of 
  G$_{\text{2}}$/M phase, toxin-treated cells.}
  \textbf{(A)} Cells were classified as either apoptotic 
  or non-apoptotic based on the contrast of their 
  brightfield image and the area of PI fluorescence. 
  Representative images of a cell in each class are 
  shown (100 ng/ml TcdB). 
  \textbf{(B)} Histograms of the area of PI fluorescence 
  of each cell show the location of apoptotic and non-apoptotic 
  cells within the cell cycle. The percentage of 
  G$_{\text{2}}$/M cells represents the proportion of non-apoptotic 
  cells with a G$_{\text{2}}$/M level of DNA.
  \textbf{(C)} Non-apoptotic G$_{\text{2}}$/M phase cells 
  were analyzed to determine the number of distinct nuclei. 
  For this analysis only, cells with an area of PI 
  fluorescence 1.85 times greater than the PI 
  fluorescence area at the G$_{\text{0}}$/G$_{\text{1}}$ 
  peak were considered to be G$_{\text{2}}$/M cells. 
  The major and minor axis intensity values are the 
  length of the axis weighted by the intensity of the 
  image along the axis.
}
  \label{bmc:fig4}
\end{figure}


In order to understand the differences between toxin-treated 
and control cells in G$_{\text{2}}$/M, we determined  several 
cellular characteristics (circumference, area, and others) of 
individual cells using an imaging flow cytometer. The feature 
that best distinguishes toxin-treated from untreated cells is 
the intensity-weighted aspect ratio of the PI fluorescence 
image. When an ellipse is fit around the image, an aspect ratio 
near one indicates a circular nucleus and a higher aspect ratio 
indicates an elliptical nucleus or multiple nuclei 
(\autoref{bmc:fig4}A). Upon visual inspection, a high aspect 
ratio is due typically to binucleation. The higher proportion 
of binucleated cells in toxin-treated cells (\autoref{bmc:fig4}C) 
indicates that the G$_{\text{2}}$/M block is attributable to 
a failure to complete cytokinesis \cite{Huelsenbeck:2009di}. 
Therefore, in addition to demonstration of a G$_{\text{1}}$--\,S 
block, our results show an inhibition of progression at 
the G$_{\text{2}}$\,--\,M transition, which is congruent 
with previous findings 
\cite{Kim:2005km, Gerhard:2008wz, Nottrott:2007ep, Fiorentini:1998uh} 
in other cell types treated with different toxin 
concentrations. Importantly, these G$_{\text{2}}$/M effects 
were observed at the same concentration of toxin used for 
microarray analysis (100 ng/ml). Again, TcdA elicited a similar 
response to TcdB at the same concentration, yet to a lesser 
extent, thus showing consistency from gene and protein 
expression to cell function.

\section{Discussion}
Understanding the differences between these two toxins is 
particularly relevant in determining their roles in \textit{C. difficile} 
infection. Toxin A appears to be the dominant virulence factor 
in animal studies, yet Toxin B has higher enzymatic activity 
in vitro and is more potent when injected into Don cells and 
for human cells studied in vitro \cite{ChavesOlarte:1997cs,Riegler:1995jz}. 
In a hamster model, Kuehne et al. found that strains of 
\textit{C. difficile} producing only TcdA or TcdB are comparable 
in their virulence, while Lyras et al used a TcdA mutant to 
show that TcdB was the key virulence factor 
\cite{Kuehne:2010hv,Lyras:2009jx}. In this study, we used a 
systems approach to understand the effects of TcdA and TcdB 
on a human colonic epithelial cell line. We observed that the 
responses to these two toxins are strikingly similar, with 
the response to TcdB occurring more rapidly. Investigation 
of one of the biological consequences of these changes in 
gene expression revealed toxin effects at both the 
G$_{\text{1}}$--\,S and the G$_{\text{2}}$\,--\,M transitions.

In order to explore the interactions between \textit{C. difficile} 
and intestinal epithelial cells, Janvilisri et al. examined 
the transcriptional responses of Caco-2 cells and 
\textit{C. difficile} organisms during an in vitro 
infection \cite{Janvilisri:2010wf}. Because expression 
was measured at no more than 2 hr post-infection, most 
of the changes in gene expression were slight, yet they 
identified functions such as cell metabolism and 
transport associated with affected genes. We focused on 
cells treated with TcdA or TcdB at a concentration and 
time course in which the cell morphology is strongly 
affected. The effects of TcdA and TcdB on gene expression 
in host cells have been interrogated in other studies 
focusing on individual pathways, but until now, an analysis 
of the comprehensive global transcriptional response 
induced by either TcdA or TcdB alone had not been performed.

Our systems approach identified a disruption of the cell 
cycle not readily apparent from a ranked list of genes. This 
approach overcame difficulties in deciphering the particular 
relevance of genes known to be induced by several stimuli 
or genes whose expression leads to many possible cellular 
phenotypes. JUN is overall the most differentially 
expressed gene in our data, and, considering TcdA or 
TcdB as a cellular stress, its dramatic increase in 
expression is consistent with it being characterized 
as a stress-response gene. However, increased JUN 
expression has also been associated with the promotion 
of G$_{\text{1}}$ progression, protection from apoptosis 
after ultraviolet radiation, and even induction of 
apoptosis \cite{Ameyar:2003cc}. Clearly, multiple 
events may lead to the same changes in expression of 
an individual gene. The identification of functions 
associated with many of the differentially expressed 
genes thus provides better evidence of actual 
biological functions important to the toxin response. 

These results have clarified the effects of TcdA and TcdB at 
each stage of the cell cycle. In studying Rho signaling, 
Welsh et al. showed that combined Rho, Rac, and Cdc42 
inhibition by TcdA (200 ng/ml) in fibroblasts led to 
decreased cyclin D1 expression and an inability of 
serum-starved cells, stimulated with fetal calf serum 
and treated with toxin, to progress past the 
G$_{\text{1}}$ phase \cite{Welsh:2001id}. Importantly, 
we show that a strong G$_{\text{1}}$ arrest occurs in 
unsynchronized, proliferating epithelial cells. Only when 
treated with higher concentrations (100 ng/ml TcdA, 
10 ng/ml TcdB) of toxin did we begin to observe the inhibition 
of cell division at the G$_{\text{2}}$/M phase in a 
significant proportion of cells. With regard to cell death, 
others have shown an increased susceptibility of S-phase cells 
to toxin treatment \cite{Huelsenbeck:2007df}. We did 
observe an increase in the proportion of apoptotic S 
or G$_{\text{2}}$/M phase cells. At low concentrations 
(10 ng/ml TcdA, 1 ng/ml TcdB), the decrease in the 
proportion of S-phase cells, however, could not be 
entirely accounted for by death of cells at a 
particular point in the growth cycle. Rather, many 
non-apoptotic cells remained in the 
G$_{\text{0}}$/G$_{\text{1}}$ phase.

\section{Conclusion}
Our results have several implications in reference to the 
role of these toxins in pathogenicity. In a host, the gut 
epithelial cells normally turn over every 2-3 days 
\cite{Kim:2010ja}. Disruption of this cellular renewal 
process, and therefore cell cycle, impairs the 
maintenance of the epithelial barrier. The ability of 
both TcdA and TcdB to arrest growth at the 
G$_{\text{0}}$/G$_{\text{1}}$ phase and the 
G$_{\text{2}}$\,--\,M transition, by likely different 
mechanisms (G$_{\text{1}}$ arrest occurs even at low 
toxin concentrations and is associated with altered 
protein signaling; G$_{\text{2}}$ arrest is likely 
associated with disorganization of the cytoskeleton), 
places each toxin in the category of cyclomodulins. 
As has been previously shown however, control of 
cell proliferation is certainly not their only or 
necessarily primary effect (e.g., inflammation, 
disruption of tight junctions). The high similarity 
in the gene expression induced by these two toxins 
indicates that, qualitatively, their effects and 
the overall cellular responses are comparable. The 
rate of internalization and/or the rapidity of 
inactivation of Rho-family proteins in different 
hosts may partially account for the different 
rates in the onset of gene expression. Though we did 
not observe synergy or antagonism between the two 
toxins, it is possible that each could differentially 
bind various cell types and therefore act synergistically 
within a host. Clearly, the response to each toxin 
is a complex process involving the activation and 
inhibition of several pathways in different cell 
types. The integration and use of the data we present 
here have and will continue to aid the organization 
of these multiple effects into a central framework 
for interrogating toxin activity.

\section{Acknowledgements}
This entire chapter is taken from 
\fullcite{DAuria:2012bd}.
I would like to thank all the co-authors. I am
also  grateful to David Lyerly, Ph.D., and 
TechLab, Inc., Blacksburg, VA, for the generous gift 
of Toxins A and B.









