\chapter[Functional transcriptomics of toxin response]{Functional transcriptomics of the host response to \textit{C. difficile} toxins}\label{chapter:introbmc}

\textit{C. difficile} causes 300,000+ infections and $\sim$20,000 deaths in the US every year,
indirectly costing the healthcare system over \${}8 billion \cite{Lucado:2012wl}.
\textit{C. difficile} strains are only pathogenic if
they release toxin A or toxin B (TcdA and TcdB).
The basic structure and enzymatic activities of the toxins
are understood but the complex host response to the toxins
is not \cite{Pruitt:2010cy,Pruitt:2012kx,Genisyuerek:2011dr,
Just:1995kz,Just:1995ei,Sun:2010kt}. 
Since the severity of illness is determined
by the the host and not the extent of infection
or the number of bacteria in the host, it is critical
to understand detrimental host responses \cite{ElFeghaly:2013gq}.

Taking a systems biology approach, I show how transcriptomics
can be used to infer previously unidentfied host cell responses.
The field of functional genomics (or functional transcriptomics) which overlaps
with systems biology and aims to determine functions from genetic data
and characterize interactions between genes and proteins. In this chapter's last section, I
discuss the concept of ``enrichment'' to identify
groups of genes or pathways that are altered 
and are likely to represent larger functional changes within the cell.

\section{\textit{C. difficile}: a dangerous pathogen}

\subsection{Historical significance}

In 1935, Hall and O'Toole isolated a novel gram-positive bacterium from infants which
they named \textit{Bacillus difficilis} because it was a rod (\textit{Bacillus})
and was an anaerobe that was difficult to grow (\textit{difficilis}). 
They also identified a \textit{B. difficilis} toxin that killed mice.
In 1943, the first rodent model of infection was made unintentionally
when penicillin induced inflammation in the cecum \cite{Hamre:1943te}.
Remarkably, these findings from 40 years ago summarize
much of what we now know: \textit{C. difficile} infection occurs
when the flora is disrupted by antibiotics and the pathogenic
effects are due to toxins. Even more interesting, several
studies over 50 years ago suggested that the most effective treatment
could be the restoration of the bacterial flora, which
in the past few years has proved to be the most effective 
treatment (reviewed in \cite{Bartlett:2008jx}).

\begin{figure}[h!]
  \centering
  \includegraphics[width=\textwidth]{introbmc/cartoon.png}
  \caption{\textbf{An oversimplified view of \textit{C. difficile} infection}}
  \label{introbmc:cartoon}
\end{figure}

After the advent of antibiotics, ``pseudomembranous colitis'' (PMC), a condition
manifesting as inflammation and diarrhea, became associated
with antibiotic treatment \cite{Tedesco:1974jo,Bartlett:2008jx}. 
The etiology of PMC remained unknown
for several years until a rapid succession of experiments between 1977 and 1981
found \textit{C. difficile} and its two large protein toxins (TcdA and TcdB)
 to be the primary cause of PMC 
\cite{Lusk:1978ub,Chang:1978um,Browne:1977ui,Fekety:1979ul,
Ebright:1981uj,Taylor:1981uda,Bartlett:1978tj,Bartlett:1977wra,
Lusk:1977wg,MRKeighley:1978ku}. 
With increased antibiotic usage and improved surveillance,
the incidience of \textit{C. difficile} infections has increased 
nearly every year since \cite{Lucado:2012wl}.
From 1981 to 1995, 
studies characterized the broad physiological toxin effects,
many of which are discussed in the following chapters.


\subsection{Toxin molecular biology}
TcdA and TcdB are among the most lethal natural substances known.
Their lethal dose is less than that of strychnine and cyanide.
However, as proteins, they are more similar to other toxins such as 
ricin, anthrax toxin,
tetanus toxin, and botulinum toxin (Botox).

The enzymatic activity of TcdA and TcdB which is responsible
for their cytopathic effects was discoverd 
in 1995 by Just et al. \cite{Just:1995ei,Just:1995kz}.
Both TcdA and TcdB, with 63\% amino acid
homology \cite{vonEichelStreiber:1992wu}, have N-terminal
domains that glucosylate Rho family proteins, disabling
them from entering their GTP-bound, active state \cite{Just:1995ei,Just:1995kz}.
The C-terminal of both toxins consists of many ``clostridial
repetitive oligopeptides'' (CROPs) that are present in 
other clostridial and streptococcal species 
\cite{vonEichelStreiber:1990ti,vonEichelStreiber:1992vb}
that are important for cell binding and 
entry \cite{Ho:2005vm,Greco:2006wf,Frisch:2003jo}.
However, it is unclear if the CROPs are entirely necessary
for cell entry or if the CROPs can alone cause
cytopathic effects \cite{Yeh:2008hda,Zemljic:2010ub}. Both toxins
enter cells by endocytosis and rearrange structurally
in the acidic endosome \cite{Papatheodorou:2010io,QaDan:2000fj}.
After translocating N-terminal domains to the cytosol through
self-formed pore in the endosome,
a cysteine protease domain cleaves off the glucosyltransferase
domain into the cytosol \cite{Egerer:2007fy,Genisyuerek:2011dr,Pfeifer:2003bx}.
Although much remains to be understood about the 
toxins' functions on the molecular level (e.g., no toxin receptors are known and
a large middle portion of the toxins has no known function), there
are even more unknowns about the disease pathogenesis disease that
is most critical to clinical outcome.

\begin{figure}[h!]
  \centering
  \includegraphics[width=0.85\textwidth]{introbmc/steps.png}
  \caption{\textbf{Sequence, structure, and functioin of TcdA and TcdB}}
  \label{introbmc:steps}
\end{figure}

\subsection{Physiological toxin responses}
The external manifestations of \textit{C. difficile} infection 
are diarrhea, abdominal pain, and sometimes fever.
Internally, pseudomembranous colitis is characterized by
an inflamed colon covered with yellow, volcano-shaped 
pseudomembranes made of cellular debris, exudate,
and inflammatory cells \cite{Tedesco:1974jo}. 
Histology reveals an abundance of neutrophils and
loss of epithelial barrier integrity \cite{Lyerly:1988dk,Kelly:1994cu}.
Accordingly, many of the hypotheses for \textit{C. difficile}
and toxin studies have been centered around known inflammatory
markers. However, how these markers interact together
is unknown. It is also unknown if any other cell functions
(e.g., regulation of metabolism) contribute to the pathogenesis
or the healing process.


\section{Functional transcriptomics: enrichment}
This dissertation presents a systems biology approach
to the host response to \textit{C. difficile} toxins,
using genome-wide expression to reveal altered cellular functions.
Appropriate data processing techniques were chosen to calculate
expression values and differentially expressed genes (\autoref{chapter:introjtm}).

Many network based algorithms can be used to reconstruct regulatory networks
and gene interactions. However, many of these methodologies are unproven
or still in the formative stages. The approach in this dissertation
is to present the differentially expressed genes using exploratory analyses
that enable novel hypotheses.

\subsection{Gene set enrichment analysis}
One of the simplest yet most helpful questions from transcriptomics data is `what
types of genes changed'. For example, is the number of DEG chemokines 
greater than expected? If so, then one could say that the data
set is \textit{enriched} with highly expressed chemokines. Alternatively,
it is common to say that chemokine genes or chemokine functions
are \textit{enriched}.

These questions require a test of proportions. For example,
is there a significantly greater proportion chemokines in the list
of DEGs compared to the list of non-DEGs?
The Fisher exact test calculates the significance of such proportions, and it
is ubiquitous in gene expression studies. The enrichment of hundreds
of predefined sets of genes from public biological databases can be used.

The Fisher test as well as the $\chi^2$ and other tests of proportions
require a threshold to define DEGs, and the results of the test are
very sensitive to the threshold chosen. A handful of algorithms have
tried to solve the threshold problem by scanning many thresholds
with little success. These tests also assume that genes within the
gene set are independent, which usually incorrect.

Like with DEG significance tests, the choices are numerous.
In trying to find the appropriate test, I evaluated over 130 articles
introducing new tests and software tools to perform enrichment tests, and there
are probably one hundred more.
Reviews of these methods can usually discuss a portion of 
the possible tests \cite{Ackermann:2009bw,Huang:2009bea,Abatangelo:2009cc,
Berg:2009tu,Dinu:2008tv}.
Also similar to DEG tests, there are many different categories (e.g., 
parametric tests, permutation tests, machine learning algorithms)
that make different statistical assumptions.

Enrichment methods are even more complicated because the hypotheses
are ill defined. For instance, are there more genes in a gene set
than all genes outside the gene set? Is the fold change of gene
in a gene set, on average, different than 1? The first question
is more restrictive than the second, but neither are wrong questions
to ask.

In later chapters, I describe some more details of different tests
and give reasoning to the enrichment tests chosen for
the transcriptomics data. The tests were chosen based on exploration
of many methods to find the most consistently highly ranked
gene sets. 












