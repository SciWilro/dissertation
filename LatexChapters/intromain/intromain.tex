
\chapter{Introduction}\label{chapter:intro1}

\section{Abstract}
Toxins A and B, two highly potent protein toxins, are the essential
virulence factors of \textit{C. difficile}, a bacterium which infects 300,000+
people in the US every year \cite{Lucado:2012wl}. 
Since the extent to which a persons body overreacts to the toxins
determines disease severity, controlling the host response is critical for improving treatments.
Given the manifestations of
diarrhea and colitis, nearly all research to date has predictably
focused on known inflammatory pathways and related cellular responses.
However, 40 years after the toxins' discovery, the fatality rate has continued to rise.
A different approach is needed. In this dissertation, I present a holistic approach,
profiling the physiological and transcriptional changes of host cells
to toxins in vitro and in vivo. I determine the most
appropriate statistical methods for identifying genes and pathways
affected by toxins, leading to discovery of an unrecognized cell-cycle
disruption of epithelial cells treated with toxins.
I then extend the approach to investigate epithelial-layer cells
in mice with toxin injected into their intestines, identifying
pathways altered only in vivo. These pathways offer new therapeutic targets,
as is shown by antibody neutralization experiments showing that the levels
of two cytokines are predictive of survival.
I again extend the systems approach to analyze toxin sensitivity
and dynamic, morphological changes of cell types in addition to epithelial cells.
Sensitivities of macrophages, epithelial, and endothelial cells
indicate that epithelial cells may not be the critical cell type
for initiating disease and show that the most well-studied toxin molecular
activity (glucosylation) is not required for all toxin-induced cellular responses.
In addition to these novel findings, this work presents new ways of thinking
about host responses to \textit{C. difficile} toxins that can be investigated
in the future with mechanistic models and reductionist experiments.


\section{A preview of this dissertation}

In \autoref{chapter:introjtm}, I explain the experimental and computational
methods at the core of the findings in all subsequent chapters.
A short primer explains the biological concepts of mRNA and transcriptomics
studies. The advantages and pitfalls of the many possible data processing
techniques that together form the analytical workflow are described in context
of the data presented in \autoref{chapter:bmc} and \autoref{chapter:ini}. 
The importance of reproducibility, a special concern of mine, is also 
discussed. \autoref{chapter:jtm} presents my transcriptomics analysis of
samples from a clinical trial for a combination melanoma therapy.
These methods set the stage for the primary focus of this dissertation,
the host response to \textit{C. difficile} toxins.

\autoref{chapter:introbmc} gives the scientific background
and clinical significance of \textit{C. difficile} infections, 
the known roles of \textit{C. difficile} toxins in infection,
and the importance of the host response to pathogenesis.
The basic concepts, advantages, and pitfalls of functional genomics 
and systems biology methods such as
gene set enrichment that come after the methods in \autoref{chapter:introjtm}
are briefly described. These systems biology methods are then used in \autoref{chapter:bmc}
to analyze the transcriptional responses of an epithelial cell line, revealing
disruptions in cell cycle that block cell growth without inducing
complete cell death.

\autoref{chapter:ini} presents physiological 
and transcriptional responses to \textit{C. difficile} toxins
in a mouse intoxication model. Changes in the expression of pathways and gene sets that
are characteristic of the response are described and compared to
the in vitro responses in \autoref{chapter:bmc}.
Follow-up experiments neutralizing two cytokines within
these gene sets proved that the systemic levels of the two cytokines
correlated with disease severity and could be used to
predict survival.

Epithelial cells are the focus of the in vitro and in vivo
transcriptional studies, yet the data indicate that other cell types
are also important. Therefore, in \autoref{chapter:imp}
the dynamic, morphological changes responses of macrophage, endothelial cells, 
and epithelial
cells are measured precisely with electrical impedance.
With this experimental framework, I also investigate the necessity
of the toxins' glucosyltransferase activity to these responses.
Software that I developed for managing and visualizing time course data from
multi-well data is presented in \autoref{chapter:wellz}.

The last chapter, \autoref{chapter:trends}, reviews
metabolic network analyses. Although not directly related to
the methods in the previous chapters, it is presented as an
example of additional analyses that could be performed
to gain more mechanistic insight.



