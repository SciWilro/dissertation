\chapter[Multi-well time course analyses]{Software for 
analyzing time course data from multi-well experiments}\label{chapter:wellz}

The data in the previous chapter was generated from thousands of wells
in multi-well plates. For each well, thousands of data points were recorded.
To manage these data, I wrote software to organize and visualize the data.
This software is written in the free, open source R programming language
so that it is easily accessible and extensible. Here, I briefly describe
its key features.


\section{Concise annotation of multiple wells}

The meta data for all the wells can be input as a spreadsheet or
table as in \autoref{wellz:tab1}. The format and notation was chosen
to minimize the amount of effort. All blank cells use the value
of the next non-blank cell above it. The well identifier also
allows shorthand (e.g., A-B1-2 refers to wells A1, A2, B1, and B2).
Each row of the table represents an experimental action that occurred
directly after the $i^{\text{th}}$ data point. The ``ID'' labels
that action so that it can be referenced in future data transformations and alignments.
``rmVol'' and ``adVol''
are the volume removed, respectively, during an action.
``name'' and ``concentration'' indicate the compounds in the solution
being added to each well. Comma-separated values are matched with the
``wells'' column. In the ``type'' column, ''final'' indicates that
the indicated concentration is the concentration after the action is
performed; ``start'' indicates the concentration referse to the solution
before it is added to the well.


\begin{table*}[h!]
%\begin{adjustwidth}{-1cm}{}
\resizebox{\textwidth}{!}{%
\begin{tabular}{ l l l c c c l l l l }
file & wells & ID & i & rmVol & adVol & name & concentration (ng/ml) & type & solvent \\ \toprule
T84-TcdB.txt & A-H5-6 & start & 1 & 0 & 100 & - & - & - & media \\
             & A-H5-6 & cellSeed & 2 &  &  & T84 & 30000 & total & \\
             & B-F6, C5, E-F5 & toxinAdd & 1429 &  &  22.2 &
             TcdB & 1000, 300, 100, 30, 10, 3, 1, 0.1 & final & \\
             & D5 & &  &  &  & - & - & - & \\
CecalCells.txt & A-H3-6 & start & 1 &   & 100 & - & - & - & \\
& A-H3-6 & cellSeed & 2 &  &  & IMCE & 30000 & total & \\
             & A-B3, D-H3 & toxinAdd & 449 &  &  22.2 &
             TcdA & 1000, 100, 10, 1, 0.1, 0.01, 0.001 & final & \\
& C3 & &  &  &  & - & - & - & \\
             & A-B3, D-H3 & & 449 &  &  &
             TcdB & 100, 10, 1, 0.1, 0.01, 0.001, 0.0001 & final & \\
\end{tabular}
}
%\end{adjustwidth}
\caption{Example metadata file}
\label{wellz:tab1}
\end{table*}

This file is parsed and stored in an object in R.

\begin{knitrout}
\definecolor{shadecolor}{rgb}{0.969, 0.969, 0.969}\color{fgcolor}\begin{kframe}
\begin{alltt}
\hlcom{# Parse and load the data}
\hlstd{wells} \hlkwb{=} \hlkwd{parse.RTCAanalyze}\hlstd{(}\hlkwc{metadata} \hlstd{=} \\ \hlstr{"./MasterSheet2.csv"}\hlstd{,} \hlkwc{data.dir} \hlstd{=} \hlstr{"./Data2"}\hlstd{)}
\end{alltt}
\begin{alltt}
\hlcom{# Show a summary of the first 8 wells}
\hlkwd{print}\hlstd{(wells[}\hlnum{1}\hlopt{:}\hlnum{8}\hlstd{])}
\end{alltt}
\begin{verbatim}
##      file points hours well   totals    ID.soln
## 1 CHO.txt  3,461   184  A01 CHO-9000  TcdA-1000
## 2       -      -     -  B01        -           
## 3       -      -     -  C01        -   TcdA-100
## 4       -      -     -  D01        -     TcdA-1
## 5       -      -     -  E01        -   TcdA-0.1
## 6       -      -     -  F01        -  TcdA-0.01
## 7       -      -     -  G01        - TcdA-0.001
## 8       -      -     -  H01        - TcdA-1e-04
\end{verbatim}
\begin{alltt}
\hlcom{# Show a summary for the first well}
\hlkwd{print}\hlstd{(wells[[}\hlnum{1}\hlstd{]])}
\end{alltt}
\begin{verbatim}
## Well: CHO.txt, A01
## 3,461 data points over 184 hours
## 
##             ID  t1  t2 rmVol notes
## 1        start   0   1     0      
## 2     cellSeed   2   3     0      
## 3        move1 285 286     0      
## 4 mediaReplace 307 308   100      
## 5     toxinAdd 611 612     0      
##       solution       status
## 1 100, lis.... 100, lis....
## 2 100, lis.... 100, lis....
## 3 NA, list.... 100, lis....
## 4 100, lis.... 100, lis....
## 5 100, lis.... 100, lis....
\end{verbatim}
\end{kframe}
\end{knitrout}


\section{Selecting and modifying wells}
Each well contains several actions. Each action contains
a \texttt{solution} object.

\singlespacing
\begin{knitrout}
\definecolor{shadecolor}{rgb}{0.969, 0.969, 0.969}\color{fgcolor}\begin{kframe}
\begin{alltt}
\hlcom{# Select wells based off metadata}
\hlstd{subset} \hlkwb{=} \hlkwd{retrieveWells}\hlstd{(wells,}
    \hlkwc{file} \hlstd{=} \hlstr{"IMCE.txt"}\hlstd{,} \hlkwc{compounds} \hlstd{=} \hlstr{"TcdB"}\hlstd{,}
    \hlkwc{max.concentrations} \hlstd{=} \hlnum{50}\hlstd{,}
    \hlkwc{controls} \hlstd{=} \hlnum{TRUE}\hlstd{)}
\hlkwd{print}\hlstd{(subset)}
\end{alltt}
\begin{verbatim}
##        file points hours well     totals    ID.soln
## 1  IMCE.txt  5,376 119.9  C03 IMCE-30000           
## 2         -      -     -  B04          -    TcdB-10
## 3         -      -     -  C04          -     TcdB-1
## 4         -      -     -  D04          -   TcdB-0.1
## 5         -      -     -  E04          -  TcdB-0.01
## 6         -      -     -  F04          - TcdB-0.001
## 7         -      -     -  G04          - TcdB-1e-04
## 8         -      -     -  H04          - TcdB-1e-05
## 9         -      -     -  C05          -           
## 10        -      -     -  A06          -    TcdB-10
## 11        -      -     -  B06          -     TcdB-1
## 12        -      -     -  C06          -   TcdB-0.1
## 13        -      -     -  D06          -  TcdB-0.01
## 14        -      -     -  E06          - TcdB-0.001
## 15        -      -     -  F06          - TcdB-1e-04
## 16        -      -     -  G06          - TcdB-1e-05
## 17        -      -     -  H06          -
\end{verbatim}
\begin{alltt}
\hlcom{# Quick look at the actions in the first well}
\hlkwd{print}\hlstd{(subset[[}\hlnum{1}\hlstd{]])}
\end{alltt}
\begin{verbatim}
## Well: IMCE.txt, C03
## 5,376 data points over 119.9 hours
## 
##         ID  t1  t2 rmVol notes
## 1    start   0   1     0      
## 2 cellSeed   2   3     0      
## 3 toxinAdd 449 450     0      
##       solution       status
## 1 100, lis.... 100, lis....
## 2 100, lis.... 100, lis....
## 3 100, lis.... 100, lis....
\end{verbatim}
\begin{alltt}
\hlcom{# The solution objects for each action}
\hlstd{solutions} \hlkwb{=} \hlstd{subset[[}\hlnum{1}\hlstd{]]}\hlopt{$}\hlstd{timeline}\hlopt{$}\hlstd{solution}
\hlkwd{print}\hlstd{(solutions)}
\end{alltt}
\begin{verbatim}
## [[1]]
## 100 uL media
## 
## [[2]]
## 100 uL 
## Solvent: media
## IMCE 30000 total.
## 
## [[3]]
## 22.2 uL media
\end{verbatim}
\begin{alltt}
\hlcom{# Solutions have compounds and solvents}
\hlstd{solutions[[}\hlnum{3}\hlstd{]]}\hlopt{$}\hlstd{compounds}
\end{alltt}
\begin{verbatim}
## [1] name conc type
## <0 rows> (or 0-length row.names)
\end{verbatim}
\begin{alltt}
\hlcom{# Adding and subtracting solutions}
\hlstd{solutions[[}\hlnum{2}\hlstd{]]} \hlopt{+} \hlstd{solutions[[}\hlnum{3}\hlstd{]]}
\end{alltt}
\begin{verbatim}
## 122.2 uL 
## Solvent: media
## IMCE 30000 total.
\end{verbatim}
\begin{alltt}
\hlstd{solutions[[}\hlnum{2}\hlstd{]]} \hlopt{+} \hlstd{solutions[[}\hlnum{3}\hlstd{]]} \hlopt{-} \hlstd{solutions[[}\hlnum{1}\hlstd{]]}
\end{alltt}
\begin{verbatim}
## 22.2 uL 
## Solvent: media
## IMCE 30000 total.
\end{verbatim}
\end{kframe}
\end{knitrout}

\section{Visualization}

Lists of well objects can be visualized and plot features can be linked to metadata.

\begin{knitrout}
\definecolor{shadecolor}{rgb}{0.969, 0.969, 0.969}\color{fgcolor}\begin{kframe}
\begin{alltt}
\hlkwd{plot}\hlstd{(subset,} \hlkwc{color} \hlstd{=} \hlstr{"by.concentrations"}\hlstd{,}
    \hlkwc{linetype} \hlstd{=} \hlstr{"by.compounds"}\hlstd{,}
    \hlkwc{replicates} \hlstd{=} \hlnum{FALSE}\hlstd{)}
\end{alltt}
\end{kframe}
\includegraphics[width=0.8\maxwidth]{wellz/suppRef3} 
\end{knitrout}

Diagnostic plots can also be made.

\begin{knitrout}
\definecolor{shadecolor}{rgb}{0.969, 0.969, 0.969}\color{fgcolor}\begin{kframe}
\begin{alltt}
\hlkwd{plot}\hlstd{(subset,} \hlkwc{color} \hlstd{=} \hlstr{"by.concentrations"}\hlstd{,}
    \hlkwc{replicates} \hlstd{=} \hlnum{FALSE}\hlstd{,} \hlkwc{diagnostic} \hlstd{=} \hlnum{1}\hlstd{)}
\end{alltt}
\end{kframe}
\includegraphics[width=1.1\maxwidth]{wellz/suppRef3b} 

\end{knitrout}
\section{Independent wells}
The software was written to be modular so that it can be easily extended.
Each well is an independent object. Functions are written for well objects
so they can easily be applied to a list of unrelated wells. Unless the
wells in the data in a well is modified, the data is passed by reference
(a copy of the data that would take more memory does \textit{not} need to be created).

\begin{knitrout}
\definecolor{shadecolor}{rgb}{0.969, 0.969, 0.969}\color{fgcolor}\begin{kframe}
\begin{alltt}
\hlcom{# Wells from any experiment can be combined}
\hlstd{well.list.1} \hlkwb{=} \hlstd{subset[}\hlnum{2}\hlstd{]}
\hlstd{well.list.2} \hlkwb{=} \hlstd{subset[}\hlnum{10}\hlopt{:}\hlnum{13}\hlstd{]}
\hlstd{new.list} \hlkwb{=} \hlkwd{c}\hlstd{(well.list.1, well.list.2)}
\hlkwd{print}\hlstd{(new.list)}
\end{alltt}
\begin{verbatim}
##       file points hours well     totals   ID.soln
## 1 IMCE.txt  5,376 119.9  B04 IMCE-30000   TcdB-10
## 2        -      -     -  A06          -         -
## 3        -      -     -  B06          -    TcdB-1
## 4        -      -     -  C06          -  TcdB-0.1
## 5        -      -     -  D06          - TcdB-0.01
\end{verbatim}
\end{kframe}
\end{knitrout}

\section{Data transformations}

Replicates can be averaged, either directly or in the plot function.

\begin{knitrout}
\definecolor{shadecolor}{rgb}{0.969, 0.969, 0.969}\color{fgcolor}\begin{kframe}
\begin{alltt}
\hlstd{subset.averaged} \hlkwb{=} \hlkwd{averageReplicates}\hlstd{(subset)}
\hlkwd{print}\hlstd{(subset.averaged)}
\end{alltt}
\begin{verbatim}
##       file points hours        well     totals    ID.soln
## 1 IMCE.txt  5,376 119.9 C03+C05+H06 IMCE-30000           
## 2        -      -     -     B04+A06          -    TcdB-10
## 3        -      -     -     C04+B06          -     TcdB-1
## 4        -      -     -     D04+C06          -   TcdB-0.1
## 5        -      -     -     E04+D06          -  TcdB-0.01
## 6        -      -     -     F04+E06          - TcdB-0.001
## 7        -      -     -     G04+F06          - TcdB-1e-04
## 8        -      -     -     H04+G06          - TcdB-1e-05
\end{verbatim}
\begin{alltt}
\hlstd{subset.transformed} \hlkwb{=} \hlkwd{transform}\hlstd{(subset,}
    \hlkwd{c}\hlstd{(}\hlstr{"tcenter"}\hlstd{,} \hlstr{"normalize"}\hlstd{),}
    \hlkwc{ID} \hlstd{=} \hlstr{"toxinAdd"}\hlstd{)}
\hlkwd{plot}\hlstd{(subset.transformed,} \hlkwc{color} \hlstd{=} \hlstr{"by.concentrations"}\hlstd{,}
    \hlkwc{xlim} \hlstd{=} \hlkwd{c}\hlstd{(}\hlopt{-}\hlnum{2}\hlstd{,} \hlnum{10}\hlstd{))}
\end{alltt}
\end{kframe}
\includegraphics[width=\maxwidth]{wellz/suppRef5b} 
\end{knitrout}

Because experimental protocols cannot be carried
simultaneously on all wells, 
wells can be aligned according to the actions defined in the metadata.
Data can then be normalized or transformed using times defined
in the metadata.

\begin{knitrout}
\definecolor{shadecolor}{rgb}{0.969, 0.969, 0.969}\color{fgcolor}\begin{kframe}
\begin{alltt}
\hlstd{subset2} \hlkwb{=} \hlkwd{retrieveWells}\hlstd{(wells,}
    \hlkwc{compounds} \hlstd{=} \hlstr{"HUVEC"}\hlstd{)}
\hlstd{new.list} \hlkwb{=} \hlkwd{c}\hlstd{(subset[}\hlnum{1}\hlopt{:}\hlnum{2}\hlstd{], subset2[}\hlnum{3}\hlopt{:}\hlnum{4}\hlstd{])}
\hlkwd{print}\hlstd{(new.list)}
\end{alltt}
\begin{verbatim}
##          file points hours well
## 1    IMCE.txt  5,376 119.9  C03
## 2           -      -     -  B04
## 3 HUVEC-a.txt  4,979 250.5  C03
## 4           -      -     -  D03
##       totals  ID.soln
## 1 IMCE-30000         
## 2          -  TcdB-10
## 3 HUVEC-5000 TcdA-100
## 4          -
\end{verbatim}
\begin{alltt}
\hlcom{# plot1 - raw data}
\hlstd{customplot} \hlkwb{=} \hlkwa{function}\hlstd{(}\hlkwc{...}\hlstd{)} \hlkwd{plot}\hlstd{(...,}
    \hlkwc{color} \hlstd{=} \hlstr{"by.concentrations"}\hlstd{,}
    \hlkwc{linetype} \hlstd{=} \hlstr{"by.total.compounds"}\hlstd{)}
\hlstd{plot1} \hlkwb{=} \hlkwd{customplot}\hlstd{(new.list)}

\hlcom{# plot2 - aligned data}
\hlstd{centered.wells} \hlkwb{=} \hlkwd{transform}\hlstd{(new.list,}
    \hlstr{"tcenter"}\hlstd{,} \hlkwc{ID} \hlstd{=} \hlstr{"toxinAdd"}\hlstd{)}
\hlstd{plot2} \hlkwb{=} \hlkwd{customplot}\hlstd{(centered.wells)}

\hlcom{# plot3 - normalized data}
\hlstd{norm.wells} \hlkwb{=} \hlkwd{transform}\hlstd{(new.list,}
    \hlkwd{c}\hlstd{(}\hlstr{"tcenter"}\hlstd{,} \hlstr{"normalize"}\hlstd{),}
    \hlkwc{ID} \hlstd{=} \hlstr{"toxinAdd"}\hlstd{)}
\hlstd{plot3} \hlkwb{=} \hlkwd{customplot}\hlstd{(norm.wells)}

\hlcom{# plot4 - take out just a}
\hlcom{# slice of the time course}
\hlstd{sliced.wells} \hlkwb{=} \hlkwd{transform}\hlstd{(new.list,}
    \hlkwd{c}\hlstd{(}\hlstr{"tcenter"}\hlstd{,} \hlstr{"normalize"}\hlstd{,}
        \hlstr{"slice"}\hlstd{),} \hlkwc{ID} \hlstd{=} \hlstr{"toxinAdd"}\hlstd{,}
    \hlkwc{xlim} \hlstd{=} \hlkwd{c}\hlstd{(}\hlopt{-}\hlnum{1}\hlstd{,} \hlnum{10}\hlstd{))}
\hlstd{plot4} \hlkwb{=} \hlkwd{customplot}\hlstd{(sliced.wells)}

\hlkwd{grid.arrange}\hlstd{(plot1, plot2, plot3,}
    \hlstd{plot4,} \hlkwc{ncol} \hlstd{=} \hlnum{2}\hlstd{)}
\end{alltt}
\end{kframe}
\includegraphics[width=\maxwidth]{wellz/suppRef5} 
\end{knitrout}


\section{Interpolation and smoothers}

Common smoothers already available in base R can be applied to multiple
wells at the same time. However, because these algorithms
are dependent on the number of data points and not the distance between
the data points, the time courses will not 
be uniformly smoothed.

In many cases, different amounts of smoothing is desired at different
points in a curve. For example, we do not wish to smooth the sharp drop in
impedance after toxin is added. However, we do wish to smooth all other 
parts of the curve. In these cases, smoothing algorithms that are local (e.g., loess
and kernel smoothing) may be preferred over splines that minimize the residuals along
the entire curve.

In some cases, smoothing with R code took several minutes, so I wrote or attached open source
C++ and Fortran functions that calculate smoothers in less than one millisecond.


\begin{itemize}
  \item \texttt{smoother\_curfit} fits a cubic spline so that the sum of the residuals
matches a user-defined value (calls Fortran code from the DIERCKX package)
  \item \texttt{smoother\_bin} bins the data and averages each bin to reduce the size of the data set
  \item \texttt{smoother\_maverage} calculates a moving average in C++
  \item \texttt{smoother\_smooth.spline} is the cubic smoothing spline in the R `stats' package
  \item \texttt{smoother\_lokern} performs ``kernel regression smoothing with local plug-in bandwidth''
     as described by Herrman using the `lokern' R package developed by Herrman \cite{Herrmann:1997gx}.
  \item \texttt{interpolate\_linear} performs linear interpolation
  \item \texttt{smooth.pchip} calculates a piecewise cubic hermitian interpolating function. This
     function is used when aligning two curves.
  \item \texttt{smoother\_composite} is a multi-step smoother that I found was particularly
     useful for the impedance data in this chapter. First, the derivative
     of the curve is estimated from \texttt{smoother\_bin}.
     \texttt{smoother\_maverage} of the absolute value of the derivative is used to weight each data
     point so that rapidly changing parts of the curve are fit more tightly.
     The weights are translated to bandwidths with user-defined limits. 
     Using these bandwidths, local polonymials are fit with \texttt{smoother\_lokern} in order to estimate
     the residuals that are used in \texttt{smoother\_curfit}. This composite smoother requires
     more user-defined parameters than any other, but it is very consistent
     between different curves and different experiments.
\end{itemize}

The polynomial smoothers can be used to calculate derivatives and
integrals. Smoothers also have the added benefit of data compression, sometimes
reducing 10,000+ data points to a spline with less than 20 knots.

\begin{knitrout}
\definecolor{shadecolor}{rgb}{0.969, 0.969, 0.969}\color{fgcolor}\begin{kframe}
\begin{alltt}
\hlstd{huvecs.a} \hlkwb{=} \hlkwd{retrieveWells}\hlstd{(subset2,}
    \hlkwc{compounds} \hlstd{=} \hlstr{"TcdA"}\hlstd{)}
\hlstd{huvecs.norm} \hlkwb{=} \hlkwd{transform}\hlstd{(huvecs.a,}
    \hlkwd{c}\hlstd{(}\hlstr{"tcenter"}\hlstd{,} \hlstr{"normalize"}\hlstd{,}
        \hlstr{"slice"}\hlstd{),} \hlkwc{ID} \hlstd{=} \hlstr{"toxinAdd"}\hlstd{,}
    \hlkwc{xlim} \hlstd{=} \hlkwd{c}\hlstd{(}\hlopt{-}\hlnum{1}\hlstd{,} \hlnum{Inf}\hlstd{))}
\hlstd{huvecs.smth} \hlkwb{=} \hlkwd{add_smoother}\hlstd{(huvecs.norm,}
    \hlkwc{method} \hlstd{=} \hlstr{"composite"}\hlstd{,} \hlkwc{x.scale} \hlstd{=} \hlnum{2}\hlopt{/}\hlnum{3}\hlstd{,}
    \hlkwc{y.scale} \hlstd{=} \hlnum{1}\hlstd{,} \hlkwc{noise.scale} \hlstd{=} \hlnum{1}\hlopt{/}\hlnum{60}\hlstd{,}
    \hlkwc{deriv.cutoffs} \hlstd{=} \hlkwd{c}\hlstd{(}\hlnum{0.05}\hlstd{,}
        \hlnum{0.25}\hlstd{))}

\hlkwd{plot}\hlstd{(huvecs.smth[}\hlkwd{c}\hlstd{(}\hlnum{1}\hlstd{,} \hlnum{8}\hlstd{,} \hlnum{10}\hlstd{,}
    \hlnum{17}\hlstd{)],} \hlkwc{xlim} \hlstd{=} \hlkwd{c}\hlstd{(}\hlopt{-}\hlnum{1}\hlstd{,} \hlnum{3}\hlstd{),} \hlkwc{showpoints} \hlstd{=} \hlnum{TRUE}\hlstd{,}
    \hlkwc{smoother} \hlstd{=} \hlnum{TRUE}\hlstd{)}
\end{alltt}
\end{kframe}
\includegraphics[width=\maxwidth]{wellz/suppRef6} 
\end{knitrout}

\section{Custom metrics}
With this framework, it is easy for one to define new functions that calculate
metrics from the curves. Below is an example of a function (\texttt{max.rate}) that calculates
the maximum slope of curves. The slopes are then plotted versus toxin concentrations.

\begin{knitrout}
\definecolor{shadecolor}{rgb}{0.969, 0.969, 0.969}\color{fgcolor}\begin{kframe}
\begin{alltt}
\hlstd{subset} \hlkwb{=} \hlkwd{retrieveWells}\hlstd{(wells,}
    \hlkwc{compounds} \hlstd{=} \hlstr{"IMCE"}\hlstd{)}
\hlstd{subset.norm} \hlkwb{=} \hlkwd{transform}\hlstd{(subset,}
    \hlkwd{c}\hlstd{(}\hlstr{"tcenter"}\hlstd{,} \hlstr{"normalize"}\hlstd{,}
        \hlstr{"slice"}\hlstd{),} \hlkwc{ID} \hlstd{=} \hlstr{"toxinAdd"}\hlstd{,}
    \hlkwc{xlim} \hlstd{=} \hlkwd{c}\hlstd{(}\hlopt{-}\hlnum{1}\hlstd{,} \hlnum{Inf}\hlstd{))}
\hlstd{subset.smth} \hlkwb{=} \hlkwd{add_smoother}\hlstd{(subset.norm,}
    \hlkwc{method} \hlstd{=} \hlstr{"composite"}\hlstd{,} \hlkwc{x.scale} \hlstd{=} \hlnum{2}\hlopt{/}\hlnum{3}\hlstd{,}
    \hlkwc{y.scale} \hlstd{=} \hlnum{1}\hlstd{,} \hlkwc{noise.scale} \hlstd{=} \hlnum{1}\hlopt{/}\hlnum{60}\hlstd{,}
    \hlkwc{deriv.cutoffs} \hlstd{=} \hlkwd{c}\hlstd{(}\hlnum{0.05}\hlstd{,}
        \hlnum{0.25}\hlstd{))}
\hlstd{subset.rate} \hlkwb{=} \hlkwd{max.rate}\hlstd{(subset.smth,}
    \hlkwc{ID} \hlstd{=} \hlstr{"toxinAdd"}\hlstd{,} \hlkwc{min.diff} \hlstd{=} \hlnum{10}\hlopt{/}\hlnum{60}\hlopt{/}\hlnum{60}\hlstd{,}
    \hlkwc{ylim} \hlstd{=} \hlnum{0.8}\hlstd{,} \hlkwc{xlim} \hlstd{=} \hlnum{2}\hlstd{)}
\hlstd{MaxS} \hlkwb{=} \hlkwd{groupMetric}\hlstd{(subset.rate,}
    \hlkwc{ID} \hlstd{=} \hlstr{"toxinAdd"}\hlstd{,} \hlkwc{metric} \hlstd{=} \hlstr{"max.rate"}\hlstd{)}
\hlkwd{plotMetric}\hlstd{(MaxS)}
\end{alltt}
\end{kframe}
\includegraphics[width=\maxwidth]{wellz/suppRef7} 
\end{knitrout}

Here is the code for the function

\begin{knitrout}
\definecolor{shadecolor}{rgb}{0.969, 0.969, 0.969}\color{fgcolor}\begin{kframe}
\begin{alltt}
\hlkwd{print}\hlstd{(max.rate.Well)}
\end{alltt}
\begin{verbatim}
## function (well, ID, duration = Inf, nbw = 10, 
##           min.diff = NULL, ylim = 0.75, xlim = 2,
##           smoother = TRUE) 
## {
##     dwell = spline_interpolate(well, deriv = 1, 
##              nbw = nbw, smooth = smoother, 
##              min.diff = min.diff)
##     times = timesdata(dwell)
##     dm.deriv = datamat(dwell)
##     dwell = spline_interpolate(well, deriv = 0, 
##         nbw = nbw, smooth = FALSE, 
##         min.diff = min.diff)
##     vals = welldata(dwell)
##     sweep = getrows.by.ID(dwell, ID)$t1
##     idx = which(sweepdata(dwell) == sweep)
##     tstart = times[idx]
##     idx.range = times > tstart & 
##                 times < (tstart + duration)
##     idx.spike = times < (tstart + xlim) & vals > ylim
##     dm.deriv = dm.deriv[idx.range & !idx.spike, ]
##     id = which.min(dm.deriv$values)
##     rate = dm.deriv[id, c("time", "values")]
##     colnames(rate) = c("time", "value")
##     well$metrics$max.rate = 
##          as.data.frame(as.list(rate))
##     return(well)
## }
\end{verbatim}
\end{kframe}
\end{knitrout}

\section{Future plans}
After approval from the maintainers of R packages, this will be
a standard R package accessible from the Comprehensive R Archive Network (CRAN).





