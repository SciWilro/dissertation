\chapter{Acknowledgements}

First and foremost, I would like to thank my family for their
love and continued support during my graduate education. 
Throughout my work in the past few years, I have finally begun to realize
the degree of the dedication of my parents, Mike and Jennifer, to
my sister and me while at the same time succeeding
in their even more challenging careers. I thank my sister and 
best friend, Wendy, for showing me how to survive, always keeping my spirits up,
and keeping me grounded and sane.

I would not be adequately equipped for my research if it weren't for
the faculty, students, and researchers in Biomedical Engineering
and the Biomedical Sciences at UVa.
As an engineer, I have had the incredible opportunity to work day in and day out
with biologists, learning how to develop and contextualize my work.
At the same time, the faculty in Biomedical Engineering have developed my intuition for applying
models to practical problems.
I've been fortunate to ``grow up'' in 
this multilingual scientific community, unknowingly becoming fluent in different 
``scientific languages''.

My dissertation would not exist if it wasn't for collaborations
and help from colleagues and friends.
The students with whom I've worked with the longest---Arvind, 
Edik, Paul, Phil, Anna, and Jennie---have provided constant critical feedback
that has improved my work and kept me focused. Edik Blais, the other
the other developing statistician in lab, has been
a fantastic resource, my go-to person for improving my visualizations.
I'd especially like to thank
Arvind Chavali, a colleague with whom I collbarated with many times, but even more
a friend who showed me the ropes, and has and I'm sure will continue to provide advice
for years to come. I've been very fortunate to have wonderful fellow students that
have made my time at UVa so enjoyable, whether it be cheering for the 'Hoos
or discovering Charlottesville's most interesting locations and activities.

I'm indebted to the handful of researchers with whom I have worked most closely
since my first day of graduate school. 
Mary Gray and Gina Donato, who always happily
answered my many questions, taught me most everything I know at the bench.
Without their instruction, I would surely have spent many months of frustration
troubleshooting every problem I encountered.
Cirle Warren, in our group's weekly meetings, has helped me
put my research into clinical cotext.
Glynis Kolling has been a tremendous resource, showing me what 
is and is not possible experimentally and helping me choose the best
path to follow based on my many crazy ideas.
I've had the privilege to work side by side with this group of researchers,
at the bench and at the whiteboard hammering out ideas and manuscripts.

Finally, I'd like to thank those on my dissertation committee:
Shayn Peirce-Cottler, Kevin Janes, Alison Criss, Erik Hewlett, and Jason Papin.
My discussions with them have shown me their genuine interest in my career.
They have pushed me far beyond my comfort zone as only
a clever engineering student to become a better thinker, to step
back, understand the broader goals, and decide the most beneficial
steps to achieve those goals.

Erik Hewlett has graciously served as an unoffical secondary advisor,
and I am grateful for the time he has given me the past five years.
After every experiment, presentation, and paper, he has been right there
excited to hear how things are going, ready to provide encouragment and criticism.
As someone from an engineering background where tools and 
methods are neat exciting,
he has contributed greatly to my ability to logically think step-by-step
through a complete study. I am also greatly appreciative for the opportunities he has provided.
I have been able to meet and work with many other great people through his initiatives.

Jason Papin has been everything that a great mentor should be.
I still don't understand how he is available to me and everyone else
whenever we need him.
I have and will continue to try to emulate him as I mentor others.
It then should go without saying that he has helped me develop
into a better scientist and all that goes along with being successful in academia.
I credit much of my progression from a fault-finding, cynic to more of a  
forward-thinking optimist to his influence.
Beyond this, what has been remarkable to me is his genuine interest for 
my own personal and career goals. It is for this that I am most thankful.
He has shown me how to better approach problems, not just scientific ones, and has
given me the freedom to think independently, to fail and to succeed on my own.


\chapter{Preface}

\section{Me, a biomedical engineer}

I don't label myself an engineer in the traditional sense; 
I don't design things to be manufactured. I also dislike 
being called a scientist because of the visual it may provoke; 
most of my time isn't spent in a long white coat at a 
bench. Though others might think me one, I don't consider 
myself a statistician, computer scientist, or biologist either.

I am a scientist in that I’m curious about nature, especially 
health and medicine. I’m an engineer in that I’m curious 
about translating scientific understanding to ideas and 
tools that will improve others' quality of life. To do 
this, I use and invent tools in mathematics, statistics, 
computer science, and the biological sciences. There is 
no formal and universally accepted definition of a 
``biomedical engineer'' and likely never will be. However, 
the above description is the one I use and, I believe, is
close to the definition of many other self-proclaimed 
biomedical engineers.

\section{Me, a computational systems biologist}

As a biomedical engineer in the information age, trillions of 
data points are available. In this dissertation alone, I gather 
data from dozens of expression states of the 3.3-billion 
letter human genome or 2.8-billion letter mouse genome. Tens 
of thousands of megapixel images of individual cells are 
captured at different wavelengths of light. Tissue and blood 
samples from hundreds of mice under different stresses are 
analyzed microscopically and by molecular assays to quantify 
pathology, the proportion of different cells (e.g., epithelial, 
white blood cell, etc.), and the amounts of dozens of proteins. 

However, these data points independent of each other are of little 
use. As an example, consider a mouse that ingests a toxin. If we then observe 
inflammation in the mouse’s intestine, we wonder about the cause.
After dissecting tissue sections, we then find that a particular 
gene is expressed prior to inflammation. Is this gene responsible for 
inflammation? Are there other genes whose regulation is 
linked to this gene of interest? How does expression 
of this gene translate to the amount of its gene product, 
the protein that physically interacts with the cells’ 
environment? Is this inflammation only due to local 
events? What about the brain and the nervous system? What 
happens after inflammation? Are there any changes throughout 
the body from the local injury in the intestine (e.g., in 
the blood)? Where did the toxin go? The questions go on, 
but it is apparent that there are many levels of data and 
interactions, a \textbf{system}, that determine the apparent 
clinical manifestation. \textbf{Systems biology} aims 
to consider biological systems as a whole and answer how 
one or many changes affects the state or output of the 
entire system.

Systems biology requires computational tools just so data 
can be managed, so the term \textbf{computational systems biology} 
is somewhat redundant. The term’s definition will change 
from person to person. As I use it in this dissertation, 
it is distinct from ``systems biology'' in that computational tools 
are used to make insights and comparisons that would be experimentally 
impossible, or would at least be prohibitively difficult.
Computational systems biology may take well-designed, simple, 
compartive experiments that are the core of good scientific 
research (e.g., think null versus 
alternate hypotheses and p-values) and then go one step further
by presenting the data in novel ways and by making formal, rigorous predictions. 


\section[My dissertation as systems biologist]{My dissertation as a biomedical engineer 
              and computational systems biologist}

As I described above, Biomedical Engineering and Computational Systems
Biology include aspects of many different fields (biology, medicine, engineering, etc.).
Therefore, this dissertation includes new contributions and tools to these fields, 
allowing for better descriptions and reliable, repeatable predictions from biological data.
The unifying element to these contributions and tools is one research question:
why do a bacterium's toxins make us sick and what are ways
to make us better once we're sick? More specifically,
I show how multi-level, systems biology data improves our understanding of host cell responses 
to the two principal virulence factors of \textit{Clostridium difficile}, toxins A and B. 
This understanding suggests new treatments and diagnostics, but also 
reveals entirely new ways of thinking that offer leads to promising targets
for future treatments. 
















