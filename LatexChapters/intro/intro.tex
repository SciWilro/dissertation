
\chapter{Introduction}\label{chapter:intro}




Consider a social science metaphor where people replace biological 
components like genes or cells. We may hypothesize that 
introverts have fewer friends than extraverts and through 
a survey find that this is indeed true. However, a network 
diagram of connections in a social network with more metadata 
may reveal richer, more specific interpretations (e.g., 
clusters of introverted individuals or people that are 
hubs, the social organizers). We then go even further 
and develop models that predict what happens when a 
social organizer is moved from one part of the network to 
another, or we may predict what combination of characteristics 
(hobbies, musical preferences, etc.) will change the number 
of friends one will have. In a similar way, computational 
systems biology obtains many types of data about proteins, 
genes, cells, organs, etc. and model and predict how they 
will affect one another and how perturbations like drugs 
will alter the system’s state.




Great comedians tell us of everyday experiences in new, 
unexpected ways. Novelists and poets express our thoughts 
better than we could. Statisticians place abstract common 
sense into concrete writing. My job is to take things that 
were always there, but we didn’t know. To present and 
communicate in a way that allows us to do things in a 
novel way that lead to discoveries and ways of thinking.  
Use many tools to reveal findings from data. Typically, 
models greatest usefulness are there ability to guide us, 
not to tell us the answers.


There is always a constant tug from “hypothesis generating” 
research and “hypothesis generated” research. Discovery 
based versus logical yes/no experiments. It’s a spectrum, 
but it’s how we think about it.








