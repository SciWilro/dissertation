
\chapter[miRNA Transcriptomics of melanoma]{ Transcriptomics characterize responses to melanoma treatment }

\section{Motivation}

Computational models that use millions of data points
often require complex, multi-step analyses that few can implement or understand completely, 
yet the end goal is most always 
to find simple, fundamental relationships
that most anyone can intellectually grasp. However, simple, well-designed
summaries and descriptions of the data are, arguably, of equal importance if not
more important than a predictive or mechanistic model.
Transitioning from a computer and data to logic and understanding is a
bottleneck for the sharing ideas with a larger community
that can derive new interpretations to advance medicine.
Visual summaries allow us to take advantage of the best
analytical tool we have, the human intellect.
Here, I present a phase II
clinical trial in which I guided the analyses and presentation of the data
after the completion of the treatment period \cite{Wagenseller:2013fj}.
Through this example, I show how appropriate visualizations identified
errors in previous analyses and enabled new interpretations and new
hypotheses to be generated.

The clinical trial was directed by Dr. Craig Slingluff at the
University of Virginia. Aubrey Wagenseller, the first author on the associated
publication \cite{Wagenseller:2013fj}, drafted the manuscript 
and led follow-up experiments. A more general and briefer background
of that given in the manuscript is provided below. My analsysis as presented
in the publication (figures and results) are also presented in this section. 
In addition, a more in depth description of the
methods are also presented. This important addition allows one to understand
the choice of methods and also reproduce
the results shown in the publication, a process which is not possible from the
supplemental data provided in the published manuscript.

\section{Introduction}

A metastatic melanoma (more formally `stage IV melanoma') is a type of cancer
in which melanocytes (cells that produce the pigmant melanin that is found in
the skin, eye, inner ear, etc.) grow improperly or uncontrollably 
and spread to other parts of the
body. Even with current treatments, the two-year survival rate is 
under 20\%, so there is a need for new therapeutics
or improvements upon current therapies \cite{Robert:2011vp}\cite{Hodi:2010wt}.
To find more potent, target-specific drugs, several studies have targeted
molecular pathways known to be disregulated in many melanomas \cite{Ko:2011dk}.
Clinical trials with many of these monotherapies have had variable results:
3\% response rate in a Temsirolimus (targeting PI3K-AKT-mTOR pathway) 
trial and 0\% and 17\% response rates in two Bevaczimub (targeting VEGF)
trials \cite{Margolin:2004vl,Varker:2007dj,Schuster:2012fo}.
However, combination therapies that simultaneously target multiple
pathways have potential to succeed where single drugs have failed.
For example, Molhoek et al. found that dual targeting of VEGF and mTOR
with bevacizumab and sirolimus synergistically reduced growth
and caused death in VEGFR-2$^{+}$, patient-derived, melanoma cell lines \cite{Molhoek:2008jx}.
In a follow-up phase II trial with 17 patients treated with temsirolimus+bevaczimub,
three patients partially responded, nine had stable disease after 8 weeks,
four had progressive disease, and one patient could not be 
evaluated (\autoref{jtm:table1}) \cite{Slingluff:2013ej}. In this clinical study, additional miRNA data
was taken from patients before and after treatment. This analysis aimed to 
identify (1) if pre-treatment miRNA expression profiles 
correlate with treatment response or (2) if any post- versus pre-treatment
changes in miRNA expression correlate with the treatment effectiveness.

\begin{table}
\begin{center}
\begin{tabular}{ c c c c c c }
Patient & Pre-tx & Post-tem & Post-combo & BRAF\textsuperscript{V600E} & Outcome \\ \hline
1 &   &   & + & ND    & Stable disease \\
2 & + & + & + & Y & Stable disease \\
3 & + & + & ND & Y & Stable disease \\
4 & + & + & + & Y & Stable disease \\
5 & + & + & + & Y & Progressive disease \\
6 & + & + & ND & N & Partial response \\
7 & + & + & + & N & Progressive disease \\
8 & + & + & + & Y & Progressive disease \\
9 & + & + & + & N & Partial response \\
10 & + & + & + & N & Not evaluable \\
11 & + & + & ND & N & Stable disease \\
12 & + & + & + & N & Stable disease \\
\end{tabular}
\end{center}
\caption[Patient outcomes and availability of miRNA data]{
    \textbf{Patient outcomes and availability of miRNA data.}
    A `+' indicates that miRNA was measured from biopsies 
    before treatment (Pre-tx), after temsirolimus treatment (Post-tem),
    or after bevacizumab+temsirolimus treatment (Post-combo). All patients
    received each treatment unless denoted with ND (not done). Blank entries indicate where
    ample RNA could not be obtained.}
\label{jtm:table1}
\end{table}

\section{Methods}

\subsection{miRNA quantification}
500 ng of total RNA was extracted from formalin-fixed, parafin-embedded
tissue sections and labeled with the dye Hy3. A universal reference sample was labeled with
Hy5. After image processing, the local \textit{median} background 
signal was subtracted from the \textit{median} signal 
of each spot on the array. The log of the ratio of these background-corrected signals
($log(\text{Hy3}/\text{Hy5})$) was then
normalized using LOWESS fits (fitting $log(\text{Hy3}/\text{Hy5})$ versus  
$log(\text{Hy3} \times \text{Hy5}) / 2$ ).
These intra-array normalized ratios were denoted as the `log median ratio' (LMR).
Inter-array normalization was not performed because of the built-in normalization provided
by the universal reference sample.
The LMR metric will be unfamiliar to most researchers who use 
popular single-channel, high-density RNA microarrays (e.g. Affymetrix).
However, the universal reference sample in this experimental
design better accounts for errors from cross-hybridization or
probe-specific affinities. The raw data processing was performed by the manufacturer of the
arrays. The images of the scanned arrays are not publicly available.
The raw data files publicly available online (GEO \#GSE37131) only include
the signal intensities for one channel, which makes it impossible to reproduce
the data normalization. The processed data (the LMR values provided to me) are available
online as an XML file. The code for parsing and analyzing this XML file is provided online
in my public code repository. For more information about experimental methods, see
Wagenseller et al. \cite{Wagenseller:2013fj}.



\section{Results}

\subsection{miRNA expression of all samples}

\autoref{jtm:fig1} summarizes the miRNA data from tumors that could be biposied and profiled
(14 of the 17 patients). The row of colors above the heatmap indicate the patient
from which miRNA samples came. The shapes above each column of the heatmap represent the time
at which the sample was obtained. Lines indicate pre-treatment samples. Filled circles represent
samples from patients one day after temsirolimus treatment (day 1). Patients received
bevacizumab+temsirolimus (day 8) and then temsirolimus (day 15). ``Double-circles'' indicate
post-combination treatment samples
that were obtained on day 23. The heatmap colors show the expression of each miRNA relative to
a universal reference sample.

\begin{figure}[h!]
  \centering
  \includegraphics[width=1\textwidth]{jtm/Figure1}
  \caption[miRNA expression profiles pre- and post-treatment]{
       \textbf{miRNA expression profiles pre- and post-treatment}
       The 50 miRNAs with the greatest variance across all samples are shown.
  }
  \label{jtm:fig1}
\end{figure}

The dendrogram in \autoref{jtm:fig1} shows a hierarchical
clustering of miRNA samples based on correlation coefficients. 
This clustering shows that the strongest
factor distinguishing the tumors is the patient. In other words, patient to patient
variability is greater than variability from any other factor such as pre- versus post-treatment.
The tight clustering of technical replicates (shaded gray) showed that variations
were due experimental or human error.
In the initial analysis, there were only four 
samples from patient 7 (there are five samples
shown in \autoref{jtm:fig1}). However, the clustering suggested
 that one of the pre-treatment samples for patient
7, which had been annotated as patient 6, was mislabeled. After checking lab notebooks
and shipping documents, we rectified the problem. Previous analyses did not visualize these results.
Only p-values were calculated, and so this mistake had not been identfied. After correcting
the annotations, the variance of samples from patients 6 and 7 decreased greatly.
This, in turn, increased statistical power so that additional differentially expressed miRNAs could
be detected.

This initial profiling provided a general understanding of the structure and scale
of the data. However, because the patient-to-patient variability so strongly
dominated the overall variability, we performed statistical tests to focus
on treatment effects.

\subsection{miRNA expression changes after treatment}

To identify differentially expressed miRNAs, paired t-tests were
performed by comparing pre- and post-treatment groups (samples from
the same patient were paired). \autoref{jtm:fig2} plots the p-values
from these tests against the effect size (shown as $\text{dLMR} =
\text{LMR}_{\text{post-treatment}} - \text{LMR}_{\text{pre-treatment}}$).
These ``volcano plots'' helps identify differences that are statistically
signficant (y-axis) while the x-axis helps identify large effects sizes
that are often interpreted as more biologically significant. miRNAs with
low p-values and large effects sizes are in the upper left and upper right
corners of the plot. The samples used in the comparisons can be
seen in \autoref{jtm:fig1} or \autoref{jtm:table1}. This visualization revealed
a technical error in previous analyses that was to be used for publication before
I joined the study. There was an unusual distribution of p-values with
an overabundance of values close to 1.0, which appeared to be an artifact of the code used
to calculate the statistics.


\begin{figure}[h!]
  \centering
  \includegraphics[width=0.65\textwidth]{jtm/Figure2}
  \caption[Effect size and statistical significance of post- versus pre-treatment miRNA expression]{
       \textbf{Effect size and statistical significance of post- versus pre-treatment miRNA expression}
       Black circles represent individual miRNAs. Red circles indicate miRNAs of interest. Closed red
       circles are putative tumor supressors.
  }
  \label{jtm:fig2}
\end{figure}

The volcano plots were used to defines miRNAs of interest, with the cutoffs
being $|dLMR|>0.5$ and $p<0.01$ (gray lines in \autoref{jtm:fig2}). 
No miRNAs met these cutoffs in the
post-temsirolimus treatment versus pre-treatement comparison. However,
there were 15 miRNAs of interest when comparing post-combination treatment
versus pre-treatment. One concern from this analysis may be that the t-test
is inappropriate if the effect sizes are not normally distributed. I therefore
performed a permutation based statistical test (`Significant Analysis of Microarrays')
and found the same 15 miRNAs as the most significant. Such close agreement
between statistical tests is rare. The same findings with two statistical tests
further supports these miRNAs as truly differentially expressed.
The expression of the 15 miRNAs was validated by qRT-PCR
Several of these miRNAs have predicte targets that are known
to be oncogenic. For a discussion of these targets, see Wagenseller
et al. \cite{Wagenseller:2013fj}.

\subsection{Different miRNA expression changes between responders and non-responders}
Last, we sought to investigate if there are particular miRNAs or sets
of miRNAs whose pre-treatment expression would separate the responders from the non-responders
(patients with `progressive disease'). This is a two-class
classification problem (10 total samples) with 1,300 features (1,300 miRNAs). 
Any linear model with 10 of the miRNAs will be guaranteed to perfectly predict
responers. The problem is then to identify which 10 or fewer miRNAs
most reliably predict responders versus non-responders. Backwards subtraction 
selection of feautures finds many possible sets of miRNAs
that perfectly predict responders. The problem of many possible models remains.
Often, miRNAs are correlated to each other so that
the independence assumption of linear models are violated. I investigated 
regularized linear models (lasso and ridge regression), yet there was still no clear
best set of miRNAs that predicted responders. Cross-validation of many models
worked 100\% of the time. Other predictive models (e.g. support vector machines and
decision trees) had similar problems. After conceding that there is no `best' model given
the miRNA data, I decided to present the four miRNAs with $p<0.01$ and $|\text{dLMR}|>0.5$
(\autoref{jtm:fig4}A). The heatmap shows that any one of these miRNAs, with an
appropriately selected cutoff, are capable of predicting responders versus non-responders.
\autoref{jtm:fig4}B shows miRNAs with $|\text{dLMR}|>0.5$ and $p<0.04$ when comparing post-combination
treatment to pre-treatment expression. Finally, BRAF\textsuperscript{V600E}, BRAF\textsuperscript{WT}, melanomas are 
treatable with vemurafenib. Since other treatments that target other pathways in BRAF\textsuperscript{WT}
are needed (e.g. temsirolimus or bevacizumab), we also tried to distinguish 
miRNAs that differ between BRAF\textsuperscript{V600E} and BRAF\textsuperscript{WT}.
The separation of samples based on mutational status is more difficult (\autoref{jtm:fig4}C).
However, there are a collection of miRNAs that warrant further investigation in the future.

\begin{figure}[h!]
  \centering
  \includegraphics[width=1\textwidth]{jtm/Figure4}
  \caption[Different miRNA expression between responders and non-responders]{
       \textbf{Different miRNA expression between responders and non-responders.}
       Green numbers and branches indicate non-responders. Red numbers and branches indicate
       BRAF\textsuperscript{V600E} tumors.
  }
  \label{jtm:fig4}
\end{figure}

\section{Conclusions}

Proper use of simple statistical tests and well-designed visualizations corrected for
small, but critical errors in the analyses of high throughput biological data
obtained from a clinicial trial (\autoref{jtm:fig1}).

Bevacizumab+temsirolimus treatment affects miRNA expression more than temsirolimus treatment alone,
suggesting that the two drugs are at least additive in their effects on melanoma miRNA expression (\autoref{jtm:fig2}).
This is in line with their synergistic growth-reducing effects in melanoma cell lines \cite{Molhoek:2008jx}.

The pre-treatment expression of small sets of miRNA
predicted the success of bevacizumab+temsirolimus treatment (\autoref{jtm:fig4}A). 
The expression changes of 7 miRNAs could classify responders versus non-responders; these
miRNAs may play a role in the response to treatment (\autoref{jtm:fig4}B).





