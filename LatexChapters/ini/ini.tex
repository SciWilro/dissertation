\chapter{ INI chapter }

\section{Abstract}

Toxin A (TcdA) and toxin B (TcdB) of Clostridium difficile cause gross pathologic changes (e.g., inflammation, secretion, and diarrhea) in the infected host, yet the molecular and cellular pathways leading to observed host responses are poorly understood. To address this gap, we evaluated the effects of single doses of TcdA and/or TcdB injected into the ceca of mice and several endpoints were analyzed, including tissue pathology, neutrophil infiltration, epithelial-layer gene expression, chemokine levels, and blood-cell counts—2, 6, and 16h after injection. In addition to confirming TcdA’s gross pathologic effects, we found that both TcdA and TcdB resulted in neutrophil infiltration. Bioinformatics analyses identified altered expression of genes associated with the metabolism of lipids, fatty acids, and detoxification; small GTPase activity; and immune function and inflammation. Further analysis revealed transient expression of several chemokines (e.g., Cxcl1 and Cxcl2). Antibody neutralization of CXCL1 and CXCL2 did not affect TcdA-induced local pathology or neutrophil infiltration, but it did decrease the peripheral blood neutrophil count. Additionally, low serum levels of CXCL1 and CXCL2 corresponded with greater survival. Though TcdA induced more pronounced transcriptional changes than TcdB and the upregulated chemokine expression was unique to TcdA, the overall transcriptional responses to TcdA and TcdB were strongly correlated, supporting differences primarily in timing and potency rather than differences in the type of intracellular host response. In addition, the transcriptional data revealed novel toxin effects (e.g., altered expression of GTPase-associated and metabolic genes) underlying observed physiological responses to C. difficile toxins.

\section{Introduction}

The toxins TcdA and TcdB are two key virulence factors of C. difficile, an intestinal, opportunistic pathogen responsible for more than 300,000 infections in the US per year (2009 data) with several estimates of annual cost between $433 million and $8.2 billion (1–4). Clinical manifestations include leukocytosis and diarrhea. The importance of TcdA and TcdB is underlined by the facts that strains without either toxin colonize but do not cause disease and that intoxication causes similar manifestations as infection (5–7). TcdA and TcdB are similar in size, amino acid sequence, and enzymatic specificity, yet exhibit different enzymatic activities and in vivo potencies (8–10). Furthermore, much remains unknown about common and divergent cellular pathways leading to toxin-mediated host responses (11, 12).

Determining the relative roles of TcdA and TcdB in pathogenesis has proven difficult in part because of variable findings within and between animal models as well as species-specific responses. Clinically, strains lacking TcdA are commonly isolated from infected patients, and no TcdA+/TcdB- clinical strain has ever been reported (13). Toxin effects in the context of infection have typically been studied using animal models in which an antibiotic regimen and subsequent disruption of intestinal flora must precede infection with C. difficile (14, 15). By generating mutant strains, Lyras et al. found that TcdB but not TcdA was essential for hamster infection, yet Kuehne et al. found, in a similar hamster infection model, that either toxin was sufficient (5, 6). Investigating toxin effects more directly, multiple intoxication models have demonstrated TcdA to be enterotoxic, while TcdB caused little to no pathology (7, 10, 16). However, epithelial damage in human xenografts in mice is greater with TcdB than TcdA, suggesting that many differences in toxin effects may be species-specific (17). The ability of either toxin to bind, enter, and/or activate intestinal cells may also explain differential effects of TcdA and TcdB. The sequences differ most in the C-terminal binding domain. TcdB has been shown to be incapable of binding the brush border membranes of hamsters, although TcdB has been found to further damage bruised ceca, synergize with TcdA, and contribute to pathogenesis during infection (5, 16, 18). Multiple receptors for TcdA have been proposed or identified, yet the roles of these receptors in different organisms, animal models, and cell types are unclear (19–24). TcdB weakly binds various trisaccharides and oligosaccharides, yet no functional receptor for TcdB has been identified (25). It is also possible that differences in intracellular actions of TcdA versus TcdB are responsible for differences in the host response. Though a similar dose of TcdA or TcdB may result in different gross pathologies, it is unclear if entirely different pathways are activated or repressed or if the same overall functions are affected to different degrees. We previously analyzed the transcriptional response of a human, ileocecal, epithelial cell line (HCT8) to TcdA and TcdB and showed that the toxins induce very similar transcriptional signatures, yet the effects of TcdB occurred earlier (26). In addition, we found altered regulation of many genes involved in cell growth and division but no overwhelming expression of inflammatory markers or other genes associated with physiological changes in vivo. The in vivo effects of these toxins have not been investigated by measuring genome-wide responses, and many of the links between cellular responses and physiological changes remain unknown. We therefore used an in vivo system, intracecal injection of toxin into mice, and collected samples to characterize the genome-wide cellular responses and gross physiological effects of each toxin over a 16h time course.

It has been difficult to tease apart the aspects of the host response to TcdA and TcdB because of the important interactions among the many tissues, cell types, and signals involved (27, 28). The intestinal epithelium, the initial barrier to these toxins, continuously interacts with surrounding cells throughout the development and resolution of disease. We therefore focused on the transcriptional response of epithelial-layer cells to toxin and other toxin-related effects. Given the importance of surrounding tissues and with recent evidence of systemic dissemination of toxins, we chose cecal injection of toxin, an open system, as opposed to closed ileal loop models or ex vivo systems that may restrict toxin to a limited area (29). Additionally, previous studies have focused on separate facets of the host response, typically with only one toxin per study (30–36). To address these deficits in the knowledge of this illness, we measured genome-wide expression from epithelial-layer cells exposed to TcdA and TcdB to simultaneously capture effects of each toxin.

Using this approach, we identify several genes differentially expressed after toxin treatment that serve as specific candidates to investigate further. Additionally, we employ currently available bioinformatic methods and also introduce novel methods to identify groups of regulated genes associated with known biological functions. Our measurements were taken on several biological levels to link changes in one set of variables (e.g., gene expression) to changes in others (e.g., pathology and blood counts). These linkages serve as tools to validate previous findings as well as identify novel functions affected by TcdA or TcdB. Of the many linkages that could be explored, we further investigated chemokine expression and the role of two chemokines in the response to TcdA cecal injection with respect to changes in pathology, neutrophil recruitment, and survival. In addition to the comparison between toxins and identification of differentially expressed genes, these associations and concepts serve as a basis for further probing the host response to these toxins in the context of C. difficile infection.

\section{Methods}

\subsection{ Cecal injection }
All procedures involving animals were conducted in accordance with the guidelines of the University of Virginia IACUC (Protocol \#3626). Purified TcdA and TcdB were generously provided by Dr. David Lyerly at TECHLAB, Inc. Mice (male C57BL/6J, 8 w.o. from Jackson Laboratories) were anesthetized with ketamine/xylazine in preparation for surgery. A midline laparotomy was performed to locate the cecum, and 20 µg of toxin in 100 µL of 0.9\% normal saline was injected into the distal tip. Incisions were sutured, and animals were monitored during recovery. Sham injected animals received only 100 µL of saline. If an animal became moribund (i.e., hunched posture, ruffled coat, or little to no movement), they were immediately euthanized. 

\subsection{ Cell culture }
An immortalized, C57BL/6 mouse, cecal epithelial cell line (passage 9) was provided from the laboratory of Dr. Eric Houpt and maintained as described by Becker et al. (37). Toxin cytopathicity was assessed by measuring changes in cell adherence and morphology using a multi-well, continuous, electrical impedance assay (xCelligence; ACEA Biosciences). In each well, 20x solutions of TcdA or TcdB (prepared in media) were added 34h after seeding 21,000 cells yielding the indicated concentrations.

\subsection{ Blood counts }
Blood was collected using cardiac puncture and complete blood counts were measured using a HEMAVET 950FS (Drew Scientific). Serum was analyzed for levels of systemic chemokines using MILLIPLEX MAP beads, and the signal was measured using a Luminex 100 IS System (UVA Flow Cytometry Core Facility).

\subsection{ Histology }
A cross section from the middle of each cecum was dissected and fixed. The tissues were paraffin-embedded, sectioned, and stained by the UVA histology core. H\&{}E sections were coded and scored by a blinded observer using parameters to assess inflammation, luminal exudates, mucosa thickening, edema, and epithelial erosions (38). Each of these five parameters was scored between zero and three yielding total pathology scores between zero and fifteen. Eosinophils were detected in tissue using Congo Red Staining (39). Tissues for H\&{}E, MPO, and eosinophil staining were fixed in Bouin’s solution; tissues for other measurements were fixed in 4\% paraformaldehyde. Immunohistochemistry was performed by the Biorepository and Tissue Research Facility at the University of Virginia. Monocytes/macrophages, dendritic cells, and neutrophils were separately identified using the markers F4/80 (clone CI:A3-1; AbD Serotec), Ly75 (EPR5233; Abcam), and myeloperoxidase (MPO; rabbit-anti-MPO; Novus Biologicals), respectively. The presence of neutrophils was quantified by averaging the number of positive cells associated with epithelial and subepithelial layers in ten random fields (40x objective). Monocytes/macrophages and dendritic cell staining was scored by analyzing each section for the number of positive cells and overall staining intensity.  Samples were assigned scores of 1 (few cells/weak staining), 2 (moderate staining), or 3 (many cells/intense staining). 

\subsection{ Isolation of cells from cecal tissue }
The remaining two sections of each cecum were opened longitudinally, rinsed with Hank’s Balanced Salt Solution (HBSS, Gibco), and shaken at 250 rpm for 30 minutes at 37°C in HBSS containing 50mM EDTA and 1mM DTT in order to remove epithelial layer cells. The digested tissue was strained with  a 100-µm cell strainer, and the filtrate was centrifuged (1,000×g, 4°C, 10 min). Cells were resuspended in red-cell lysis buffer (150 mM NH4Cl, 10mM NaHCO3, 0.1 mM EDTA) and centrifuged again. The pelleted cells were used immediately for flow cytometry or stored at -80°C until RNA isolation. RNA was isolated using the RNeasy mini kit (Qiagen) with on-column DNase digestion according to manufacturer’s instructions. Protein was collected from cell lysate supernatants, which were made using a lysis buffer containing 50 mM HEPES, 1\% Triton-X100, and HALT protease inhibitor. The lysate was incubated on ice for 30 minutes and centrifuged (13,000×g, 4°C, 10 min). Clarified supernatants were stored at -80°C.

\subsection{ Flow cytometry }
Epithelial layer cells were isolated from mice 16h after injection with TcdA (n=2), TcdB (n=3), or saline control (n=3). The cell preparations were stained with markers including CD3ε (BV421; BioLegend), cytokeratin (PE; Novus Biologicals), CD11b (APC-Cy7; BioLegend), B220 (FITC; BD Bioscience), and CD45 (V500; BD Bioscience) to detect different populations.  Samples were analyzed on the CyAn ADP analyzer; 50,000 events were collected and subsequently analyzed using FlowJo software.  

\subsection{Antibody-mediated neutralization of chemokines}
For each antibody, 100 \textmu{}g was administered by intraperitoneal injection 16h before sham/toxin injection. Mice received a combination of anti-CXCL1 (clone 48415) and anti-CXCL2 (clone 40605) or the relevant isotype controls (clones 54447 and 141945). All antibodies were purchased from R\&{}D systems and resuspended according to the manufacturer’s directions. Decreased levels of CXCL1 and CXCL2 in the serum of mice receiving neutralizing antibody (compared to isotype controls) were indicative that the neutralization was successful.

\subsection{ Microarray procedure }
An Agilent 2100 BioAnalyzer was used to assess RNA integrity. cDNA was synthesized, biotin labeled, and hybridized to Affymetrix Mouse Genome 430 2.0 GeneChip according to the manufacturer’s instructions. Arrays were scanned with a GeneChip System 3000 7G (Affymetrix).

\subsection{ Statistical analysis }
Unless otherwise stated, each two-sample test is a two-sided Mann-Whitney U test.

\subsection{ Microarray analysis }
Full descriptions of the computational and statistical analyses as well as the data and computer code to reproduce all analyses, figures, and tables are included in the Supplement. Briefly, each microarray was background corrected according to the RMA algorithm, and probe sets were summarized using the median polish algorithm (40). All arrays were normalized to each other by cyclic loess normalization. Differentially expressed genes were detected using the Cyber-T statistical test with a Benjamini-Hochberg p-value correction (41). Competitive gene set enrichment was performed with a modified version of Wu et al.’s “correlation adjusted mean rank gene set test” (CAMERA) (42). Self-contained gene set enrichment was performed by calculating gene-gene correlations within each gene set.

\section{Results}

\section{Discussion}

\section{Conclusion}

\section{Authors' Contributions}

\section{Acknowledgements}

\section{Funding}


\section{Supporting Figures}











