\doublespacing

\section{Supporting Material}

This section includes all supplemental figures, descriptions of results relevant
to these supplemental figures, and fuller explanations of
methods which were briefly mentioned or referred to in
the manuscript. Another supplemental document describes,
in step-by-step fashion, how to use the computer scripts to
reproduce all figures and tables; it also
briefly desribes the purpose of the scripts and how to access the appropriate
data files. All of the scripts and data, except for the microarray data, are stored 
in a compressed folder that can be accessed at
\url{http://bme.virginia.edu/csbl/downloads-cdiff}.

\subsection{Dose response to TcdA cecal injection}

\begin{figure}[b!]
\centering
\includegraphics[width=\textwidth]{ini/supplement/doseresponse.pdf}
\caption{Histopathology 6h after cecal injection with TcdA.}
\label{fig:dose}
\end{figure}

To understand the potency of TcdA in a cecal injection system, a dose response
experiment was performed with histopathology as the measured outcome.
Five histopathology parameters were scored from zero to three, for a total possible score
between zero and fifteen (Methods). Excluding the largest dose (40 \textmu{}g),
TcdA dose was positively correlated with the parameters ``Architecture and epithelial erosions'', 
``Congestion and luminal exudates'', and
``Inflammation'' (Figure \ref{fig:dose}). The total histopathology score was similarly correlated.
The 20 \textmu{}g dose led to the highest scores; the scores with
the 40 \textmu{}g dose were similar to the scores for the 5 or 10 \textmu{}g dose.
A slight correlation between toxin dose and the ``Edema'' and ''Mucosal thickening''
parameters may be observed from Figure \ref{fig:dose}, yet this possible
correlation is unclear due to Sham mice scoring similar to all other toxin doses.
TcdA thus affects aspects of histopathology beginning at 5 \textmu{}g, the lowest
dose tested. Maximal effects occur with 20 \textmu{}g, the dose used in all our
other experiments.

\FloatBarrier

\subsection{Survival data from all cecal injection experiments}

\begin{table}[ht]
\centering
\begin{adjustwidth}{-1cm}{}
\resizebox{1.12\textwidth}{!}{%
\begin{tabular}{ c | c c c c | c c c c | c c c | c c }
   & \multicolumn{4}{c|}{2 hours} & \multicolumn{4}{c}{6 hours} & 
     \multicolumn{3}{|c|}{16 hours} & TcdA & TcdB \\
  Toxin: & A & B & A+B & Sham & A & B & A+B & Sham & A & B & Sham & Lot \# & Lot \# \\ \hline
  Dose response & --- & --- & --- & --- & 15/16* & --- & --- & 3/3 & --- & --- & --- & 0810123 & --- \\
  Exp. 1 & 3/3 & 5/5 & 3/3 & 3/3 & 3/3 & 3/3 & 1/3 & 3/4 & 3/3 & 2/3 & 3/3 & 0909101 & 0209165 \\
  Exp. 2 & 3/4 & 4/4 & --- & 4/4 & 3/4 & 3/4 & --- & 3/4 & 5/5 & 4/5 & 5/5 & 0909101 & 0209165\\
  Exp. 3 & --- & --- & --- & --- & --- & --- & --- & --- & 2/5 & 3/4 & 3/3 & 0810123 & 0209165\\ \hline
  Totals & 6/7 & 9/9 & 3/3 & 7/7 & 10/11 & 6/7 & 1/3 & 9/11 & 10/13 & 9/12 & 11/11 &  & \\
   & 86\% & 100\% & 100\% & 100\% & 91\% & 86\% & 33\% & 82\% & 77\% & 75\% & 100\% &  & \\
\end{tabular}
}
\end{adjustwidth}
\caption{Cecal injection experiments }
\label{tab:sizes}
\end{table}

In addition to the dose response experiment, three cecal injection experiments were performed, each on
different days (Table \ref{tab:sizes}). The ``15/16*'' in the table are the mice shown in
Figure \ref{fig:dose}. Only four of these mice were given 20 \textmu{}g, and
only these four are included in the ``Totals''. The one mouse that did not
survive in the dose response experiment was given 5\textmu{}g of TcdA.
The TECHLAB\textsuperscript{\textregistered{}} lot numbers 
of the toxins used are given in the last two columns. Mice used for microarrays
and all protein measurements were from experiments 1 and 2. The numbers in the tables
below give the fraction of survivors in each sample group (\# survivors/total mice).

\begin{table}[ht]
\centering
\begin{tabular}{ c | c c | c c | c c | c c | c }
   & \multicolumn{4}{c|}{6 hours} & \multicolumn{4}{c|}{16 hours} & TcdA  \\
   & \multicolumn{2}{c|}{Isotype} & \multicolumn{2}{c|}{Neutr. Abs.} &
     \multicolumn{2}{c|}{Isotype} & \multicolumn{2}{c|}{Neutr. Abs.} & Lot \\
   Toxin: & A & Sham & A & Sham & A & Sham & A & Sham & \# \\ \hline
   Exp. 4 & 2/4 & 2/2 & 4/4 & 2/2 & 2/4 & 2/2 & 4/4 & 2/2 & 0810123 \\
   Exp. 5 & 2/4 & 3/3 & 3/4 & 3/3 & --- & --- & --- & --- & 0810123 \\ \hline
   Totals & 4/8 & 5/5 & 7/8 & 5/5 & 2/4 & 2/2 & 3/4 & 2/2 &  \\
    & 50\% & 100\% & 88\% & 100\% & 50\% & 100\% & 75\% & 100\% &  \\
\end{tabular}
\caption{Cecal injection experiments with isotype antibody pretreatment}
\label{tab:absizes}
\end{table}   

Two cecal injection experiments were also performed with antibody 
preatreatment (Figure 4 of Manuscript). The sample sizes for TcdA and Sham
groups pretreated with isotype antibodies or anti-Cxcl1+anti-Cxcl2 (Neutr. Abs.) 
are shown in Table \ref{tab:absizes}.

\subsection{Cytopathic effects on a mouse, cecal, epithelial cell line}

\begin{figure}[b!]
\centering
\includegraphics[width=\textwidth]{ini/supplement/cytopathiceffects.png}
\caption{Cytopathic effects of TcdA and TcdB on a mouse epithelial cell line.}
\label{fig:cells}
\end{figure}

The cytopathic effects (e.g. cell rounding) of TcdA and TcdB were confirmed using an
immortalized, mouse, cecal epithelial cell line (Methods). These effects
were quantified by continuously measuring the impedance across the surface
of electrode-embedded wells (ACEA Biosciences). Similar methods and instrumentation
have been used previously in an ultrasensitive assay to detect TcdA and TcdB. \autocite{He2009}
Changes in impedance represent changes in
cell adherence, morphology, and number. Our visual observations of cell rounding
corresponded with changes in impedance. 

Figure \ref{fig:cells} shows impedance before and after toxin
addition. All readings are normalized to the impedance
at the time the toxin was added. Gray ribbons surrounding the lines represent
the standard deviation (n=2 per concentration, n=3 for control). If ribbons are not visible, 
this is because the replicates are highly similar. An impedance of zero
is the impedance of media before cells were seeded.

The minimum concentration of 
TcdA needed to alter an impedance profile (as shown in Figure \ref{fig:cells}) is
between 10 and 100 pg/ml (in a volume of 210.5 \textmu{}l); the concentration for TcdB is 
approximately 10 fg/ml. 
Hence, TcdB is at least 1,000 times more potent than TcdA
\emph{in vitro} with immortalized epithelial cells from a mouse cecum. Interestingly, our results
clearly show that TcdA more potently induces inflammation and
tissue damage \emph{in vivo}. 
The initial concentration in our cecal injection experiments was 20\textmu{}g/100\textmu{}l=200\textmu{}g/ml. 
Nevertheless, the purifications of TcdA and TcdB used
in this study affect mouse cells rapidly (e.g. $<$10min for TcdB at 100 ng/ml, Figure \ref{fig:cells}) at concentrations comparable to and even far lower than 
the initial concentrations in our \emph{in vivo} experiments.

\subsection{Altered blood cell distributions after toxin injection}

Complete blood counts indicated that the concentration of circulating
blood cells was altered during intoxication (Figure \ref{fig:cbcs}). 
Blood was drawn by cardiac puncture immediately after mice were euthanized.
Only measurements for mice which survived to the end point are represented
in the figure. Sufficient blood to measure complete blood counts was not 
capable of being drawn from every mouse. The mean corpuscular volume (MCV), 
mean corpuscular hemoglobin (MCH), MCH concentration (MCHC), red blood
cell distribution width (RDW), and mean platelet volume (MPV) were
similar across all three sample groups and are not shown.

\begin{figure}[ht]
\centering
\begin{adjustwidth}{-1cm}{}
\includegraphics{ini/supplement/cbcs.pdf}
\end{adjustwidth}
\caption{Complete blood counts in Sham, TcdA-, or TcdB-treated mice}
\label{fig:cbcs}
\end{figure}

\subsection{Immunohistochemstry: immune cell infiltration}

To determine if monocytes/macrophages, dendritic cells, and eosinophils
infiltrate the epithelial layer after cecal injection of toxin, we used
immunohistochemistry markers (Methods). Cecal tissue sections were scored 
by analyzing each section for the number of positive cells and overall
staining intensity.  Samples were assigned scores of + (few cells/weak staining), 
++ (moderate staining), or +++ (increased cell/intense staining). Table \ref{tab:ihc}
below shows the scores for Sham and TcdA-injected mice at 6 and 16 hours.

\begin{table}[ht]
\centering
\begin{tabular}{ r c c c c }
  Cell type: & Macrophages & Dendritic Cells & Eosinophils &  \\
  Marker: & F4/80 & Ly75 & Congo Red & \emph{n}\\ \hline
  Sham, 6h: & + & + & + & 4 \\
  TcdA, 6h: & +++ & ++ & + & 8 (4 for F4/80) \\
  Sham, 16h: & + & + & + & 4 \\
  TcdA, 16h: & +++ & +++ & + & 7 \\
\end{tabular}
\caption{Infiltration of immune cells 6h and 16h after cecal injection}
\label{tab:ihc}
\end{table}

\subsection{Cell type proportions in epithelial-layer cell isolation}

The cell type proportions within the epithelial-layer cell isolation
were determined by flow cytometry (Figure \ref{fig:flow}). Two
separate flow cytometry panels were used. For each sample, 50,000
events, or particles, were captured. 
For both panels, all particles were initially gated using pulse 
width and forward scatter-area to select for single 
cells, or singlets.  Events in this gate were then separated 
by forward scatter and side scatter to identify epithelial 
and non-epithelial fractions.  Epithelial cells were then 
validated using cytokeratin positivity in the first panel.  
The non-epithelial fractions from both panels were further characterized
with CD45, a leukocyte marker. Finally, the leukocytes were further characterized
using CD3 (a T cell marker), CD11b (a myeloid lineage marker), and 
B220 (a primarily B cell marker). The results are discussed in the manuscript
and shown in Figure \ref{fig:flow}.


\begin{figure}[h!]
  \centering
    \includegraphics[height=6in]{ini/supplement/flow.pdf}
  \caption{Cell type proportions as determined by flow cytometry}
  \label{fig:flow}
\end{figure}

\subsection{Comparisons of gene and protein expression of cytokines}

To test the common assumption that gene expression correlates
with protein expression, we compared the microarray data for chemokines to the
intracellular protein expression. After mapping
microarray probes to the appropriate Ensembl gene IDs, we found that TcdA-induced 
changes in gene expression correlated to changes in protein 
expression (r=0.67, Figure \ref{fig:geneprotein}A). Our data for Ccl4 is not shown
in Figure \ref{fig:geneprotein}A because the protein concentration measurements
were at or below the detection limit of our assay. The Ccl4 concentration from
three mice treated with toxin A were 0.8, 0.8, and 4.8 pg/ml; the concentrations
from three sham mice were 0.8, 0.8, and 14.72 pg/ml. Cell 
lysates were obtained as described in the Methods of the Manuscript.
Cytokine levels in serum and lysates were measured by using
MILLIPLEX\textsuperscript{\textregistered{}}
MAP beads and the signal was measured using a Luminex 100 IS System.

\begin{figure}[ht]
\centering
\includegraphics{ini/supplement/cytokines.pdf}
\caption{Cytokine gene and protein expression}
\label{fig:geneprotein}
\end{figure}

Figure \ref{fig:geneprotein}B compares our measurements of changes in the gene expression 
of cecal epithelial cells (Toxin A+B versus Sham at 6h; cecal injection) 
to protein concentration changes 
in mouse colonic tissue lysates as measured by
Hirota \emph{et al.} (Toxin A+B versus Sham at 4h; intrarectal instillation).\autocite{Hirota2012}
Figure \ref{fig:geneprotein}C compares cecal epithelial layer protein expression (Toxin A
versus Sham at 6h) to the same Hirota \emph{et al.} data.
Overall, the changes in the protein concentrations of colonic tissue were greater
than measurements of gene or protein expression from cecal cells. 

In addition to differences in the location of the intestine and 
differences in the biological molecules being measured, the 
Hirota \emph{et al.} samples were obtained from a different source, colonic tissue.
The colonic tissue presumably includes proteins from
intracellular and extracellular proteins, while many of our measurements included
only intracellular protein or mRNA. Nevertheless, Figures \ref{fig:geneprotein}B
and \ref{fig:geneprotein}C show correlation between the colonic tissue
data and cecal epithelial layer gene expression (r=0.5) and protein
expression (r=0.36). Additionally, our measurements of changes in blood serum
cytokines after 6h of TcdA injection correlated best with the data from 
Hirota \emph{et al.} (Figure \ref{fig:geneprotein}D).

\subsection{Inflammation and immune regulatory associated gene expression}

Figure 3C of the manuscript shows the expression change of inflammation associated
and immune regulatory genes. Figure \ref{fig:inflam} extends this figure to also
show TcdB-induced and TcdA-induced expression changes on separate, side-by-side
plots.

\begin{figure}[ht]
  \centering
    \includegraphics[height=7.4in]{ini/supplement/inflam.pdf}
  \caption{Expression changes of inflammation associated
           and immune regulatory genes}
  \label{fig:inflam}
\end{figure}


\subsection{Additional measurements after Cxcl1 \& Cxcl2 neutralization}

\subsubsection{\emph{Cxcl1} and \emph{Cxcl2} expression}

Aside from any possible feedback and regulatory mechanisms, i\@.p\@. administration of anti-Cxcl1 and
anti-Cxcl2 were predicted to neutralize extracellular proteins and not directly affect 
\emph{Cxcl1} and \emph{Cxcl2} expression. Using qRT-PCR to measure mRNA
in epithelial-layer cells, we found that antibody neutralization did not block the transient
increase in \emph{Cxcl1} and \emph{Cxcl2} expression that we previously 
measured by microarray (Figure \ref{fig:qrtpcr}). \emph{Actb} and \emph{Hprt}
were used as ``housekeeping'' genes.
The quantity of each transcript was calculated as $2^{-\Delta \Delta Ct}$ where
$\Delta \Delta Ct$ is the difference in cycle threshold between the transcript and the 
geometric mean of the housekeeping genes).

\begin{figure}[ht]
\centering
\includegraphics{ini/supplement/qrtpcr.pdf}
\caption{ \emph{Cxcl1} and \emph{Cxcl2} expression after Cxcl1 \& Cxcl2 neutralization }
\label{fig:qrtpcr}
\end{figure}


\subsubsection{MPO positive cell infiltration}

We hypothesized that neutralization of Cxcl1 and Cxcl2
would inhibit neutrophil infiltration in the cecum. We found that 
our neutralization did not affect the amount of infiltration after injection
of TcdA (Figure \ref{fig:neutralization}). The high variability for these 
limited samples, however,
makes it difficult to claim that isotype, anti-Cxcl1, and anti-Cxcl2 antibodies
do or do not have an effect on neutrophil infiltration 6h after cecal injection---at 
least with the method in which we administered the antibodies.

\begin{figure}[ht]
\centering
\includegraphics{ini/supplement/neutralization.pdf}
\caption{Neutrophil infiltration 6h after TcdA cecal injection}
\label{fig:neutralization}
\end{figure}


\subsection{\emph{In vivo} transcriptome response \emph{versus
            in vitro} transcriptome and proteome responses}

In previous studies with epithelial cell lines, the response 
to TcdA and/or TcdB has been analyzed with transcriptomic and 
proteomic techniques. In order to merge the data in this present study
to our previous data from a human ileocecal epithelial
cell line (HCT8 cells) treated with TcdA or TcdB, we mapped orthologous
mouse and human genes.\autocite{DAuria2012} These genes were further mapped
to human proteins in order to be compared to a recent proteomics study
by Zeiser \emph{et al.} who treated Caco-2 cells with TcdA.\autocite{Zeiser2013}

The fold changes of sample groups relative
to control groups were compared using nonparametric correlation
coefficients (Figure \ref{fig:correlations}). The values in cells of the figure
are the correlation coefficients squared (the square of the Pearson
correlation coefficient is the coefficient of determination, or $R^2$ value).
The colors of the cells correspond to the correlation coefficients. Negative
correlations are blue; positive correlations are red.
One-to-one orthologs were used to compare the two
transcriptional data sets (19,643 genes). The proteomics data
set was merged separately since data is available for many fewer
proteins than genes (4,090 gene-protein pairs).

\begin{figure}[ht]
\centering
\includegraphics{ini/supplement/correlations.pdf}
\caption{Correlations between \emph{in vitro} and \emph{in vivo} responses to toxins}
\label{fig:correlations}
\end{figure}

As seen above, there is little correlation between different studies
and experimental systems. In Figure \ref{fig:commons}, we show
13 of the most differentially expressed genes that were similarly 
expressed in both our \emph{in vivo} and \emph{in vitro} data sets.

\begin{figure}[ht]
\centering
\includegraphics{ini/supplement/commongenes.pdf}
\caption{Similarly expressed genes \emph{in vitro} and \emph{in vivo}}
\label{fig:commons}
\end{figure}

\subsection{Processing microarrays}

Multiple steps are involved in processing the raw data of microarrays
to a matrix of signal intensity values. We first 
introduce the usual steps and then describe the methods we chose
for each step.

\subsubsection{Possible steps in processing microarrays}

Except for probe set summarization, each one of the steps below is
optional. The order of some steps may be alternated, though this is not
typical. 

\begin{enumerate}
  \item \emph{Background correction}: Each array is corrected for background intensity.
  \item \emph{Probe level normalization}: All arrays are normalized to each 
  other based on the probe level data.
  \item \emph{Mismatch correction}: The signal intensity of perfect-match probes 
  is corrected based on the signal from corresponding mismatch probes.
  \item \emph{Probe set summarization}: The probes in each probe set are 
  summarized into one probe set signal intensity
\end{enumerate}

Each step has about four to ten different computational
methods that can be used. Many of the methods have several optional parameters.
Hence, the number of ways to process microarrays is vast.

\subsubsection{Steps taken in processing microarrays}

In the list below are the computatonal methods we chose for the microarray
processing steps described in the last section. 
Instead of presenting the lengthy comparative analysis of the many
sequences of methods that we examined, we provide only a brief description of either 
why a method was chosen or why others were not.

\begin{enumerate}
  \item \emph{RMA Background correction}\autocite{Irizarry2003}: 
  gcRMA\autocite{Wu2004} was not used since it
  has been shown to cause artificially high gene-gene correlation which
  would affect our gene set enrichment tests.\autocite{Lim2007}
  
  \item \emph{Cyclic loess probe level normalization}: Cyclic loess normalization was 
  used over quantile normalization because of quantile normalization's assumption
  that the probes on each microarray have the exact same 
  distribution.\autocite{Ballman2004} Invariant
  set normalization\autocite{LiWong2001b} caused some microarrays to become outliers, which we observed
  using principal components analysis. A second cyclic loess normalization was applied
  at the probe set level.
  
  \item \emph{Perfect match probes only}: Only perfect match probes were used.
  Although several model-based processing techniques have shown that mismatch
  probes provide useful information, the precise way in which mismatch probes should
  be used to correct for the signal from the corresponding perfect match 
  probes remains unclear.\autocite{Wang2007}
  
  \item \emph{Median polish probe set summarization}: Median polish was 
  used as opposed to Tukey's Bi-Weight algorithm to better account for 
  differences in probe affinities.\autocite{Tukey1977,AffymetrixWhitePaper}
    
\end{enumerate}

\subsection{Determining differentially expressed genes}\label{S:DE}

As with processing raw microarray data, there are many methods and options
for determining the significance of a probe set's change in expression between
two sample groups. We analyzed the results from many available methods, but only
briefly explain some of the findings that led us to the statistical test 
which we chose. Though permutation tests such as significance analysis of
microarrays\autocite{Tusher2001} are advantageous because they are less sensitive to outliers,
we did not use such tests because our low sample numbers did not allow
many permutations to be generated. We did not use simple fold change cutoffs because
they do not account for variation among replicates. We also considered limma's 
moderated t-statistic.\autocite{Smyth2004,Smyth2005} We found
that many of the significant probe sets identified by limma had low intensity values but very
small expression differences between sample groups. This was due to the near zero standard 
errors of some probe sets which we believed to occur for mostly technical,
not biological reasons. We therefore used cyberT\autocite{Baldi2001}, which, in practice,
adjusts the standard deviation for a probe set depending on the signal intensity
of that probe set.

The next problem is deciding what p-value cutoff to use for determining which
probe sets are ``significant''. To help with this, various ``p-value adjustments'' or
``FDR corrections'' use the p-value distribution to estimate the false discovery
rate (FDR). FDR corrections generate q values that describe the false discovery
rate for a group of genes. For instance, if the tenth ranked gene has a q value of 0.2, then
2 of the top ten genes should be false positives and the other 8 should be true
positives. We chose to use the Benjamini-Hochberg p-value adjustment\autocite{Benjamini1995}
instead of other more conservative methods such as the Bonferroni correction which
controls the family-wise error rate. This 
helps conceptually, yet still leaves an arbitrary choice of 
what q-value cutoff to use. We chose a cutoff of q\textless{}0.01 for mostly the practical
reason that the number of significant genes it led to was manageable.

Of the many statistical tests that we analyzed, we found that the number of
differentially expressed genes varied from test to test. However, for each test,
the number of differentially expressed genes for one comaparison relative to
another comparison was similar. For instance,
if statistical test $A$, relative to test $B$, found double the number of differentially expressed genes
for the TcdA versus sham comparison at 2h, then test $A$ also found about double the number of
differentially expressed genes in all other comparisons. Also,
the ranking of the genes was fairly consistent for different statistical tests.
Hence, in our next analysis of gene set enrichment, we sought enrichment
methods which use expression values or test statistics, not those that require 
genes to be set as ``differentially expressed'' or ``not differentially expressed''.

\subsection{Gene set enrichment}

As briefly mentioned in the manuscript, gene set enrichment methods help to 
understand the biological functions associated with changes in gene expression
data. To define gene sets, we used gene function annotations from the Gene 
Ontology and Reactome databases.

Much of the bioinformatics literature has categorized
different enrichment methods by the type of hypothesis tested.\autocite{Ackermann2009}

\subsubsection{Competitive}

One possible hypothesis tested in the ``competitive'' gene set enrichment method
CAMERA, developed by Wu \emph{et al.}, is whether the mean t-statistic for the 
genes in a gene set is significantly different than the genes not in the 
gene set.\autocite{Wu2012} In other words, it tests whether or not the genes in a gene set are 
more differentially expressed than the genes outside of the gene set. As 
this type of test compares the differential expression of two competing 
groups of genes (genes in the set versus not in the set), it has come to 
be named a ``competitive'' test. CAMERA was written to use the moderated t-statistic 
from limma.\autocite{Smyth2005} We wrote a modified version of CAMERA to instead use
the cyberT t statistic\autocite{Baldi2001}, the same statistic used in our differential gene
expression analysis.

\subsubsection{Self-contained}

Another important and common null hypothesis is that the genes in the gene set are 
not differentially expressed at all. Since this type of test requires only data 
from genes in the gene set, it has been coined a ``self-contained'' test. 
Our self-contained method tests whether the avearge t-statistic for the genes 
in a gene set is zero.

To describe our self-contained test, we use the 
notations from the CAMERA manuscript.\autocite{Wu2012} Consider a gene-wise statistic, $z_i$ for gene $i$ in
a gene set with $m$ genes. In our self-contained test, this statistic is the 
standard normal deviate which has the
same quantile as the cyberT t-statistic; a similar transformation is also 
performed with CAMERA. Under the null hypothesis that
the average of the gene expression differences between the two sample groups is zero, 
the gene set's mean test statistic, $\bar{z}$, is normally distributed with
$\mu{}=0$ and

\begin{equation}
Var( \bar{z} ) = \frac{1}{m^2} 
                 \left(   
                   \sum\limits_{i=1}^m \sigma{}_i^2 + 
                   \sum\limits_{i<j} \sigma{}_i \sigma{}_j \rho_{ij}
                 \right).
\end{equation}

Since $z_i \sim \mathcal{N}(0,1)$ for all $i$,

\begin{equation}
Var( \bar{z} ) = \frac{1}{m}  + \frac{m-1}{m}\bar{\rho}
\end{equation}

where $\bar{\rho}$ is the average correlation between genes in the gene set. Note for
large $m$, $Var(z) \approx \bar{\rho}$. If $\bar{z_k}$ is the 
mean test statistic for gene set $k$ calculated from the data, then a p-value
is calculated using location of $\bar{z_k}$ in the cumulative distribution 
function of $\bar{z}$. This novel method improves upon another published, parametric gene
set enrichment method, namely PAGE.\autocite{Kim2005}









